% FILE: main.tex  Version 2.1
% AUTHOR:
% Universit�t Duisburg-Essen, Standort Duisburg
% AG Prof. Dr. G�nter T�rner
% Verena Gondek, Andy Braune, Henning Kerstan
% Fachbereich Mathematik
% Lotharstr. 65., 47057 Duisburg
% entstanden im Rahmen des DFG-Projektes DissOnlineTutor
% in Zusammenarbeit mit der
% Humboldt-Universitaet zu Berlin
% AG Elektronisches Publizieren
% Joanna Rycko
% und der
% DNB - Deutsche Nationalbibliothek

\chapter{Zusammenfassung}

Eine geriatrisch-onkologische Konferenz ist am besten geeignet, die
Therapief�higkeit geriatrisch-onkologischer Patienten zu
beurteilen. Anhand der Daten eines umfassenden geriatrischen
Assessments von �ber 300 Patienten, die an dieser Klinik 2014 und 2015
routinem��ig bei allen station�ren Patienten, die 65 Jahre oder �lter
waren, vor Einleitung einer spezifischen Therapie erhoben wurden, wird
gezeigt, dass die Entscheidung einer geriatrisch-onkologischen
Konferenz entscheidend von der Klassifizierung aufgrund des
Assessments aber auch von der pers�nlichen Einsch�tzung des Behandlers
abweicht. Zus�tzlich wird analysiert inwiefern ein statistisches Modell in
der Lage ist, anhand der Assessmentdaten die Klassifizierung, die
innerhalb einer geriatrisch-onkologischen Konferenz getroffen wird, zu
modellieren.

\begin{enumerate}

\item Analyse der Daten des geriatrischen Assessments

\item Unterschiede der pers�nlichen Einsch�tzung des Behandlers, der
Auswertung des geriatrischen Assessments und der Konferenzentscheidung
auf Grundlage des Assessments

\item Analyse eines Statistisches Modells der Konferenzentscheidung -
Vorhersage der Konferenzentscheidung

\begin{itemize}

\item als logistische Regression
\item Diskriminanzanalyse
\item als KNN
\item als neuronales Netz

\end{itemize}

\end{enumerate}
