% FILE: main.tex  Version 2.1
% AUTHOR:
% Universit�t Duisburg-Essen, Standort Duisburg
% AG Prof. Dr. G�nter T�rner
% Verena Gondek, Andy Braune, Henning Kerstan
% Fachbereich Mathematik
% Lotharstr. 65., 47057 Duisburg
% entstanden im Rahmen des DFG-Projektes DissOnlineTutor
% in Zusammenarbeit mit der
% Humboldt-Universitaet zu Berlin
% AG Elektronisches Publizieren
% Joanna Rycko
% und der
% DNB - Deutsche Nationalbibliothek

\chapter{Zusammenfassung}

Onkologen entscheiden aufgrund des Alters und des klinischen Eindrucks
t�glich welche Patienten welche Chemotherapie erhalten sollen. Bei
�lteren Patienten sind diese Entscheidungen aufgrund fehlender Evidenz
von der pers�nlichen Erfahrung und Einstellung abh�ngig. Ein
umfassendes geriatrisches Assessment wurde vorgeschlagen, um diese
Entscheidung datenbasiert und reproduzierbar f�llen zu k�nnen. Beide
Ans�tze haben zahlreiche Vor- und auch Nachteile. In der vorliegenden
Arbeit wird eine Kombination beider Ans�tze beschrieben und
untersucht. Bei dieser Kombination wird innerhalb einer
geriatrisch-onkologischen Konferenz auf der Datengrundlage eines
umfassenden geriatrischen Assessments unter Einbeziehung des
Behandlers, eines erfahrenen Onkologen und eines Geriaters sowie der
das Assessment durchf�renden Berufsgruppen eine gemeinsame
Entscheidung getroffen.

\begin{itemize}
\item Wie unterscheidet sich die Entscheidung der
  geriatrisch-onkologischen Konferenz von der pers�nlichen
  Entscheidung des Behandlers und der Einsch�tzung allein aufgrund des
  Assessments?
\item Wie h�ngt die Entscheidung der geriatrisch-onkologischen
  Konferenz von den einzelnen Kategorien des Assessments und der
  Einsch�tzung des Behandlers ab?
\item Ist durch ein statistisches Modell die Entscheidung der
  geria\-trisch"=onkolo\-gischen Konferenz vorhersagbar? Wie verl�sslich
  ist eine solche Vorhersage?
\end{itemize}

Eine geriatrisch-onkologische Konferenz ist am besten geeignet, die
Therapief�higkeit geriatrisch-onkologischer Patienten zu
beurteilen. Anhand der Daten eines umfassenden geriatrischen
Assessments von �ber 300 Patienten, die an dieser Klinik 2014 und 2015
routinem��ig bei allen station�ren Patienten, die 65 Jahre oder �lter
waren, vor Einleitung einer spezifischen Therapie erhoben wurden, wird
gezeigt, dass die Entscheidung einer geriatrisch-onkologischen
Konferenz entscheidend von der Klassifizierung aufgrund des
Assessments aber auch von der pers�nlichen Einsch�tzung des Behandlers
abweicht. Zus�tzlich wird analysiert inwiefern ein statistisches Modell in
der Lage ist, anhand der Assessmentdaten die Klassifizierung, die
innerhalb einer geriatrisch-onkologischen Konferenz getroffen wird, zu
modellieren.

\begin{enumerate}

\item Analyse der Daten des geriatrischen Assessments

\item Unterschiede der pers�nlichen Einsch�tzung des Behandlers, der
Auswertung des geriatrischen Assessments und der Konferenzentscheidung
auf Grundlage des Assessments

\item Analyse eines logistischen Regression der Konferenzentscheidung -
Vorhersage der Konferenzentscheidung

\end{enumerate}
