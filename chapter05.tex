% FILE: main.tex  Version 2.1
% AUTHOR:
% Universit�t Duisburg-Essen, Standort Duisburg
% AG Prof. Dr. G�nter T�rner
% Verena Gondek, Andy Braune, Henning Kerstan
% Fachbereich Mathematik
% Lotharstr. 65., 47057 Duisburg
% entstanden im Rahmen des DFG-Projektes DissOnlineTutor
% in Zusammenarbeit mit der
% Humboldt-Universitaet zu Berlin
% AG Elektronisches Publizieren
% Joanna Rycko
% und der
% DNB - Deutsche Nationalbibliothek

\chapter{Zusammenfassung}

Onkologen entscheiden auf der Basis von Alter und klinischem Eindruck
t�glich, welche Patienten welche Chemotherapie erhalten sollen. Bei
�lteren Patienten sind diese Entscheidungen aufgrund fehlender Evidenz
von der pers�nlichen Erfahrung und "`Behandlungsphilosophie"' des
jeweils behandelnden Onkologen abh�ngig. Ein
umfassendes geriatrisches Assessment wurde vorgeschlagen, um diese
Entscheidung datenbasiert und reproduzierbar treffen zu k�nnen. Beide
Strategien haben Vor- und Nachteile. In der vorliegenden
Arbeit wird eine Synthese beider Ans�tze beschrieben und weiter
untersucht. Bei der Kombination beider Ans�tze wird innerhalb einer
geriatrisch-onkologischen Konferenz auf der Datengrundlage eines
umfassenden geriatrischen Assessments und zus�tzlich unter
Einbeziehung des Behandlers, eines erfahrenen Onkologen und eines
Geriaters sowie der das Assessment durchf�hrenden Berufsgruppen eine
gemeinsame Entscheidung getroffen.

Anhand der Daten eines umfassenden geriatrischen
Assessments von �ber 400 Patienten, die an dieser Klinik 2014 und 2015
routinem��ig bei allen station�ren Patienten, die 65 Jahre oder �lter
waren, vor Einleitung einer spezifischen Therapie erhoben wurden, wurde
gezeigt, dass die Entscheidung einer geriatrisch-onkologischen
Konferenz entscheidend von der Klassifizierung allein aufgrund des
Assessments, aber auch von der pers�nlichen Einsch�tzung des
Behandlers abweicht. Es wurde ein statistisches Modell erstellt,
welches anhand der Assessmentdaten die Klassifizierung von Patienten
in einer geriatrisch-onkologischen Konferenz vorhersagt. 

Bis zur weiteren prospektiven Validierung und Bewertung der denkbaren
Entscheidungsstrategien ist eine geriatrisch-onkologische Konferenz am
besten geeignet, die Therapief�higkeit geriatrisch-onkologischer
Patienten zu beurteilen, da sie alle verf�gbaren Informationen
ber�cksichtigen und damit die maximale Patientensicherheit
gew�hrleisten kann.

