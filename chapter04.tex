% FILE: main.tex  Version 2.1
% AUTHOR:
% Universit�t Duisburg-Essen, Standort Duisburg
% AG Prof. Dr. G�nter T�rner
% Verena Gondek, Andy Braune, Henning Kerstan
% Fachbereich Mathematik
% Lotharstr. 65., 47057 Duisburg
% entstanden im Rahmen des DFG-Projektes DissOnlineTutor
% in Zusammenarbeit mit der
% Humboldt-Universitaet zu Berlin
% AG Elektronisches Publizieren
% Joanna Rycko
% und der
% DNB - Deutsche Nationalbibliothek

\chapter{Diskussion}

\section{Pers�nliche Einsch�tzung des Behandlers}

\section{Modell der Konferenz als logistische Regression}

\section{Simulation der Konferenz als Web-basierte Abfrage}

\section{Wie geht es weiter, weitere Analysen, Studie planen}

\section{Empfehlungen f�r die t�gliche Praxis}

\begin{itemize}

\item Alle Patienten, 65 Jahre oder �lter, f�r die eine potentiell
  nebenswirkungsreiche Therapie geplant ist (Chirurgie,
  Strahlentherapie, medikament�se Therapie), sollen auf das Risiko von
  funktionellen Einschr�nkungen und geriatrischen Syndromen gescreent
  werden (G8 oder VES-13 oder in Kombination).

\item Bei station�ren Patienten oder auff�lligem Screening soll ein
  umfassendes geriatrisches Assessment (CGA) durch ein erfahrenes
  multiprofessionelles Team stattfinden.

\item Die Ergebnisse dieses Assessments sollen in einer Konferenz von
  einem multiprofessionellen Team bestehend aus Ergotherapie, Pflege,
  Geriatrie, Onkologie besprochen werden.

\item Basierend auf der Konferenz sollen Ma�nahmen eingeleitet werden,
  um funktionelle, psychische, soziale oder medizinische Probleme zu bessern.

\item Basierend auf der Konferenz soll die weiterf�hrende Therapie
  angepasst an die individuelle Leistungsf�higkeit und Lebenssituation
  geplant werden.

\item Das geriatrische Assessment kann zur Verlaufsbeobachtung w�hrend
  und nach der Therapie regelm��ig wiederholt werden, um neue Probleme
  zu entdecken und auch ggf. eine Besserung bestehender Probleme zu
  dokumentieren.

\item Die Ergebnisse der Assessments und der Konferenzen sollen dem
  Patienten, den Angeh�rigem und dem Hausarzt mitgeteilt werden.

\end{itemize}