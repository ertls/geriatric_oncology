% FILE: main.tex  Version 2.1
% AUTHOR:
% Universit�t Duisburg-Essen, Standort Duisburg
% AG Prof. Dr. G�nter T�rner
% Verena Gondek, Andy Braune, Henning Kerstan
% Fachbereich Mathematik
% Lotharstr. 65., 47057 Duisburg
% entstanden im Rahmen des DFG-Projektes DissOnlineTutor
% in Zusammenarbeit mit der
% Humboldt-Universitaet zu Berlin
% AG Elektronisches Publizieren
% Joanna Rycko
% und der
% DNB - Deutsche Nationalbibliothek

\chapter{Einleitung}

\section{Inhalt und Ziel}

In der folgenden Arbeit werden erstens die Daten eines umfassenden
geriatrischen Assessments von �ber 300 Patienten vorgestellt, die an
dieser Klinik 2014 und 2015 routinem��ig bei allen station�ren
Patienten, die 65 Jahre oder �lter waren, vor Einleitung einer
spezifischen Therapie erhoben wurden. Es wird heutzutage gefordert,
wichtige Therapieentscheidungen eher aufgrund dieses Assessments zu
treffen, als aufgrund der pers�nlichen Einsch�tzung des Behandlers.
Zweitens wird analysiert, inwiefern die Entscheidung, die anhand der erhobenen Daten in einer
geriatrisch-onkologischen Konferenz getroffen wird, von der
pers�nlichen Einsch�tzung des behandelnden Arztes und von der
Klassifizierung durch das Assessment �bereinstimmt oder
abweicht. Drittens wird gepr�ft inwiefern ein statistisches Modell in
der Lage ist, anhand der Assessmentdaten die Klassifizierung, die
innerhalb einer geriatrisch-onkologischen Konferenz getroffen wird, zu
modellieren.

\begin{enumerate}

\item Analyse der Daten des geriatrischen Assessments

\item Unterschied der pers�nlichen Einsch�tzung des Behandlers, der
Auswertung des geriatrischen Assessments und der Konferenzentscheidung
anhand des Assessments

\item Analyse eines Statistisches Modells der Konferenzentscheidung -
Vorhersage der Konferenzentscheidung

\begin{itemize}

\item als logistische Regression
\item Diskriminanzanalyse
\item als KNN
\item als neuronales Netz

\end{itemize}

\end{enumerate}
