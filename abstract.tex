% FILE: abstract.tex  Version 2.1
% AUTHOR:
% Universit�t Duisburg-Essen, Standort Duisburg
% AG Prof. Dr. G�nter T�rner
% Verena Gondek, Andy Braune, Henning Kerstan
% Fachbereich Mathematik
% Lotharstr. 65., 47057 Duisburg
% entstanden im Rahmen des DFG-Projektes DissOnlineTutor
% in Zusammenarbeit mit der
% Humboldt-Universitaet zu Berlin
% AG Elektronisches Publizieren
% Joanna Rycko
% und der
% DNB - Deutsche Nationalbibliothek

%-englische-Zusammenfassung---------------------------------------
\selectlanguage{english}
\begin{abstract}

\subsection*{The Geriatric-Oncologic Conference, a new approach in
de\-cision-making}

\textbf{Background}\\
A comprehensive geriatric assessment (CGA) is recommended before
treating elderly cancer patients. However, it is not proven that
additional information from the CGA will change our treatment
decision.

\textbf{Methods}\\
421 cancer patients, 65 years or older, were judged by their treating
oncologist regarding fitness for chemotherapy. Accompanying a CGA was
performed and each patient was discussed in a multidisciplinary board
(MB) including a geriatrician. The differences between the judgement
of the treating oncologist, the MB, and a classification based solely
on the CGA were examined. Additionally a statistical model of the
decision making process within the MB, based on the findings of the
CGA was established and evaluated.

\textbf{Results}\\
Treating oncologist and MB judged 12\% and 15\% of the patients as
frail, 41\% and 38\% as vulnerable, 46\% and 47\% as fit. 83\% of
congruence was observed. Based on the proposal of Balducci, 55\% of
the patients were classified as frail, 30\% as vulnerable and 15\% as
fit. 34\% of congruence with treating oncologist judgement was
observed. In the 2-stage logistic model the activities of daily living
and the mini mental state examination (MMSE) discriminated between
frail and vulnerable or fit. Tinetti test, age, Charlson comorbidity
index, living alone, MMSE and mini nutritional assessment
discriminated between vulnerable and fit. The statistical models were
able to differentiate with an accuracy of 95\% between frail and
vulnerable or fit and 83\% between vulnerable and fit.

\textbf{Conclusions}\\
To our experience, the judgement of an experienced oncologist is well
comparable with the judgement of a MB. Nevertheless, for some patients
discussion of CGA data in the MB may essentially change treatment
decisions. A logistic regression model of the decision making process
within the MB may replace the elaborate team discussion, if a
conference is not feasible.

\end{abstract}

%-deutsche Zusammenfassung----------------------------------------
\selectlanguage{ngerman}
