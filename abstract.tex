% FILE: abstract.tex  Version 2.1
% AUTHOR:
% Universit�t Duisburg-Essen, Standort Duisburg
% AG Prof. Dr. G�nter T�rner
% Verena Gondek, Andy Braune, Henning Kerstan
% Fachbereich Mathematik
% Lotharstr. 65., 47057 Duisburg
% entstanden im Rahmen des DFG-Projektes DissOnlineTutor
% in Zusammenarbeit mit der
% Humboldt-Universitaet zu Berlin
% AG Elektronisches Publizieren
% Joanna Rycko
% und der
% DNB - Deutsche Nationalbibliothek

%-englische-Zusammenfassung---------------------------------------
\selectlanguage{english}
\begin{abstract}
Here is the english abstract. 				
\end{abstract}

%-deutsche Zusammenfassung----------------------------------------
\selectlanguage{ngerman}
\begin{abstract}
Onkologen entscheiden aufgrund des Alters und des klinischen Eindrucks
t�glich welche Patienten welche Chemotherapie erhalten sollen. Bei
�lteren Patienten sind diese Entscheidungen aufgrund fehlender Evidenz
von der pers�nlichen Erfahrung und Einstellung abh�ngig. Ein
umfassendes geriatrisches Assessment wurde vorgeschlagen, um diese
Entscheidung datenbasiert und reproduzierbar f�llen zu k�nnen. Beide
Ans�tze haben zahlreiche Vor- und auch Nachteile. In der vorliegenden
Arbeit wird eine Kombination beider Ans�tze beschrieben und
untersucht. Bei dieser Kombination wird innerhalb einer
geriatrisch-onkologischen Konferenz auf der Datengrundlage eines
umfassenden geriatrischen Assessments unter Einbeziehung des
Behandlers, eines erfahrenen Onkologen und eines Geriaters sowie der
das Assessment durchf�renden Berufsgruppen eine gemeinsame
Entscheidung getroffen.

\begin{itemize}
\item Wie unterscheidet sich die Entscheidung der
  geriatrisch-onkologischen Konferenz von der pers�nlichen
  Entscheidung des Behandlers und der Einsch�tzung allein aufgrund des
  Assessments?
\item Wie h�ngt die Entscheidung der geriatrisch-onkologischen
  Konferenz von den einzelnen Kategorien des Assessments und der
  Einsch�tzung des Behandlers ab?
\item Ist durch ein statistisches Modell die Entscheidung der
  geriatrisch-onkologischen Konferenz vorhersagbar? Wie verl�sslich
  ist eine solche Vorhersage?
\end{itemize}

\end{abstract}
