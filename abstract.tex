% FILE: abstract.tex  Version 2.1
% AUTHOR:
% Universit�t Duisburg-Essen, Standort Duisburg
% AG Prof. Dr. G�nter T�rner
% Verena Gondek, Andy Braune, Henning Kerstan
% Fachbereich Mathematik
% Lotharstr. 65., 47057 Duisburg
% entstanden im Rahmen des DFG-Projektes DissOnlineTutor
% in Zusammenarbeit mit der
% Humboldt-Universitaet zu Berlin
% AG Elektronisches Publizieren
% Joanna Rycko
% und der
% DNB - Deutsche Nationalbibliothek

%-englische-Zusammenfassung---------------------------------------
\selectlanguage{english}
\begin{abstract}

\subsection*{The Geriatric-Oncologic Conference, a new approach in
de\-cision-making}

\textbf{Background}\\
A comprehensive geriatric assessment (CGA) is recommended before
treating elderly cancer patients. However, it is not proven that
additional information from the CGA will change our treatment
decision.

\textbf{Methods}\\
421 cancer patients, 65 years or older, were judged by their treating
oncologist regarding fitness for chemotherapy. Accompanying a CGA was
performed and each patient was discussed in a multidisciplinary board
(MB) including a geriatrician. The differences between the judgements
of the treating oncologist, the MB, and a classification based solely
on the CGA were examined. Additionally a statistical model of the
decision making process within the MB, based on the findings of the
CGA was established and evaluated.

\textbf{Results}\\
Treating oncologist and MB judged 12\% and 15\% of the patients as
frail, 41\% and 38\% as vulnerable, 46\% and 47\% as fit. 83\% of
congruence was observed. Based on the proposal of Balducci, 55\% of
the patients were classified as frail, 30\% as vulnerable and 15\% as
fit. 34\% of congruence with treating oncologist judgement was
observed. In the 2-stage logistic model the activities of daily living
and the mini mental state examination (MMSE) discriminated between
frail and vulnerable or fit. Tinetti test, age, Charlson comorbidity
index, living alone, MMSE and mini nutritional assessment
discriminated between vulnerable and fit. The statistical models were
able to differentiate with an accuracy of 95\% between frail and
vulnerable or fit and 83\% between vulnerable and fit.

\textbf{Conclusions}\\
To our experience, the judgement of an experienced oncologist is well
comparable with the judgement of a MB. Nevertheless, for some patients
discussion of CGA data in the MB may essentially change treatment
decisions. A logistic regression model of the decision making process
within the MB may replace the elaborate team discussion, if a
conference is not feasible.

\end{abstract}

%-deutsche Zusammenfassung----------------------------------------
\selectlanguage{ngerman}

\section*{Abstract}

\textbf{Hintergrund}\\
Vor der Behandlung geriatrisch-onkologischer Patienten sollte ein
umfassendes geriatrisches Assessment (CGA) durchgef�hrt werden. Es ist
jedoch unklar, inwieweit die erhobenen Zusatzinformationen die
Behandlungsentscheidung beeinflussen.

\textbf{Methoden}\\
421 onkologische Patienten, ab 65 Jahren, wurden durch ihren
behandelnden Onkologen bez�glich der Durchf�hrbarkeit einer
Chemotherapie eingesch�tzt. Begleitend wurde ein Assessment
durchgef�hrt und die Ergebnisse in einer interdisziplin�ren Konferenz
an der auch ein Geriater beteiligt war, besprochen. Die Unterschiede
beider Einsch�tzungsm�glichkeiten und einer Klassifikation, die allein
auf die Ergebnisse des Assessments beruht, wurden
untersucht. Zus�tzlich wurde ein statistisches Modell des
Entscheidungsprozesses der interdisziplin�ren Konferenz erstellt und
analysiert.

\textbf{Ergebnisse}\\
Die behandelnden Onkologen und die interdisziplin�re Konferenz
beurteilten 12\% und 15\% der Patienten als nicht therapief�hig, 41\%
und 38\% als eingeschr�nkt therapief�hig und 46\% und 47\% als
uneingeschr�nkt therapief�hig. In 83\% der F�lle wurde gleich
entschieden. Nach dem Vorschlag von Balducci wurden 55\% der
Patienten als nicht therapief�hig, 30\% als eingeschr�nkt
therapief�hig und 15\% als uneingeschr�nkt therapief�hig
klassifiziert. In 34\% der F�lle herrschte �bereinstimmung mit
der Beurteilung des behandelnden Onkologen. In einer zweistufigen
logistischen Regression unterschieden der Barthel-Index und der
Mini-Mental-Test zwischen nicht therapief�hig und
therapief�hig. Tinetti-Test, Alter, Charlson-Komorbidit�tsindex,
Mini-Mental-Test und Ern�hrungsassessment unterschieden zwischen
eingeschr�nkt und uneingeschr�nkt therapief�hig. Die Modelle
prognostizierten die Konferenzentscheidung in
95\% der F�lle bez�glich nicht therapief�hig und therapief�hig und
83\% der F�llt bez�glich eingeschr�nkt oder uneingeschr�nkt
therapief�hig korrekt.

\textbf{Schlussfolgerung}\\
Nach unserer Erfahrung ist die Beurteilung eines erfahrenen Onkologen
durchaus vergleichbar mit der Beurteilung einer interdisziplin�ren
Konferenz. Allerdings k�nnen f�r einzelne Patienten die in einem
Assessment erhobenen Zusatzinformationen durchaus
entscheidungsrelevant sein. Ein statistisches Modell k�nnte
gegebenenfalls die aufw�ndige Entscheidungsfindung in einer
interdisziplin�re Konferenz ersetzen, wenn diese nicht m�glich ist.
