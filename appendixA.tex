% FILE: appendixA.tex  Version 2.1
% AUTHOR:
% Universit�t Duisburg-Essen, Standort Duisburg
% AG Prof. Dr. G�nter T�rner
% Verena Gondek, Andy Braune, Henning Kerstan
% Fachbereich Mathematik
% Lotharstr. 65., 47057 Duisburg
% entstanden im Rahmen des DFG-Projektes DissOnlineTutor
% in Zusammenarbeit mit der
% Humboldt-Universitaet zu Berlin
% AG Elektronisches Publizieren
% Joanna Rycko
% und der
% DNB - Deutsche Nationalbibliothek

\chapter{Anhang}

\section{Abk�rzungen}

%\begin{longtable}[l]{p{2.5cm}p{10cm}} alte Implementierung jetzt
%Versuch Acronym Package:

%ADL & Aktivit�ten des t�glichen Lebens, Bartel-Index\\
%BMI & Body Mass Index\\
%CGA & Comprehensive geriatric assessment, umfassendes geriatrisches Assessment\\
%CLL & Chronisch lymphatische Leuk�mie\\
%cm & Zentimeter\\
%DemTect & Demenz-Detektion\\
%DIA & Depression im Alter\\
%ECOG & Eastern Cooperative Oncology Group\\
%GDS & Geriatric Depression Score\\
%IADL & Instrumentelle Aktivit�ten des t�glichen Lebens\\
%kg & Kilogramm\\
%m & Meter\\
%MMST & Mini Mental Status Test\\
%MNA & Mini Nutritional Assessment\\
%NA & Not available (fehlende Variable)\\
%WHO & World Healt Organisation\\
%PS & Performance Status\\
%s & Sekunden\\
%TUG & Timed up and go\\
%\end{longtable}

\begin{acronym}
  \acro{ADL}{Aktivit�ten des t�glichen Lebens}
  \acro{BMI}{Body Mass Index}
  \acro{CGA}{Comprehensive geriatric assessment}
  \acro{CLL}{Chronisch lymphatische Leuk�mie}
  \acro{cm}{Zentimeter}
  \acro{DemTect}{Demenz-Detektion}
  \acro{DIA}{Depression im Alter}
  \acro{ECOG}{Eastern Cooperative Oncology Group}
  \acro{GDS}{Geriatric Depression Score}
  \acro{IADL}{Instrumentelle Aktivit�ten des t�glichen Lebens}
  \acro{kg}{Kilogramm}
  \acro{m}{Meter}
  \acro{MMST}{Mini Mental Status Test}
  \acro{MNA}{Mini Nutritional Assessment}
  \acro{NA}{Not available}
  \acro{WHO}{World Health Organisation}
  \acro{PS}{Performance Status}
  \acro{s}{Sekunden}
  \acro{TUG}{Timed up and go}
\end{acronym}

\section{Quelldateien}

\url{https://github.com/ertls/geriatric_oncology}

\section{Quelldokumente}

\begin{figure}[htbp]
  \centering
  \includegraphics[width=\textwidth]{Anhang/Information.pdf}
  \caption{Einverst�ndniserkl�rung}
  \label{fig:einverstaendnis}
\end{figure}

\begin{figure}[htbp]
  \centering
  \includegraphics[width=\textwidth]{Anhang/Ethik.pdf}
  \caption{Votum der Ethikkommission}
  \label{fig:ethik}
\end{figure}

\begin{figure}[htbp]
  \centering
  \includegraphics[width=\textwidth]{Anhang/Barthel.pdf}
  \caption{Barthel-Index}
  \label{fig:barthel}
\end{figure}

\begin{figure}[htbp]
  \centering
  \includegraphics[width=\textwidth]{Anhang/IADL.pdf}
  \caption{IADL-Skala}}
  \label{fig:iadl}
\end{figure}

\begin{figure}[htbp]
  \centering
  \includegraphics[width=\textwidth]{Anhang/TUG.pdf}
  \caption{TUG-Test}
  \label{fig:tug}
\end{figure}

\begin{figure}[htbp]
  \centering
  \includegraphics[width=\textwidth]{Anhang/Tinetti.pdf}
  \caption{Tinetti-Test}
  \label{fig:tinetti}
\end{figure}

\begin{figure}[htbp]
  \centering
  \includegraphics[width=\textwidth]{Anhang/MMST.pdf}
  \caption{MMS-Test}
  \label{fig:mmst}
\end{figure}

\begin{figure}[htbp]
  \centering
  \includegraphics[width=\textwidth]{Anhang/DemTect.pdf}
  \caption{DemTect}
  \label{fig:demtect}
\end{figure}

\begin{figure}[htbp]
  \centering
  \includegraphics[width=\textwidth]{Anhang/Charlson.pdf}
  \caption{Charlson-Score}
  \label{fig:charlson}
\end{figure}

\begin{figure}[htbp]
  \centering
  \includegraphics[width=\textwidth]{Anhang/GDS.pdf}
  \caption{GDS}
  \label{fig:gds}
\end{figure}

\begin{figure}[htbp]
  \centering
  \includegraphics[width=\textwidth]{Anhang/DIA.pdf}
  \caption{DIA-Skala}
  \label{fig:dia}
\end{figure}

\begin{figure}[htbp]
  \centering
  \includegraphics[width=\textwidth]{Anhang/MNA.pdf}
  \caption{MNA}
  \label{fig:mna}
\end{figure}
