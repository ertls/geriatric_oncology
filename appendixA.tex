% FILE: appendixA.tex  Version 2.1
% AUTHOR:
% Universit�t Duisburg-Essen, Standort Duisburg
% AG Prof. Dr. G�nter T�rner
% Verena Gondek, Andy Braune, Henning Kerstan
% Fachbereich Mathematik
% Lotharstr. 65., 47057 Duisburg
% entstanden im Rahmen des DFG-Projektes DissOnlineTutor
% in Zusammenarbeit mit der
% Humboldt-Universitaet zu Berlin
% AG Elektronisches Publizieren
% Joanna Rycko
% und der
% DNB - Deutsche Nationalbibliothek

\chapter{Quelldateien}

Wie in Abschnitt~\ref{sec:reproducability} beschrieben ist jede
wissenschaftliche Forschung ohne die M�glichkeit der
Reproduzierbarkeit sowohl der vorgestellten Daten als auch der
gefolgerten Ergebnisse vergebens. Aus diesem Grund wurde in dieser
Arbeit die bestm�gliche Dokumentation der Herkunft der Daten, der
durchgef�hrten Berechnungen und der daraus abgeleiteten Ergebnisse
angestrebt. Alle von mir erstellten und verwendeten Quelldateien sind
im Internet unter der Adresse
\url{https://github.com/ertls/geriatric_oncology} einsehbar und mit
entsprechender Kennzeichnung der Herkunft weiter verwendbar. Enhalten
sind die R-Skripte, die zur Aufbereitung der Daten und zur
anschlie�enden statistischen Analyse verwendet wurden und die
Latex-Dateien, aus denen das vorliegende Dokument erstellt wurde. Aus
datenschutzrechtlichen Bedenken habe ich mich trotz Anonymisierung der
Patientendaten gegen ein Ver�ffentlichung des verwendeten Datensatzes
entschieden. Sollte ein begr�ndetes, wissenschaftliches Interesse an
der Einsicht und gegebenefalls Weiterverwendung der Daten bestehen, so
bitte ich um Kontaktaufnahme mit mir.

\chapter{Verwendete Dokumente}

Im folgenden sind die bei der Durchf�hrung des umfassenden
geriatrischen Assessments verwendeten Dokumente der einzelnen Tests
abgebildet. Zus�tzlich wurden die verwendete Einverst�ndniserkl�rung
und das Votum der Ethikkommission eingef�gt.

\begin{figure}[htbp]
  \centering
  \includegraphics[width=\textwidth]{Anhang/Information.pdf}
  \caption{Einverst�ndniserkl�rung}
  \label{fig:einverstaendnis}
\end{figure}

\begin{figure}[htbp]
  \centering
  \includegraphics[width=\textwidth]{Anhang/Ethik.pdf}
  \caption{Votum der Ethikkommission}
  \label{fig:ethik}
\end{figure}

\begin{figure}[htbp]
  \centering
  \includegraphics[width=\textwidth]{Anhang/Barthel.pdf}
  \caption{Barthel-Index}
  \label{fig:barthel}
\end{figure}

\begin{figure}[htbp]
  \centering
  \includegraphics[width=\textwidth]{Anhang/IADL.pdf}
  \caption{IADL-Skala}}
  \label{fig:iadl}
\end{figure}

\begin{figure}[htbp]
  \centering
  \includegraphics[width=\textwidth]{Anhang/TUG.pdf}
  \caption{TUG-Test}
  \label{fig:tug}
\end{figure}

\begin{figure}[htbp]
  \centering
  \includegraphics[width=\textwidth]{Anhang/Tinetti.pdf}
  \caption{Tinetti-Test}
  \label{fig:tinetti}
\end{figure}

\begin{figure}[htbp]
  \centering
  \includegraphics[width=\textwidth]{Anhang/MMST.pdf}
  \caption{MMS-Test}
  \label{fig:mmst}
\end{figure}

\begin{figure}[htbp]
  \centering
  \includegraphics[width=\textwidth]{Anhang/DemTect.pdf}
  \caption{DemTect}
  \label{fig:demtect}
\end{figure}

\begin{figure}[htbp]
  \centering
  \includegraphics[width=\textwidth]{Anhang/Charlson.pdf}
  \caption{Charlson-Score}
  \label{fig:charlson}
\end{figure}

\begin{figure}[htbp]
  \centering
  \includegraphics[width=\textwidth]{Anhang/GDS.pdf}
  \caption{GDS}
  \label{fig:gds}
\end{figure}

\begin{figure}[htbp]
  \centering
  \includegraphics[width=\textwidth]{Anhang/DIA.pdf}
  \caption{DIA-Skala}
  \label{fig:dia}
\end{figure}

\begin{figure}[htbp]
  \centering
  \includegraphics[width=\textwidth]{Anhang/MNA.pdf}
  \caption{MNA}
  \label{fig:mna}
\end{figure}

\chapter{Abk�rzungsverzeichnis}

\begin{acronym}
  \acro{ADL}{Aktivit�ten des t�glichen Lebens}
  \acro{AIC}{Akaike Information Criterion}
  \acro{AUC}{Area Under the Curve}
  \acro{BMI}{Body Mass Index}
  \acro{CGA}{Comprehensive geriatric assessment}
  \acro{CLL}{Chronisch Lymphatische Leuk�mie}
  \acro{cm}{Zentimeter}
  \acro{CTCAE}{Common Terminology for Adverse Events}
  \acro{DemTect}{Demenz-Detektion}
  \acro{DIA}{Depression im Alter}
  \acro{ECOG}{Eastern Cooperative Oncology Group}
  \acro{GDS}{Geriatric Depression Score}
  \acro{IADL}{Instrumentelle Aktivit�ten des t�glichen Lebens}
  \acro{IQR}{Interquartile Range}
  \acro{kg}{Kilogramm}
  \acro{m}{Meter}
  \acro{MMST}{Mini Mental Status Test}
  \acro{MNA}{Mini Nutritional Assessment}
  \acro{NA}{Not available}
  \acro{WHO}{World Health Organisation}
  \acro{PS}{Performance Status}
  \acro{ROC}{Receiver Operating Characteristics}
  \acro{s}{Sekunden}
  \acro{TUG}{Timed up and go}
\end{acronym}

%\newpage
%\addcontentsline{toc}{chapter}{D Abbildungsverzeichnis}
\chapter{Abbildungsverzeichnis}

\renewcommand\listfigurename{}
\begingroup
\let\clearpage\relax
\listoffigures
\endgroup

%\newpage
%\addcontentsline{toc}{chapter}{E Tabellenverzeichnis}
\chapter{Tabellenverzeichnis}

\renewcommand\listtablename{}
\begingroup
\let\clearpage\relax
\listoftables
\endgroup

%\newpage
%\printglossary
%\addcontentsline{toc}{part}{Abk�rzungsverzeichnis}

%\cleardoublepage
%\phantomsection
%\addcontentsline{toc}{chapter}{F Literaturverzeichnis}
