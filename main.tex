% FILE: main.tex  Version 2.1
% AUTHOR:
% Universität Duisburg-Essen, Standort Duisburg
% AG Prof. Dr. Günter Törner
% Verena Gondek, Andy Braune, Henning Kerstan
% Fachbereich Mathematik
% Lotharstr. 65., 47057 Duisburg
% entstanden im Rahmen des DFG-Projektes DissOnlineTutor
% in Zusammenarbeit mit der
% Humboldt-Universitaet zu Berlin
% AG Elektronisches Publizieren
% Joanna Rycko
% und der
% DNB - Deutsche Nationalbibliothek

% Dies ist die Hauptdatei Ihrer Dissertation. Mit * gekennzeichnete
% Elemente sind optional und können durch Entfernen/Hinzufügen des
% %-Zeichens gewählt/abgewählt werden. Weitere Informationen
% entnehmen Sie bitte der beiliegenden Broschüre!

% Dokumentenklasse
\RequirePackage[patch]{kvoptions} 
\documentclass{DissOnlineLatex}
% in DissOnlineLatex geometry
% Einstellungen geändert, da
% Optionsübergabe nicht klappt

\usepackage{setspace}
\usepackage[babel,german=guillemets]{csquotes}
\usepackage[citestyle=authoryear,bibstyle=nejm,sorting=nyt,mincitenames=1,maxcitenames=2,maxbibnames=99]{biblatex}
\DefineBibliographyStrings{ngerman}{
  andothers = {{et\,al\adddot}},
}
\usepackage{varioref}
\usepackage{hyperref}
\usepackage{cleveref}
\usepackage{booktabs}
\usepackage{mathtools}
\usepackage{graphicx}
\usepackage{epigraph}
\setlength\epigraphwidth{8cm}
\setlength\epigraphrule{0pt}
\usepackage{etoolbox}
\usepackage{acronym}
%\usepackage[toc]{glossaries}
%\makeglossary
\makeatletter
\patchcmd{\epigraph}{\@epitext{#1}}{\itshape\@epitext{#1}}{}{}
\makeatother
%wegen Sweave
%\usepackage{color}
%\usepackage{framed}
%\usepackage{alltt}
%\IfFileExists{upquote.sty}{\usepackage{upquote}}{}

\bibliography{bibliography}

% Eigene Trennungsregeln* 
\include{hyphenations} 
                          
%-zusaetzliche Kommandos*
%\include{command}

% Dokumentenanfang
\begin{document}

%selbst eingefügt zur Vermeidung von Schusterjungen und Hurenkindern
\clubpenalty=1000
\widowpenalty=1000
\displaywidowpenalty=1000

% Seitennummerierung für Titel, Widmung, Danksagung, Zusammenfassung,
% Inhaltsverzeichnis werden in römischen Zahlen gesetzt
\pagenumbering{roman}
\pagestyle{empty} 

% Titelblatt
% FILE: titlepage.tex  Version 3.0
% AUTHOR:
% Universit�t Duisburg-Essen, Standort Duisburg
% AG Prof. Dr. G�nter T�rner
% Verena Gondek, Andy Braune, Henning Kerstan
% Fachbereich Mathematik
% Lotharstr. 65., 47057 Duisburg
% entstanden im Rahmen des DfG-Projektes DissOnlineTutor
% in Zusammenarbeit mit der
% Humboldt-Universitaet zu Berlin
% AG Elektronisches Publizieren
% Joanna Rycko
% und der
% DNB - Deutsche Nationalbibliothek

%----------Generierung der Titelseite------------------------------------------

\makeatletter % Diese Zeile darf nicht gesl�scht werden!

\title{ \vspace{-3.5cm}\Huge \@Titel \\
\vspace{1cm}
\Large \@Untertitel
\vspace{2cm}
{\large{Medizinische Fakult�t \\ 
der \\
\@Universitaet \\
\vspace{1cm}
Aus der Klinik f�r H�matologie und Onkologie \\
des Huyssenstift \\
\vspace{2cm}
Inaugural - Dissertation \\
zur \\
Erlangung des Doktorgrades der Medizin \\
durch die Medizinische Fakult�t \\
der \@Universitaet
}}}

\author{Vorgelegt von \\  \@Anrede\ \@Vorname\ \@Nachname\ \\
aus \@Geburtsort \\ 2015}

%\date{\vspace{2cm}
%\raggedright{
%Datum der Einreichung: \@Abgabedatum \\
%\vspace{1cm}
%Dekan: \@Dekan \\
%1. Gutachter: \@GutachterA \\
%\vspace{1cm}
%Tag der m\"undlichen Pr\"ufung: \@Pruefungsdatum 
%}
%}

\makeatother % Diese Zeile darf nicht gel�scht werden!
\maketitle

% Widmung*
%% FILE: dedication.tex  Version 2.1
% AUTHOR:
% Universit�t Duisburg-Essen, Standort Duisburg
% AG Prof. Dr. G�nter T�rner
% Verena Gondek, Andy Braune, Henning Kerstan
% Fachbereich Mathematik
% Lotharstr. 65., 47057 Duisburg
% entstanden im Rahmen des DFG-Projektes DissOnlineTutor
% in Zusammenarbeit mit der
% Humboldt-Universitaet zu Berlin
% AG Elektronisches Publizieren
% Joanna Rycko
% und der
% DNB - Deutsche Nationalbibliothek

%\thispagestyle{empty}
%\vspace*{\fill}
%\begin{center}
%F�r unsere Patienten.
%\end{center}
%\vspace*{\fill}

\makeatletter
\cleardoublepage
\vspace*{\fill}
\begin{table}[h!]
\begin{tabular}{ll}
Dekan:        &\@Dekan \\
1. Gutachter: & \@GutachterA \\
2. Gutachter: & \@GutachterB \\
\end{tabular}
\end{table}

\begin{table}[h!]
\begin{tabular}{l}
Tag der m\"undlichen Pr\"ufung: \@Pruefungsdatum \\
\end{tabular}
\end{table}
\makeatother

% Danksagung*
%\include{acknowledgement}

% Zusammenfassung/Abstract
% FILE: abstract.tex  Version 2.1
% AUTHOR:
% Universit�t Duisburg-Essen, Standort Duisburg
% AG Prof. Dr. G�nter T�rner
% Verena Gondek, Andy Braune, Henning Kerstan
% Fachbereich Mathematik
% Lotharstr. 65., 47057 Duisburg
% entstanden im Rahmen des DFG-Projektes DissOnlineTutor
% in Zusammenarbeit mit der
% Humboldt-Universitaet zu Berlin
% AG Elektronisches Publizieren
% Joanna Rycko
% und der
% DNB - Deutsche Nationalbibliothek

%-englische-Zusammenfassung---------------------------------------
\selectlanguage{english}
\begin{abstract}
Here is the english abstract. 				
\end{abstract}

%-deutsche Zusammenfassung----------------------------------------
\selectlanguage{ngerman}
\begin{abstract}
Onkologen entscheiden aufgrund des Alters und des klinischen Eindrucks
t�glich welche Patienten welche Chemotherapie erhalten sollen. Bei
�lteren Patienten sind diese Entscheidungen aufgrund fehlender Evidenz
von der pers�nlichen Erfahrung und Einstellung abh�ngig. Ein
umfassendes geriatrisches Assessment wurde vorgeschlagen, um diese
Entscheidung datenbasiert und reproduzierbar f�llen zu k�nnen. Beide
Ans�tze haben zahlreiche Vor- und auch Nachteile. In der vorliegenden
Arbeit wird eine Kombination beider Ans�tze beschrieben und
untersucht. Bei dieser Kombination wird innerhalb einer
geriatrisch-onkologischen Konferenz auf der Datengrundlage eines
umfassenden geriatrischen Assessments unter Einbeziehung des
Behandlers, eines erfahrenen Onkologen und eines Geriaters sowie der
das Assessment durchf�renden Berufsgruppen eine gemeinsame
Entscheidung getroffen.

\begin{itemize}
\item Wie unterscheidet sich die Entscheidung der
  geriatrisch-onkologischen Konferenz von der pers�nlichen
  Entscheidung des Behandlers und der Einsch�tzung allein aufgrund des
  Assessments?
\item Wie h�ngt die Entscheidung der geriatrisch-onkologischen
  Konferenz von den einzelnen Kategorien des Assessments und der
  Einsch�tzung des Behandlers ab?
\item Ist durch ein statistisches Modell die Entscheidung der
  geriatrisch-onkologischen Konferenz vorhersagbar? Wie verl�sslich
  ist eine solche Vorhersage?
\end{itemize}

\end{abstract}

  
% Inhalts-, Abbildungs*-, Tabellen*-Verzeichnis
\tableofcontents

% Kapitel
\onehalfspacing % eigener Befehl aus dem empfohlenen Paket setspace,
                % Deckblätter auch mit anderthalbfachem Zeilenabstand?
                % Dann Befehl an den Anfang stellen

% Definition der Ergebnisse, nur hier berechnete Daten eingeben,
% später im Text dann nur per Variablennamen ansprechen.

% Beschreibung der analysierten Patienten
\newcommand{\ersterPatient}{Januar 2014}
\newcommand{\letzterPatient}{Dezember 2015}
\newcommand{\nPatients}{421} % Gesamtzahl der Patienten
\newcommand{\nMale}{233}
\newcommand{\nFemale}{188}
\newcommand{\percentMale}{55}
\newcommand{\percentFemale}{45}
\newcommand{\alterMedian}{76}
\newcommand{\alterMin}{60}
\newcommand{\alterMax}{94}
\newcommand{\nWeightHeight}{415}
\newcommand{\percentWeightHeight}{99}
\newcommand{\meanHeight}{169}
\newcommand{\sdHeight}{8}
\newcommand{\meanWeight}{74}
\newcommand{\sdWeight}{16}
\newcommand{\meanBmi}{25,7}
\newcommand{\sdBmi}{5,0}
\newcommand{\nUnterAltersDef}{12}
\newcommand{\percentGeriatrischOnkologie}{56}
% zwei Abbildungen: altersverteilung.pdf und altersverteilungAllgemein.pdf

% Barthel-Index
\newcommand{\nBarthel}{412}
\newcommand{\percentBarthel}{98}
\newcommand{\barthelMax}{100}
\newcommand{\percentBarthelMax}{51}
\newcommand{\medianBarthel}{100}

% IADL
\newcommand{\nIadl}{395}
\newcommand{\percentIadl}{94}
\newcommand{\iadlMax}{8}
\newcommand{\percentIadlMax}{36}
\newcommand{\medianIadl}{6}

% Tinetti
\newcommand{\nTinetti}{400}
\newcommand{\percentTinetti}{95}
\newcommand{\tinettiMax}{28}
\newcommand{\tinettiMin}{0}
\newcommand{\percentTinettiMax}{22}
\newcommand{\percentTinettiMin}{1}
\newcommand{\medianTinetti}{24}
\newcommand{\percentTinettiModerateRisk}{19}
\newcommand{\percentTinettiHighRisk}{33}

%TUG
\newcommand{\nTug}{365}
\newcommand{\percentTug}{87}
\newcommand{\tugNotPossible}{41}
\newcommand{\tugMin}{5}
\newcommand{\tugMax}{120}
\newcommand{\medianTug}{10}

% MMST
\newcommand{\nMmst}{374}
\newcommand{\percentMmst}{89}
\newcommand{\mmstMax}{30}
\newcommand{\percentMmstNormal}{78}
\newcommand{\percentMmstLight}{20}
\newcommand{\percentMmstModerate}{2}
\newcommand{\percentMmstSevere}{0}
\newcommand{\nMmstNormal}{292}
\newcommand{\nMmstLight}{73}
\newcommand{\nMmstModerate}{9}
\newcommand{\nMmstSevere}{0}

% DEMTECT
\newcommand{\nDemtect}{157}
\newcommand{\percentDemtect}{79}
\newcommand{\DemtectMax}{18}
\newcommand{\percentDemtectMax}{17}
\newcommand{\percentDemtectNormal}{62}
\newcommand{\percentDemtectLight}{24}
\newcommand{\percentDemtectDemenz}{13}
\newcommand{\nDemtectNormal}{98}
\newcommand{\nDemtectLight}{38}
\newcommand{\nDemtectDemenz}{21}

%GDS
\newcommand{\nGds}{115}
\newcommand{\percentGds}{91}
\newcommand{\gdsMax}{14}
\newcommand{\gdsMin}{0}
\newcommand{\percentGdsNormal}{77}
\newcommand{\percentGdsSuspect}{20}
\newcommand{\percentGdsDepression}{3}

%DIA-S
\newcommand{\nDia}{265}
\newcommand{\percentDia}{90}
\newcommand{\diaMax}{10}
\newcommand{\percentDiaNormal}{52}
\newcommand{\percentDiaSuspect}{11}
\newcommand{\percentDiaDepression}{37}

%Charlson
\newcommand{\nCharlson}{399}
\newcommand{\percentCharlson}{95}
\newcommand{\percentCharlsonZero}{34}
\newcommand{\percentCharlsonThreeOrMore}{23}

%MNA
\newcommand{\nMna}{358}
\newcommand{\percentMna}{85}
\newcommand{\mnaNormal}{33}
\newcommand{\mnaRisk}{51}
\newcommand{\mnaMalnourished}{16}
\newcommand{\nMnaNormal}{117}
\newcommand{\nMnaRisk}{182}
\newcommand{\nMnaMalnourished}{59}

%Soziale Situation
\newcommand{\nSocialStatus}{379}
\newcommand{\percentSocialStatus}{90}
\newcommand{\percentSocialAlone}{30}

%Anzahl der Defizite
%Viele Werte, direkt in Tabelle in chapter03.tex einfügen!

%Persönliche Einschätzung
\newcommand{\nPersonal}{356}
\newcommand{\percentPersonal}{96}
\newcommand{\percentPersonalGogo}{46}
\newcommand{\percentPersonalSlowgo}{41}
\newcommand{\percentPersonalNogo}{12}

%Konferenzeinschätzung
\newcommand{\nKonferenz}{392}
\newcommand{\nKonferenzNa}{29}
\newcommand{\percentKonferenz}{93}
\newcommand{\percentKonferenzGogo}{47}
\newcommand{\percentKonferenzSlowgo}{39}
\newcommand{\percentKonferenzNogo}{14}

%Vergleich Persönlich Konferenz
\newcommand{\nKonferenzPersonal}{337}
\newcommand{\percentAgreementKP}{83}

%Balducci
\newcommand{\nBalducci}{371}
\newcommand{\percentBalducci}{88}
\newcommand{\percentBaludcciGogo}{15}
\newcommand{\percentBalducciSlowgo}{30}
\newcommand{\percentBalducciNogo}{55}

%Vergleich Persönlich Balducci
\newcommand{\nBalducciPersonal}{322}
\newcommand{\percentAgreementBP}{34}

%Vergleich Balducci Konferenz
\newcommand{\nBalducciKonferenz}{359}
\newcommand{\percentAgreementBK}{36}

% Ende Ergebnisdefinition % eigene ergebnisse.tex Datei mit den
                     % numerischen Ergebnissen, benötigt für
                     % chapter03.tex

\newdoublepage

% Seitennummerierung im Hauptteil

\setcounter{page}{1}
\pagenumbering{arabic}
\pagestyle{myheadings} 

% FILE: main.tex  Version 2.1
% AUTHOR:
% Universit�t Duisburg-Essen, Standort Duisburg
% AG Prof. Dr. G�nter T�rner
% Verena Gondek, Andy Braune, Henning Kerstan
% Fachbereich Mathematik
% Lotharstr. 65., 47057 Duisburg
% entstanden im Rahmen des DFG-Projektes DissOnlineTutor
% in Zusammenarbeit mit der
% Humboldt-Universitaet zu Berlin
% AG Elektronisches Publizieren
% Joanna Rycko
% und der
% DNB - Deutsche Nationalbibliothek

\chapter{Einleitung}

\epigraph{``Begin at the beginning,'' the King said gravely, ``and go
  on till you come to the end: then stop.''}{--- \textup{Lewis
    Carroll, Alice in Wonderland}}

\section{Geriatrische Onkologie}

Vor drei Jahren lernte ich eine Patientin kennen, die gerade aufgrund
ihrer chronisch lymphatischen Leuk�mie schon zum zweiten Mal
chemotherapeutisch behandelt wurde. Bereits vor drei Jahren war sie schon 91
Jahre alt.

Ist in diesem Alter eine
Chemotherapie sinnvoll? Kann eine Chemotherapie in diesem Alter
vertragen werden? Verl�ngert eine Chemotherapie in diesem Alter das
�berleben? Welchen Patienten kann in diesem Alter eine Chemotherapie
zugemutet werden? Ab welchem Alter wird eine Chemotherapie schlechter
vertragen? Ab welchem Alter sind t�dliche Komplikationen der
Chemotherapie zu h�ufig, um hingenommen zu werden? Welche zus�tzlichen
Parameter neben dem Alter helfen bei der Einsch�tzung unserer
Patienten?

Auf all diese Fragen hat unsere heutige, moderne
Evidenz-basierte Medizin keine Antworten. Und trotzdem
sind das die Fragen, mit denen sich klinisch t�tige Onkologen t�glich
befassen und Antworten gegen�ber den Patienten, den Angeh�rigen und
sich selbst finden m�ssen.
In der Onkologie ist es eine allt�glichen Abw�gung, ob ein Patient in der
Lage ist, die von uns
empfohlene Chemotherapie zu vertragen oder nicht. Manchmal ist die
Einsch�tzung abh�ngig von Faktoren wie der Art und Aggressivit�t der
Tumorerkrankung, in manchen Extremsituationen auch z.B. vom
Ern�hrungszustand oder der geistigen Verfassung. Immer spielt auch das
Alter der Patienten eine entscheidene Rolle.

Die Geriatrische Onkologie ist ein Teilgebiet der
H�matologie/Onkologie und besch�ftigt sich mit den Besonderheiten
�lterer Menschen mit Krebserkrankungen.

Kein Onkologe w�rde einen einj�hrigen S�ugling chemotherapeutisch
wegen einer malignen Erkrankung behandeln. In dieser speziellen
Lebensphase des Menschen, charakterisiert durch eine besondere
Verletzlichkeit der sozialen, psychischen und physischen Situation der
kleinen Patienten, bedarf es neben onkologischer auch spezieller
p�diatrischer Expertise. In Deutschland ist daf�r eine eigener
Facharzt zust�ndig.

Aufgrund der demografischen Entwicklung besteht der Gro�teil der
Patienten eines klinisch t�tigen Onkologen und
H�matologen aus Menschen, die sich ebenfalls in einer
speziellen Lebensphase befinden. Auch die sp�te Lebensphase ist
gekennzeichnet durch eine besondere Verletzlichkeit bezogen auf die
soziale, psychische und physische Situation unserer �lteren Patienten. Um
dieser speziellen Situation gerecht zu werden bedarf es neben
h�matologisch/onkologischer Expertise auch spezieller geriatrischer
Erfahrung. Versucht wird dies in Deutschland aktuell durch die von
allen beteiligten Fachgesellschaften empfohlene, jedoch nur sehr
sp�rlich umgesetzte Zusammenarbeit von H�matologen/Onkologen mit
Geriatern.

Klinisches Ziel der Geriatrischen Onkologie ist eine bestm�gliche,
individuell auf die allgemeine (physische und psychische)
Belastbarkeit und die Lebenssituation angepasste Behandlung der
Patienten. Dies beinhaltet die interdisziplin�re Betreuung der
Patienten und ihrer Angeh�rigen unter Einbeziehung verschiedener
Berufsgruppen, und insbesondere die Steuerung
der spezifischen antitumor�sen Therapie.

Wissenschaftliches Ziel der Geriatrischen Onkologie ist speziell die
Erforschung der Besonderheiten von Krebserkrankungen �lterer Menschen
und der physiologischen Besonderheiten des �lteren Organismus in Bezug
auf Krebsentstehung, Krebsentwicklung und Reaktion auf eine antitumor�se
Therapie. In den meisten Zulassungsstudien und auch
Therapieoptimierungsstudien der Onkologie sind im Vergleich mit dem
klinischen Alltag �ltere Patienten dramatisch
unterrepr�sentiert. Daher ist ein allgemeines Ziel der Geriatrischen
Onkologie, den Anteil �lterer Patienten in onkologischen Studien zu
erh�hen und auch zus�tzliche Studien durchzuf�hren, die sich mit den
Besonderheiten der Behandlung �lterer Patienten besch�ftigt. Dabei
ist entscheidend zwischen dem chronologischen Alter, dem
physiologischen Alter und begleitenden geriatrischen Syndromen und
anderen Komorbidit�ten zu unterscheiden. Geriatrische Syndrome
bezeichnen dabei Geriatrie-typische Problemkonstellationen wie zum
Beispiel das Surzsyndrom, Immobilit�t, chronische Schmerzsyndrome,
Ess- und Trinkst�rungen, Schluckst�rungen, Demenz,
Depression.

Die vorliegende Arbeit beschreibt die Erfahrungen, die in den letzten
zwei Jahren aus der Zusammenarbeit der Klinik f�r Geriatrie und der
Klinik f�r internistische Onkologie der Kliniken Essen-Mitte gesammelt
wurden. Schwerpunkt ist die Beschreibung eines Kollektivs von �ber 400
Patienten, bei denen in diesem Zeitraum ein umfassendes geriatrisches
Assessment durchgef�hrt wurde und die anschlie�end in einer
interdisziplin�ren, geria\-trisch"=onkolo\-gischen Konferenz besprochen
wurden, um eine Einsch�tzung bez�glich der Therapief�higkeit der
Patienten abzugeben. Dies ist die erste mir bekannte Beschreibung
einer geriatrisch-onkologischen Zusammenarbeit, die nicht nur ein
Assessment beinhaltet, sondern eine solche interdisziplin�re Konferenz
und damit die individuelle Besprechung eines Patienten unter Einbeziehung der
Assessmentdaten in den Mittelpunkt der Zusammenarbeit stellt.

\section{Begriffserkl�rungen}

Der geriatrische Patient ist durch die Deutsche Gesellschaft f�r
Geriatrie definiert als Patient mit Geriatrie-typischer Multimorbidit�t
und h�herem Lebensalter, wobei explizit eine kalendarische
Altersangabe vermieden wird. Dies l�sst bewusst viel Spielraum f�r
Interpretationen.

Der �ltere Mensch wird in vielen geria\-trisch"=onkolo\-gischen Arbeiten ab
65 Jahren definiert. Diese kalendarische Definition habe ich hier
aufgrund der f�r ein Screening besser geeigneten festen Definition
�bernommen.

Umfassendes geriatrisches Assessment, geriatrisches Assessment und
Assessment werden synonym gebraucht und meinen ein in der Geriatrie
gebr�uchliches Erhebungsverfahren mit einer Kombination von
verschiedenen Tests. Abgek�rzt wird dies oft mit CGA, comprehensive
geriatric assessment.

Im folgenden Text wird das Wort signifikant auschlie�lich im
statistischen Sinn gebraucht, mit einem wie in der Medizin �blichen
Signifikanzniveau von 0,05.

\section{Kurze Geschichte der Geriatrischen Onkologie}

In der Geriatrie werden seit langer Zeit Assessments genutzt
\autocite{Leischker2007}. Bereits 1946, interessanterweise in der
gleichen Zeit, in der erstmals systematisch toxische und oftmals
t�dliche Chemotherapien in der Behandlung von Krebserkrankungen
eingesetzt wurden, wurde die moderne Form eines Assessments in der
Medizin durch Marjory Warren in England entwickelt
\autocite{Matthews1984}. Ziel von Warren war es, mit Hilfe eines
Assessments Patienten aus einem heterogenen Patientengut zu
selektionieren, die von einer medizinischen Intervention profitieren
w�rden. Bis heute ist es der prinzipielle Nutzen von verschiedenen
Assessments, aus einer heterogenen Gruppe von Patienten bez�glich
bestimmter Merkmale homogenere Untergruppen zu
definieren/selektionieren, die
dann Ausgangspunkt bestimmter Interventionen oder weiterer
Untersuchungen sind.

Die erste mir bekannte Ver�ffentlichung erschien 1973, in der von
Williams und Kollegen der Nutzen eines ambulanten geriatrischen
Assessments in Bezug auf die Vermeidung einer Pflegeheimeinweisung
gezeigt wurde \autocite{Williams1973}. Der Nutzen eines station�ren geriatrischen
Assessments wurde erstmals 1984 nachgewiesen \autocite{Rubenstein1984}. Dort wurden
Patienten, die systematisch mit Hilfe eines geriatrischen Assessments
in einer spezialisierten geriatrischen Abteilung behandelt wurden mit
Patienten aus der Normalversorgung verglichen. Zum
Entlassungszeitpunkt waren die Patienten die im Rahmen der
Normalversorgung behandelt wurden, in h�herem Ausma� auf fremde Hilfe angewiesen und
wurden h�ufiger in ein Pflegeheim eingewiesen. Die Einbeziehung eines
geriatrischen Assessments in die Behandlung verbessert also die
Selbst�ndigkeit und kann eine das Risiko einer Pflegeheimeinweisung reduzieren.

\section{Epidemiologie}

Die Lebenserwartung der Menschen hat im vergangenen Jahrhundert stetig
zugenommen. Da die meisten b�sartigen Erkrankungen mit steigendem
Alter zunehmen, stieg damit auch die Inzidenz von
Krebserkrankungen. Hinzu kommen bei �lteren Patienten au�erdem eine
erh�hte Inzidienz an Komorbidit�ten und funktionellen
Defiziten. Im Gegensatz zu j�ngeren Tumorpatienten sind �ltere
Tumorpatienten sowohl tumorbiologisch \autocite{Kumar2011} als auch bez�glich
Komorbidit�ten und funktioneller Probleme eine erheblich heterogenere
Gruppe.

Bezogen auf den einzelnen Menschen steigt die Krebsinzidenz in den
letzten Lebensdekaden exponentiell. Mehr als 60\% aller neu
diagnostizierter Krebserkrankungen und sogar mehr als 70\% der
Krebstodesf�lle betreffen Patienten, die 65 Jahre oder �lter sind. Die
altersangeglichene Krebsinzidenz ist bei den gr��er gleich 65 j�hrigen
10 fach h�her, die altersangeglichene Krebsterblichkeit sogar 16 fach
h�her verglichen mit den unter 65 j�hrigen \autocite{Berger2006,
  Owusu2014, Krebsregister2013}

Die moderne evidenzbasierte Medizin beruht auf der Durchf�hrung von
randomisierten, kontrollierten Studien. In diesen Studien waren bis
vor kurzem �ltere Patienten von vornherein ausgeschlossen. Dies �ndert
sich zwar in letzter Zeit, so ist allein das Alter in aktuellen
Studien kein Ausschlusskriterium mehr, jedoch sind auch heute noch die
�lteren Patienten in gro�en klinischen Studien, inklusive der
Zulassungsstudien der meisten etablierten Chemotherapien,
unterrepr�sentiert. Daher sind die Ergebnisse dieser Studien streng
genommen nicht auf diese �ltere Patientengruppe �bertragbar, was
jedoch mangels Alternativen im klinischen Alltag h�ufig nicht
eingehalten wird. Zus�tzlich ist das Verh�ltnis der verschiedenen
Altersgruppen im Krankenhaus genau umgekehrt, bezogen auf die
klinischen Studien. Der Gro�teil unserer Patienten sind �lter als 64
Jahre. (eigene Daten 2014: 62\% �lter als 64 Jahre, medianes Alter 67 Jahre)
Daher muss jeder klinisch t�tige internistische H�matologe und
Onkologe sich dieses Problems bewusst sein und somit auch ein
geriatrischer Onkologe sein.

\section{Geriatrische Assessments in der Onkologie}

\subsection{Nutzen bei klinischen Studien}

Ein Ziel der Geriatrischen Onkologie ist es, diese Situation zu
bessern. So m�ssen in den Studien mehr �ltere Patienten eingeschlossen
werden, um die �bertragbarkeit der Daten in den klinischen Alltag zu
erm�glichen. Zus�tzliche Studien sollten au�erdem die Besonderheiten
der Krebserkrankungen �lterer Patienten und die besonderen Reaktionen 
�lterer Patienten auf die spezifische Therapie
untersuchen. Dies verspricht nicht nur Erkenntnisse f�r die
t�gliche Arbeit, sondern auch ein zus�tzliches Verst�ndnis des Ph�nomens
einer Tumorerkrankung.

Altern ist ein hoch komplexer biologischer Vorgang. Sowohl angeborene
und erworbene genetische Ver�nderungen, als auch vergangene und
aktuelle �u�ere Einfl�sse spielen eine bedeutende Rolle. Es resultiert
eine enorme interindividuelle Verschiedenartigkeit der Krebserkrankung
und auch der erkrankten Patienten. Damit ist die ohnehin schon enorme
Komplexit�t einer Krebserkrankung eines Menschen noch erheblich
gesteigert. So weisen �ltere Patienten eine gr��ere
interindividuelle Schwankungsbreite bez�glich Organfunktion und
Wiederstandsf�higkeit bezogen auf �u�ere Einfl�sse auf, als j�ngere
Patienten.

Ein geriatrisches Assessment ist der Versuch durch einige
wenige, einfach durchzuf�hrende Tests eine Datengrundlage f�r eine
weitere Unterteilung zu schaffen. Dabei hat sich eine Dreiteilung der Population
in die Kategorien voll behandlungsf�hig, eingeschr�nkt behandlungsf�hig und nicht
behandlungsf�hig etabliert. Erstmals wurde diese Unterteilung in den
Behandlungsempfehlungen der chronisch lymphatischen Leuk�mie etabliert
und fand dort auch Eingang in die Leitlinie.

Sowohl die Dreiteilung als auch die Bezeichnung der Kategorien ist
willk�rlich, jedoch weit verbreitet. Im angels�chsischen Sprachraum
werden die Kategorien meistens fit oder go go, vulnerable, unfit oder
slow go und frail oder no go genannnt. Alle Bezeichnungen �berlappen
in der Aussagekraft weitgehend. Ausgehend von dieser
Kategorisierung ist zu bestimmen, inwiefern diese relevant sind, das
hei�t, inwiefern sie
prognostischen Charakter auf den Verlauf der Erkrankung oder auch
auf die Toxizit�t der Therapie besitzt. Dies
ist jedoch nicht Gegenstand der vorliegenden Arbeit.

\subsection{Nutzen in der t�glichen Arbeit}

Warum sind geriatrische Assessments n�tig? Jede medizinische Therapie,
oder weiter gefasst, jede medizinische Intervention soll dem Patienten
helfen. Symptome lindern, Funktionalit�t wiederherstellen, kausal
heilen, Krebswachstum eind�mmen. Dies ist allerdings immer auch
zwingend mit Wirkungen auf den Organismus verbunden, die diesem Ziel
nicht entsprechen, unerw�nschte Wirkungen, oder allgemein auch
Nebenwirkungen genannt. Manchmal sind diese harmlos, sie k�nnen jedoch
auch lebensbedrohlich sein. Die moderne Medizin besch�ftigt sich
zwingend auch mit der Einsch�tzung und bestenfalls Vermeidung und
Behandlung dieser Nebenwirkungen medizinischer Therapien. Es gibt
sogar Erhebungen, die feststellen, dass ein Gro�teil aller
Krankenhausaufenthalte zur Behandlung von unerw�nschten
Arzneimittelwirkungen n�tig sind \autocite{Hohl2001}.

In der H�matologie und Onkologie sind die diagnostizierten
Krankheiten sehr oft prim�r lebensbedrohlich, sowohl medizinisch objektiv als auch
subjektiv von Patienten und deren Umfeld empfunden. In dieser
Situation werden sowohl von Behandlern als auch den Patienten
Therapien toleriert, die zum Teil mit Erheblicher Morbidit�t und auch
Mortalit�t verbunden sind. Dies gilt ganz allgemein in der Therapie
von Krebskerkrankungen, sei es chirurgisch
(weiter ausf�hren: entstellende Chirurgie in den Anf�ngen der
Krebsbehandlung beim Mammakarzinom), strahlentherapeutisch, oder
klassich internistisch-onkologisch, also medikament�s durch zytostatische
Chemotherapien. Eine diesbez�glich sehr wichtige Frage ist: Wie k�nnen
die Risiken unserer
potentiell t�dlichen Therapien auf den individuellen Patienten besser
eingsch�tzet werden? In experimentellen,
kontrollierten Studien sehen wir die Wirkung, aber auch die
prozentuale Rate und die Schwere der Nebenwirkungen der getesteten
Therapien auf die Studienpopulation. Der Patient, aber auch der
Behandler, m�chte nicht nur wissen, wie das Medikament in der
Studienpopulation gewirkt hat, sondern auch in welchem Prozentsatz es welche
Nebenwirkungen verursacht hat, im Ernstfall, wie viele Patienten an
dieser Behandlung gestorben sind. Der Patient, die Angeh�rigen und der
Behandler wollen wissen, wie der potentielle Nutzen und das
potentielle Risiko dieser Therapie, bezogen auf den individuellen
Patienten aussehen. Dies ist kein spezielles Anliegen der
Geriatrischen Onkologie, sondern der gesamten Medizin.

Seit �ber 50 Jahren wird in der Onkologie der Karnofsky-Index (KI)
eingesetzt \autocite{Karnofsky1948}, eine 10 Punkte-Skala mit deren Hilfe der
Allgemeinzustand eines Patienten eingesch�tzt wird. J�ngere g�ngige Scores
sind der ECOG und der WHO-PS, 5 Punkte-Skalen, die in Ihrer Aussage
ann�hernd deckungsgleich sind. Validiert wurden alle Skalen an j�ngeren
Tumorpatienten. Dennoch sind sie von Onkologen in der t�glichen
Praxis h�ufig verwendete Instrumente, um die Therapief�higkeit von Patienten
einzusch�tzen. Dar�ber hinaus ist ein ausreichend hoher Score die
Voraussetzung zur Aufnahme in eine klinische Studie. Es besteht nur
eine sehr geringe Korrelation mit einem geriatrischen Assessment
\autocite{Extermann1998}. ECOG und KI erfassen die akuten Einschr�nkungen des
funktionellen Zustands durch die Tumorerkrankung. Alltagsrelevante
Einschr�nkungen, die schon vor der Tumorerkrankung vorhanden waren und
bei �lteren Tumorpatienten h�ufig bestehen werden in diesen Skalen
jedoch nur unzureichend dargestellt. Um diese auch f�r die
tumorspezifische Behandlung relevanten Einschr�nkungen zu erkennen,
sind die geriatrischen Funktionscores entwickelt worden. Daher sollte
meiner Meinung nach zuk�nftig das sehr viel pr�zisere geriatrische
Assessment den Performance Score vor Aufnahme in eine klinische Studie
abl�sen.

Die Prognose eines Patienten mit einer Krebserkrankung wird bestimmt
von Stadium und Art der Krebserkrankung und den Organfunktionen des
Patienten. Dabei spielt weniger das chronologische Alter als das
biologische Alter eine Rolle, also die k�rperliche und psychische
Leistungsf�higkeit sowie Komorbidit�ten. Routinem��ig werden Patienten
in jedem Bereich der Medizin von �rzten in biologisch �lter und
biologische j�nger eingeteilt. Diese pers�nliche und subjektive
Einteilung hat jedoch keine prognostische Relevanz
\autocite{Chow2001}. Manche therapierelevante Defizite im
funktionellen Leistungsverm�gen k�nnen nicht allein durch die Anamnese und
k�rperliche Untersuchung eines einzelnen Arztes entdeckt
werden. Ein interdisziplin�r durchgef�hrtes geriatrisches Assessment
kann unter Zuhilfenahme von standardisierten Testverfahren
im Alter auftretende Funktionsst�rungen h�ufiger erkennen \textbf{Literatur!}.

Das dr�ngende Problem der Geriatrischen Onkologie ist nun, dass
geriatrische Patienten, oder besser allgemein �ltere Patienten,
bez�glich der Organfunktion, als auch anderer funktioneller Parameter
eine heterogenere Gruppe darstellen, als j�ngere Patienten. Dies macht
diese Population f�r Studien so ungeeignet und ist die Ursache, dass
f�r diese gr��ere Gruppe so viel weniger
wissenschaftlich belastbare Daten vorliegen.

Ein geriatrisches Assessment ist in diesem Dilemma eine diagnostische
Ma�nahme, ein Versuch, den einzelnen Patienten besser
einzusch�tzen. Anhand einfacher und schnell zu erhebender Parameter
einen Eindruck zu gewinnen, in welchem Bereich des breiten Spektrums
dieser Patient bez�glich Organfunktion und Funktionalit�t steht. Ein
systematisches Assessment kann dabei auch therapeutisch beeinflussbare
Funktionseinschr�nkungen erkennen, die in der normalen �rztlichen
Anamnese und Untersuchung unauff�llig bleiben \autocite{Alessi1997}.

\subsection{Anforderungen an ein Geriatrisch-onkologisches Assessment}

Ziele des geriatrischen Assessments (Auflistung nach zunehmender Bedeutung)
\begin{itemize}
\item Absch�tzung der Lebenserwartung
\item Bestimmung der Einschr�nkung der Lebensqualit�t durch die
  Krebserkrankung
\item Absch�tzung der Wahrscheinlichkeit des Auftretens von tumorbedingten
Symptomen
\item Absch�tzung der Vertr�glichkeit/Toxizit�t der m�glichen
  Therapie
\item Absch�tzung der Einschr�nkung der Lebensqualit�t durch die
  m�gliche Therapie
\item Besteht die M�glichkeit, die Therapie zeitgerecht und vollst�ndig durchzuf�hren
\end{itemize}
  
Balducci und Extermann haben 2000 eine Klassifizierung von Patienten
aufgrund eines umfassenden geriatrischen Assessments in drei
Kategorien vorgeschlagen \autocite{Balducci2000}. Ber�cksichtigung finden
dabei jedoch lediglich drei Werte aus dem Assessment, der
Barthel-Index, der IADL und der Charlson-Komorbidit�tsindex.

\begin{enumerate}
\item Patienten, die funktionell unabh�ngig sind und keine
  Komorbidit�t aufweisen und damit Kandidaten f�r jeder Form der
  Standardtherapie sind, mit der m�glichen Ausnahme einer
  Knochenmarktransplantation.
\item Patienten die weder der einen noch der anderen Gruppen
  zuzuordnen sind und damit von einer prim�r reduzierten spezifischen
  Therapie profitieren k�nnten.
\item Patienten, die gebrechlich sind (Abh�ngigkeit in einer oder mehr
  Aktivit�t des t�glichen Lebens, drei oder mehr Komorbidit�ten, ein
  oder mehr geriatrisches Syndrom) und damit Kandidaten f�r eine
  rein symptomatische palliative Therapie sind.
\end{enumerate}

�bersetzt in Parameter aus dem geriatrischen Assessment zeigt dies Tabelle~\ref{tab:balducci}.

\begin{table}[htbp]
\caption{Vorschlag von Balducci \autocite{Balducci2000}}
\label{tab:balducci}
\centering
\begin{tabular}{llll}
\toprule
Bezeichnung nach Balducci & Barthel & IADL & Charlson \\
\midrule
fit/fit & 100 & 8 & 0 \\
vulnerable/gef�hrdet & 100 & >5 & <3 \\
frail/gebrechlich & <100 oder & <6 oder & >2 \\
\bottomrule
\end{tabular}
\end{table}

Es ist bekannt, dass durch eine Anamnese und k�rperlichen Untersuchung des
Patienten durch den Behandler, wesentliche Einschr�nkungen bei
geriatrischen Patienten verborgen bleiben k�nnen \autocite{Alessi1997}. Daher ist auch die
folgende Einsch�tzung der Therapief�higkeit des Patienten durch den
Behandler durch m�gliche fehlende wesentliche Informationen sehr
fehleranf�llig. Ein umfassendes geriatrisches Assessment kann diese
wesentlichen Einschr�nkungen bei geriatrischen Patienten besser
identifizieren \autocite{Alessi1997}. Aufgrund des Assessments k�nnen dann
spezifische Ma�nahmen eingeleitet werden, um diese Probleme zu
bessern. Aufgrund dieses Assessments kann dann jedoch auch eine
Einteilung der Patienten in die verschiedenen Kategorien nach dem
Vorschlag von Balducci und Extermann aus dem Jahr 2000 wie oben aufgef�hrt
erfolgen.

Es wurde in den letzten Jahren mehrfach festgestellt, dass eine
Klassifikation durch den Behandler mit der Klassifikation
aufgrund eines geriatrischen Assessments nur wenig korreliert
\autocite{Wedding2007, Tucci2009, Tucci2015}.
MEMO! Weiter ausf�hren, in Zahlen und Worten.

Die Einteilung durch den Behandler hat den Vorteil, dass sie die
klinische Erfahrung des Behandlers beinhaltet und damit zahlreiche
Faktoren, die nur schlecht gemessen werden k�nnen. Zum Beispiel die
Art der Tumorerkrankung, eine Einsch�tzung der aktuellen Situation mit
Abw�gung der Probleme, die durch die Erkrankung hervorgerufen werden
zus�tzlich mit der Einsch�tzung, wie diese mit einer spezifischen
Therapie gebessert werden k�nnen. Zus�tzlich die Information, welche
spezifische Therapie in Frage k�me und wie das Nebenwirkungsprofil
dieser speziellen Therapie ausf�llt.
Der Nachteil ist, dass wesentliche Informationen fehlen k�nnen, weil
der Patient diese nicht mitteilt und Einschr�nkungen vorliegen, die
wie schon ausgef�hrt der Anamnese und k�rperlichen Untersuchung
entgehen. Dies wird durch die pers�nliche Erfahrung jedes Behandlers
best�tigt, wie schwierig es zum Beispiel ist, durch nur wenige Kontakt
mit dem Patienten eine beginnende oder milde Demenz zu erkennen. Dies
kann jedoch w�hrend der Behandlung mit einer Chemotherapie wesentliche
und potentiell t�dliche Probleme verursachen.

Die Einteilung zum Beispiel durch den Vorschlag von Balducci auf der
Grundlage eines umfassenden geriatrischen Assessments hat den Vorteil,
dass durch das Assessment wesentliche Informationen zus�tzlich
gewonnen werden, die dem Behandler bei seiner Einsch�tzung nicht zur
Verf�gung stehen. Der Nachteil jeder Einsch�tzung aufgrund eines
Assessments ist, dass sowohl Art und Stadium der Erkrankung
unber�cksichtigt bleiben, als auch Vertr�glichkeit und Toxizit�t der
m�glichen Therapie. Spezieller Nachteil der Einteilung nach Balducci
ist, dass nur drei Elemente des multidimensionalen Assessments die
Einteilung beeinflussen. Lediglich die Aktivit�ten des t�glichen
Lebens, also die Pflegebed�rftigkeit durch den Barthel-Index, die
instrumentellen Aktivit�ten des t�glichen Lebens, also die
Unabh�ngigkeit im Leben durch den IADL-Score und die Komorbidit�t
durch den Charlson-Score beeinflussen die Einteilung nach
Balducci. Damit bleiben wesentliche Informationen des Assessments, wie
die geistige Leistungsf�higkeit (MMST) oder die Mobilit�t (TUG und
Tinetti) oder die soziale Situation unber�cksichtigt.

Mir erscheint die Einteilung nach Balducci au�erdem zu leicht
Patienten als gef�hrdet zu klassifizieren, zum Beispiel allein
aufgrund der Tatsache, dass sie nicht einkaufen oder nicht die W�sche
wachen. In einer Ehegemeinschaft mit lange bestehender Aufteilung der
Haushaltspflichten sicherlich kein gutes Ma� um eine solche
Unterteilung vorzunehmen. Auch die Einteilung aufgrund der
Komorbidit�t erscheint zumindest fragw�rdig. So sind im
Charlson-Komorbidit�tsscore zwar nur Erkrankungen aufgenommen, die
nachweislich die Mortalit�t wesentlich beeinflussen
\autocite{Charlson1987}. Jedoch ist anzumerken, dass Erkrankungen
aufgenommen sind, die vor 1987 die Mortalit�t wesentlich beeinflussen.
Dabei ist sicherlich unstrittig, dass mit den heutigen
Behandlungsm�glichkeiten zum Beispiel ein
Diabetes mellitus, der gut medikament�s behandelt ist, kein Grund ist,
einem Patienten die Standardtherapie vorzuenthalten. In der Einteilung
nach Balducci w�re allein dies jedoch der Grund einen Patienten als
gef�hrdet einzustufen. In der Zeit der Evaluation des
Charlson-Komorbidid�tsindexes war sogar ein Magenulcus prognostisch so
bedeutsam, dass es Eingang in den Index gefunden hat. Seit der
Einf�hrung der Protonenpumpeninhibitoren sicherlich kein Grund mehr
einen Patienten allein aufgrund eines in der Anamnese dokumentierten
Ulcus als gef�hrdet einzustufen.

Der neue Ansatz, der in dieser Arbeit erstmals verfolgt und
dargestellt wird, besteht darin beide Verfahren zu kombinieren. Damit
bleiben die Vorteile beider Ans�tze bestehen und die Nachteile
werden behoben. Die Kombination besteht darin, an Hand der Daten, die
in einem Assessment erhoben werden in einer interdisziplin�ren
Konferenz eine Klassifikation vorzunehmen. Damit sind die durch das
Assessment aufgedeckten Probleme zum Zeitpunkt der Beurteilung bekannt
aber auch die gerade geschilderten besonderen klinische Situationen
wie zum Beispiel ein sehr gut oder nur sehr schlecht eingestellter
Diabetes k�nnen Ber�cksichtigung finden. Zus�tzlich findet auch die
unsch�tzbare klinische Erfahrung der beteiligten Berufsgruppen
Beachtung. Hier dargestellt kommt dabei noch die Expertise eines
erfahrenen Geriaters und eines speziell im Bereich der geriatrischen Onkologie
t�tigen Onkologen hinzu.

MEMO Abbildung der drei verschiedenen Entscheidungswege?

Weder f�r die Einteilung durch den Behandler, noch f�r die
Klassifikation nach Balducci oder der hier vorgestellten
Konferenzentscheidung gibt es prospektive Untersuchungen zur Relevanz
der Einteilung. In wie weit kann eine Einteilung Nutzen und Risiko
einer Therapie vorhersagen?
Ein erster Schritt in der Beantwortung dieser Frage w�re, in jeder
klinischen Studie ein geriatrisches Assessment als Teil des
Screenings zu fordern.

In der vorliegenden Arbeit geht es isoliert um die Klassifikation
durch die geria\-trisch"=onkolo\-gische Konferenz. Therapief�higkeit wird
also nicht absolut beobachtet, sondern der Beschluss und die
entsprechende Klassifikation spiegelt die Beurteilung des Patienten
auf Grundlage der klinisch zur Verf�gung stehenden und durch das
Assessment erhobenden Daten durch die Anwesenden wieder.

Sich daran direkt anschlie�ende Fragen beinhalten, zum Beispiel
retrospektiv, oder aber auch prospektiv
im Sinne einer Registerstudie, welche Patienten nach der
Konferenzentscheidung welche Therapie mit welcher Toxizit�t erhalten
haben. Welche Therapien wurden
bei nicht therapief�hig eingestuften Patienten gestartet? Werden diese
Therapien schnell wegen �berm��iger Toxizit�t abgebrochen? Oder
wom�glich gut vertragen?

Inspirierend f�r diese Arbeit war f�r mich die in meiner Zeit in der
H�matologie h�ufig benutzte Risiko-Einsch�tzung von neu
diagnostizierten AML-Patienten \autocite{Krug2010}. Aufgrund standarm��ig erhobener
Parameter kann dort anhand statistischer Daten einer gro�en Datenbank
von AML-Patienten die Wahrscheinlichkeit einer Remission auf der einen
Seite und das Risiko von Fr�hsterblichkeit auf der anderen Seite,
ermittelt werden. Trotz fehlender prospektiver Validierung ist dies
meiner Meinung nach ein sinnvolles Hilfsmittel in der in
Grenzsituationen oft schwierigen Entscheidung zwischen einer
potentiell kurativen aber auch potentiell t�dlichen
Induktionschemotherapie oder doch hin zu einer besser vertr�glichen
palliativen Therapie.

Genau diese Hilfe kann auch das ebenfalls noch
nicht prospektiv validierte Hilfsmittel eines geriatrischen
Assessments mit anschlie�ender geria\-trisch"=onkolo\-gischer Konferenz
bieten. Neben der Analyse der Unterschiede und Gemeinsamkeiten aller
drei Klassifikationssysteme ist ein praktisches Ziel dieser Arbeit ein
statistisches Modell, welches aus den Assessmentdaten eine
Konferenzentscheidung vorhersagt. Dieses statistische Modell kann anhand der
vorhandenen Daten in seiner Vorhersagegenauigkeit beschrieben werden
und als Hilfsmittel im Internet zur Verf�gung gestellt werden.

\section{Sinnhaftigkeit eines Assessments in der Geriatrie}

Das umfassende geriatrische Assessment (CGA) erfasst multidiziplin�r
verschiedene altersassoziierte physiologsiche und psychologische
Faktoren, die Gesundheit und Krankheit �lterer Patienten mehr
beinflussen k�nnen, als allein das chronologische Alter.

Zus�tzlich ist
ein systematisches geriatrisches Assessment in der Lage
Funktionseinschr�nkungen zu erkennen, die der routinem��igen �rztlichen
Anamnese und k�rperlichen Untersuchung entgehen \autocite{Alessi1997}.

Daher sind in der Geriatrie verschiedene Assessmentsysteme Teil der
Regelversorgung. Nicht nur zu initialen Erfassung von Problemen,
sondern auch in der Verlaufsbeurteilung der Behandlung, ist diese Art
der multiprofessionellen Erfassung der verschiedenen Parameter neben
klassischer �rztlicher Anamnese und k�rperlicher Untersuchung im
Routinebetrieb geriatrischer Abteilungen etabliert.

\section{Sinnhaftigkeit eines Assessments in der Onkologie}

Der Beginn der modernen Onkologie liegt in der ersten H�lfte des
zwanzigsten Jahrhunderts mit dem Beginn der systematischen
Verabreichung von Chemotherapien.  Durch die Verabreichung von
Zellgiften konnten erstmals in der Behandlung von akuten Leuk�mien und
aggressiven Lymphomen, sp�ter dann auch einzelner solider Tumore wie
dem Hodenkarzinom, Patienten mit weit fortgeschrittener Erkrankung
geheilt werden. Wegen der zum Teil dramatischen Nebenwirkungen dieser
Therapien die oftmals t�dlich endeten, besch�ftigten sich die in
diesem Bereich t�tigen �rzte von Beginn an auch mit der
Frage, welchen Patienten eine solche Therapie zugemutet werden
kann. F�r welche Patienten besteht also die M�glichkeit, eine
Verbesserung der Erkrankung zu erreichen, ohne sie durch die Therapie
vutal zu gef�hrden. Die Entscheidung, welche Patienten f�r eine zytostatische
Therapie in Frage kamen, traf der Behandler, im besten Fall das Behandlungsteam.
Diese Situation hat sich bis heute in der Mehrzahl der modernen
Onkologien weltweit nicht ge�ndert. Durch moderne supportive Therapien
und Erfahrungen aus kontrollierten Studien in der Verabreichung der
Zytostatika sind die Therapien zwar sehr viel besser vertr�glich und
auch sicherer geworden als in den Anf�ngen der Onkologie. Trotzdem handelt es
sich bei der Chemotherapie weiterhin um die kontrollierte
Verabreichung von Zellgiften, die
potentiell t�dlich verlaufen kann. Jeder Onkologe macht im Verlauf
seines Berufslebens die Erfahrung, Patienten nicht an der
Krebserkrankung sterben zu sehen, sondern an den Folgen der durch ihn
eingeleiteten Therapie.

Zus�tzlich hat sich die Altersgrenze durch die moderne
Supportivtherapie (Wachstumfaktoren, Antiemetika, Antibiotika, etc.)
und der damit verbundenen Verbesserung der Akutvertr�glichkeit, als
auch der Folgeprobleme immer weiter hin zu h�herem Alter
verschoben. Hochdosischemotherapien mit autologen und allogenen
Stammzelltransplantationskonzepten wurden zu Beginn dieser Therapien
nur Patienten verabreicht, die j�nger als 50 Jahre waren. Mit
zunehmender Erfahrung mit diesen Behandlungsformen wurde das Alter
zum Beispiel bei der autologen Stammzelltransplantation in den letzten
Jahren bis auf 75 Jahre verschoben.
Durch die Weiterentwicklung in der Onkologie hin zu einer immer
zielgerichteteren Therapie,
die gegen Merkmale des Tumors oder auch des Tumor umgebenen Gewebes
gerichtet ist, sind andere Nebenwirkungsprofile als die der
klassischen Zellgifte zu beobachten. Neue Medikamente, die direkt zum
Teil hoch spezifisch in den Tumorzellstoffwechsel eingreifen haben
ebenfalls ein v�llig neues Nebenwirkungsprofil. Auch die aktuell
erstmals breit angewendete, erfolgreiche Beeinflussung des Immunsystems
durch sogenannte Immun-Checkpoint-Inhibitoren bringt neue Herausforderungen mit
sich. Mit der besseren Vertr�glichkeit einiger Substanzklassen ist au�erdem,
wie bei Imatinib oder Ibrutinib eine lebenslange kontinuierliche
Verabreichung m�glich. Weitgehend ungekl�rt sind in diesem Zusammenhang
m�gliche Interaktionen bez�glich Pharmakologie und Pharmakokinetik der
Substanzen mit den bei �lteren Patienten in zunehmender Zahl
eingesetzten Medikamenten, die zur Behandlung der Komorbidit�t n�tig sind.

Der erste mir bekannte Versuch einer systematischen Erfassung des
Allgemeinzustandes und der Leistungsf�higkeit, erfolgte durch
Karnofsky w�hrend der Behandlung von Lungenkarzinompatienten mit
Senfgas, einem chemischem Kampfstoff aus dem ersten Weltkrieg. Initial
nutzte Karnofsky den schon damals so genannten Performance Status
nicht zur Auswahl der Patienten, die eine Therapie erhalten konnten,
sondern zur Messung einer Besserung des Allgemeinzustandes mit dem
Ziel der Einsch�tzung der Wirksamkeit der Therapie.
Benutzt wird der Karnofsky-Index weiterhin, um eine �nderung des
Allgemeinzustands des Patienten w�hrend einer Chemotherapie zu
dokumentieren, aber auch um Patienten auszusortieren, die sich nicht
f�r eine spezifische Therapie eignen.
Zwei weitere, deckungsgleiche Skalen sind der ECOG oder WHO
Performance Status, kurz ECOG oder PS. Der Vorteil dieser Skalen liegt
in der Einfachheit und damit
direkt m�glichen Einsch�tzung des Behandlers nach dem ersten
Patientenkontakt. Auch in Therapiestudien hat die Einsch�tzung des
Allgemeinzustandes eines Patienten �ber den Performance Status gro�e
Bedeutung. So ist praktisch in allen onkologischen Studien ein PS von
gr��er 2, bei vielen Studien auch gr��er 1, Ausschlusskriterium. In
der t�glichen Arbeit eines Onkologen ist diese Art der Einsch�tzung
allgegenw�rtig. Patienten mit einem PS von 4 werden sicherlich nur in
sehr besonderen F�llen chemotherapeutisch
behandelt. Vorstellbar w�re zum Beispiel ein junger Patient, der durch
eine akute Leuk�mie in diesem schlechten Allgemeinzustand ist, und bei
dem man damit rechnet, dass eine Reduktion der Tumorlast dies
rasch bessert.
Aber auch bei Patienten mit einem PS von drei m�ssen gute Gr�nde f�r
eine spezifische Behandlung vorliegen. Eine intensive
Kombinationstherapie w�rde diesen Patienten wohl auch kein Onkologe
verabreichen.
Dass diese alleinige Einsch�tzung aufgrund des Performance Scores, sei
es durch Karnofsky oder ECOG/WHO nicht optimal ist, wei� jeder
klinisch t�tige Arzt. Aufgrund fehlender Alternativen ist es weiterhin
ein breit eingesetzter Ma�stab, entweder explizit bei der Auswahl von
Studienpatienten oder auch implizit in der Einsch�tzung des
Behandlers.
Au�erdem sind zahlreiche zus�tzliche Umst�nde denkbar, die nicht
unmittelbar den Performance Score beeinflussen, jedoch trotzdem
Einfluss auf die Vertr�glichkeit und Tolerierbarkeit einer
spezifischen Therapie haben. So ist zum Beispiel belegt, dass bei
�lteren Krebspatienten der Performance Score unabh�ngig von den
Komorbidit�ten ist \autocite{Extermann1998}.
Zus�tzlich w�rde sich ein fr�he Demenz ebenfalls nicht im Performance
Score wiederfinden, h�tte jedoch gerade bei allein lebenden Patienten
eine erhebliche Bedeutung bei der Planung und Durchf�hrung der
Chemotherapie.

Um zu entscheiden, ob ein individueller �lterer Patient spezifisch
behandelt werden soll, oder eine reine symptomatische Therapie
stattfinden soll, stellen sich zwei wesentliche Fragen:

\begin{itemize}
\item Wird der Patient an der Krebserkrankung sterben, oder
  wahrscheinlich wegen anderer Ursachen oder Komorbidit�ten?
\item Was sind die W�nsche und Erwartungen des Patienten und welche
  Therapie assoziierte Toxizit�t wird zum Erreichen toleriert\autocite{Blanco2015}?
\end{itemize}

Anhand des �ber 3000 geriatrische Patienten umfassenden
IN-GHO-Registers (Initiative Geriatrische H�amtologie und Onkologie)
ist belegt, dass Mortalit�t und Therapieerfolg bei �lteren
Tumorpatienten nicht vom Lebensalter abh�ngen, sondern besser auf Basis des
geriatrischen Assessments vorausgesagt werden k�nnen (noch nicht voll
publiziert) \textbf{Honecker???IN-GHO}.

Ohne ein strukturiertes Assessment laufen die genannten Absch�tzungen
intuitiv im Behandler ab, lediglich auf der Grundlage der beiden
Parameter des biologischen Alters und der klinischen Einsch�tzung.
Es bleiben prospektive Daten abzuwarten, ob ein geriatrisches
Assessment allein, oder mit anschlie�ender interdisziplin�rer
geria\-trisch"=onkolo\-gischer Konferenz, diese Absch�tzung besser
vornehmen kann, als der Behandler. Sicher ist jedoch, dass f�r eine
systematische Untersuchung in Studien das Assessment besser geeignet
ist.

\subsection{Therapiesicherheit}

In einer Studie an 348 Patienten, 70 Jahre oder �lter, die eine
aufgrund verschiedener Tumore eine Chemotherapie erhielten, korrelierte das
Risiko eines fr�hen Todes innerhalb der ersten sechs Monate der
Behandlung mit einem schlechten Ern�hrungsstatus, m�nnlichem Geschlecht und eine
eingeschr�nkte Mobilit�t. Das Risiko eines fr�hen Todes innerhalb der ersten
sechs Monate korrelierte nicht mit dem Performance Status, dem
Barthel-Index, den Instrumentellen Aktivit�ten des t�glichen Lebens
(IADL), dem Mini-Mental-Status-Test (MMST) oder des Geriatrischen
Depressions-Scores (GDS). Interessant ist au�erdem, dass
eine durch das Assessment ausgel�ste prim�re Dosisreduktion der
Chemotherapie nicht mit dem fr�hen Tod korrelierte \autocite{Soubeyran2012}.

Das mittlere �berleben von 566 Patienten, 75 Jahre oder �lter, mit
einem nicht-kleinzelligen Lungenkarzinom, die innerhalb einer Studie
behandelt wurden war 30 Wochen. Eine multivariate Analyse f�r
prognostische Faktoren ergab einen Einfluss des IADL-Scores und der
Lebenqualit�t, sowie der Anzahl und der Orte der Metastasierung. Ein
allgemeinerer Score als
der IADL, der ADL-Score und auch die Komorbidit�t, gemessen durch den
Charlson-Score, ergab keine zus�tzlichen Informationen bez�glich der
Prognose der Patienten \autocite{Maione2005}.

Eine andere Studie ebenfalls mit Patienten mit einem
nicht-kleinzelligen Lungenkarzinom, 70 Jahr oder �lter, zeigte eine
Korrelation von Ergebnissen eines geriatrischen Assessments mit
neuropsychiatrischer Toxizit�t und der F�higkeit eine voll dosierte
Chemotherapie zu tolerieren, nicht jedoch mit der Lebensqualit�t
\autocite{Biesma2011}.

In einer Studie an Patienten, 70 Jahre oder �lter, mit aggressiven
Lymphomen die eine dosismodifizierte CHOP-Therapie erhielten, konnte
das Geriatrische Assessment, speziell ADL und IADL einen fr�hen Tod
vorhersagen \autocite{Soubeyran2014}. Speziell in der Situation der aggressiven
Lymphome ist das f�hrende Dilemma der geriatrischen Onkologie
besonders deutlich sichtbar. Durch aggressivere Therapien auch der
�lteren Patienten konnte die Prognose dieser Patienten in den letzten
Jahren deutlich gesteigert werden \autocite{Feugier2005, Kouroukis2002}. Mit der
Aggressivit�t der Therapie steigt jedoch auch die Toxizit�t vor allem
auch bei �lteren Patienten bis hin zur in Kaufnahme von
therapieassoziierten Todesf�llen. In einer Serie von Patienten mit
einem aggressivem Lymphom, 65 Jahre oder �lter, konnte gezeigt werden,
dass ein geriatrischen Assessment effektiver in der Vorhersage von
Ansprechen, progressionsfreiem �berleben und Gesamt�berleben als die
klinische Einsch�tzung der Behandler war.

Bei Brustkrebspatientinnen, 65 Jahre oder �lter, die eine palliative
Erstlinienchemotherapie erhielten konnte ein pr�therapeutisch
durchgef�hrtes geriatrisches Assessment Grad 3/4 Toxizit�t der
Chemotherapie voraussagen. Polypharmazie war ebenfalls ein
unabh�ngiger Pr�diktor von Grad 3/4 Toxizit�t durch die
Chemotherapie. Dabei ist unklar, ob dies auch unabh�ngig von der im
Assessment erfassten Komorbidit�t, zum Beispiel durch direkte
Arzneimittelinterationen verursacht wird, oder lediglich, weil
Patientinnen mit h�herer Komorbidit�t auch mehr Medikamente einnahmen
\textbf{Hamaker2014}.

\subsection{Zus�tzliche therapeutische M�glichkeiten}

Bereits im geschichtlichen �berblick wurde beschrieben, wie ein
geriatrisches Assessment zus�tzlich zur �rztlichen Anamnese und
k�rperlichen Untersuchung in der Lage ist, Funktionseinschr�nkungen
bei �lteren Patienten zu erkennen \autocite{Alessi1997}. Das Erkennen von
Problemen ist der erste Schritt, diese Funktionseinschr�nkungen sind
zus�tzlich in den meisten F�llen durch spezifische Therapien,
medizinisch, physiotherapeutisch oder ergotherapeutisch behandelbar
und damit verbesserungsf�hig. Aber auch Probleme die nicht
therapeutisch angehbar sind, wie zum Beispiel die soziale Situation
des Patienten, sind oftmals im multidisziplin�ren Team zumindest
verbesserungsf�hig.

\subsection{Vorteile f�r den Behandler}

Jede Sekunde sterben Menschen aus verschiedensten beeinflussbaren
und auch unbeeinflussbaren Gr�nden. In der Onkologie ist das Sterben
aufgrund der palliativen Situation der meisten station�ren
Patienten ein gro�er Teil der t�glichen Realit�t. Selbst in dieser
schwierigen Situation kann die moderne Onkologie zusammen mit der
Palliativmedizin Gro�artiges f�r die Patienten leisten. F�r mich
pers�nlich ist das Sterben meiner Patienten aufgrund ihrer Erkrankung
ein normaler Teil meiner Arbeit, bei dem ich sowohl den Patienten als
auch ihren Angeh�rigen zum Teil erheblich helfen kann.
Stirbt ein Patient jedoch nicht an seiner Erkrankung, sondern an der
Therapie ist die Belastung des gesamten Behandlungsteams deutlich
erkennbar. Auch ist das Leid der Patienten in dieser Situation meist
h�her, da palliative, linderne Therapien wie eine therapeutische Sedierung seltener
und Patient und Angeh�rige belastende intensivmedizinische Therapien
h�ufiger zum Einsatz kommen.

In der folgenden Arbeit werden erstens die Daten eines umfassenden
geriatrischen Assessments von �ber 300 Patienten vorgestellt, die an
dieser Klinik 2014 und 2015 routinem��ig bei allen station�ren
Patienten, die 65 Jahre oder �lter waren, vor Einleitung einer
spezifischen Therapie erhoben wurden. Zweitens wird analysiert,
inwiefern die Entscheidung, die anhand der erhobenen Daten in einer
geria\-trisch"=onkolo\-gischen Konferenz getroffen wird, von der
pers�nlichen Einsch�tzung des behandelnden Arztes und von der
Klassifizierung durch das Assessment �bereinstimmt oder
abweicht. Drittens wird gepr�ft inwiefern ein statistisches Modell in
der Lage ist, anhand der Assessmentdaten die Klassifizierung, die
innerhalb einer geria\-trisch"=onkolo\-gischen Konferenz getroffen wird, zu
modellieren.

% FILE: main.tex  Version 2.1
% AUTHOR:
% Universit�t Duisburg-Essen, Standort Duisburg
% AG Prof. Dr. G�nter T�rner
% Verena Gondek, Andy Braune, Henning Kerstan
% Fachbereich Mathematik
% Lotharstr. 65., 47057 Duisburg
% entstanden im Rahmen des DFG-Projektes DissOnlineTutor
% in Zusammenarbeit mit der
% Humboldt-Universitaet zu Berlin
% AG Elektronisches Publizieren
% Joanna Rycko
% und der
% DNB - Deutsche Nationalbibliothek

\chapter{Methoden}

\epigraph{Nicht weil es schwer ist, wagen wir es nicht, sondern weil
  wir es nicht wagen, ist es schwer.}{--- \textup{Lucius Annaeus
    Seneca}}

\section{Elemente des umfassenden geriatrischen Assessments}

In Abschnitt~\ref{subsec:assessment} wurden die verschiedenen
Anforderungen an ein geriatrisches Assessment bei Krebspatienten
geschildert. Um diese Anforderungen bestm�glich zu erf�llen wurden
verschiedene Tests vorgeschlagen; die genaue Zusammensetzung des umfassenden
geriatrischen Assessments ist mit zunehmender prospektiver Evaluation
der Tests in Entwicklung. An den Kliniken Essen-Mitte hat sich
ausgehend von den aktuellen internationalen Empfehlungen
\autocite{Hurria2014} ein Standard f�r das umfassende geriatrische
Assessment etabliert:

\begin{itemize}
\item Funktioneller Status
  \begin{itemize}
  \item Aktivit�ten des t�glichen Lebens, Barthel Index, ADL
  \item Instrumentelle Aktivit�ten des t�glichen Lebens, IADL
  \item Mobilit�t, Timed up an go Test, TUG
  \item Sturzrisiko, Tinetti-Test
  \end{itemize}
\item Kognitive Funktion
  \begin{itemize}
  \item Mini Mental Status Test, MMST
  \item Demenz-Detektion, Demtect, zus�tzlich seit Februar 2015
  \end{itemize}
\item Komorbidit�t
  \begin{itemize}
  \item Charslon-Score
  \end{itemize}
\item Psychologischer Status
  \begin{itemize}
  \item Geriatrische Depressions Skala, GDS, bis August 2014
  \item Depression im Alter-Skala, DIA, seit September 2014
  \end{itemize}
\item Soziale Unterst�tzung
  \begin{itemize}
  \item Fragebogen
  \end{itemize}
\item Ern�hrungsstatus
  \begin{itemize}
  \item Mini Nutritional Assessment, MNA, inklusive Gr��e, Gewicht, BMI
  \end{itemize}
\end{itemize}

\subsection{Funtionalit�t}

\subsubsection{Aktivit�ten des t�glichen Lebens}

1965 wurde von der Physiotherapeutin Barthel und der �rztin Mahoney
ein Instrument zur  Verlaufsbeurteilung des funktionellen Status von
Patienten mit neuromuskul�ren und muskuloskelettalen Erkrankungen
entwickelt, welches bis heute als Barthel-Index verwendet wird
\autocite{Barthel1965}. Der Barthel-Index ist einfach zu erheben und
gibt die F�higkeit zur Selbstversorgung wieder. F�r die Aktivit�ten
Essen, Aufsetzen und Umsetzen, sich waschen, Toilletenbenutzung, Baden
und Duschen, Aufstehen und Gehen, Treppensteigen, An- und Auskleiden,
Stuhlkontinenz, Harnkontinenz wird erfasst, ob sie selbst�ndig, mit
Hilfe oder nicht ausgef�hrt werden k�nnen. Die einzelnen drei- oder
viergeteilten Beurteilungen addieren sich zu einem Wert zwischen 0 und
100 Punkten, durch 5 Punkte-Schritte unterteilt. Statistisch handelt
es sich dabei um eine diskrete, also diskontinuierliche
Verh�ltniskala. 0 Punkte zeigt dabei v�llige Pflegebed�rftigkeit, 100
Punkte v�llige Selbst�ndikeit an. Ab etwa 80 Punkten ist eine
weitergehende Erfassung mit den Instrumentellen Aktivit�ten des
t�glichen Lebens sinnvoll.

Mehr auf die Lebensf�hrung als die pflegerische Selbstversorgung gehen
die Instrumentellen Aktivit�ten des t�glichen Lebens (IADL) ein
\autocite{Lawton1969}. Dieser Score gibt die F�higkeit zur selbst�ndigen
Lebensf�hrung wieder. Beurteilt werden die F�higkeit zu telefonieren,
einzukaufen, zu kochen, den Haushalt zu f�hren, die W�sche zu waschen,
zur Benutzung von Transportmitteln, zur Medikamenteneinnahme, sowie zum
Umgang mit Geld. Maximal k�nnen 8 Punkte erreicht werden. Wegen der
fr�her �blichen Aufteilung der Haushaltsf�hrung erreichen M�nner auch
ohne Einschr�nkungen h�ufig nicht die volle Punktzahl. Im hohen
Punktzahlbereich des Barthel-Index und im niedrigen Bereich der IADL
beobachtet man eine �berlappung beider Skalen.

Eine eingeschr�nkte Mobilit�t beeintr�chtigt die Lebensqualit�t
erheblich. Ein erh�htes Sturzrisiko korrelliert au�erdem mit erh�hter
Mortalit�t, weiterem Verlust von Mobilit�t und funktioneller
Selbst�ndigkeit und h�ufigen Krankenhausaufenthalten. Ein Drittel
aller Menschen im Alter �ber 65 Jahre und die H�lfte aller �ber
90-J�hrigen st�rzt pro Jahr mindestens einmal \autocite{Trilling1995}.
Der Timed-up-and-go-Test \autocite{Podsiadlo1991} ist schnell und einfach
durchzuf�hren. Aus dem Sitzen auf einem Stuhl wird der Patient
angewiesen aufzustehen, drei Meter zu gehen und sich wieder
hinzusetzen. Hilfsmittel d�rfen dabei verwendet werden. Die ben�tigte
Zeit wird gemessen. Unter zehn Sekunden ist von einer
uneingeschr�nkten Mobilit�t im Alltag ausgegangen werden. Eine Zeit
von 10 bis 19 Sekunden zeigt eine eingeschr�nkte Mobilit�t an, die
jedoch noch zu keiner Einschr�nkung im Alltag f�hrt. Ab einer Zeit von
20 Sekunden muss von einer Beeintr�chtigung in den Aktivit�ten des
t�glichen Lebens ausgegangen werden.
Ein fehlender Wert im Timed-up-and-go-Test (TUG) kann in den
vorliegenden Daten zwei verschiedene Dinge bedeuten. Entweder wurde
der Test nicht durchgef�hrt, oder aber er war nicht m�glich, zum
Beispiel aufgrund Bettl�gerigkeit. Um diese wichtige Information nicht
zu verlieren, wurde eine zus�tzliche Variable eingef�hrt, die anzeigt,
ob die Durchf�hrung des Tests m�glich war.

Beim zweiteiligen Tinetti-Test \autocite{Tinetti1986} werden im ersten
Teil Standsicherheit und Balance �berpr�ft und im zweiten Teil das
Gangbild analysiert. Maximal sind 28 Punkte erreichbar, bei weniger
als 20 Punkten ist das Sturzrisiko erh�ht.

\subsection{Kognitive Funktion}
\label{sec:kognition}

Ein kognitives Assessment ist insbesondere bei onkologischen Patienten
wichtig, um subklinische kognitive Defizite fr�hzeitig zu
erfassen. Diese Defizite m�gen im Alltag des Patienten noch keine
Rolle spielen, k�nnten die n�tige Compliance bei einer Chemotherapie
jedoch entscheidend erschweren. Auch die Zustimmungsf�higkeit f�r eine
komplexische onkologische Therapie k�nnte dadurch eingeschr�nkt
sein. Jeder klinisch t�tige Kollege kennt aus eigener Erfahrung die
Situation, dass im routinem��igen Umgang mit Patienten eine leichte
Demenz erst erkannt wird, wenn sie Probleme bereitet, oder aber auch
�berbewertet wird, zum Beispiel in der extremen Belastungssituation
der Diagnose einer b�sartigen Krankheit.

Das international am h�ufigsten verwendete Screeninginstrument f�r
kognitive Defizite ist der Mini-Mental-Status-Test nach Folstein
\autocite{Folstein1975}. Mit maximal 30 zu vergebenen Punkten werden
zeitliche und �rtlicher Orientierung, das Kurzzeitged�chtnis, Lesen,
Schreiben und konstruktive F�higkeiten (Zeichnen) abgefragt. Unter 26
Punkte besteht Demenzverdacht. Allerdings ist der MMST zur Erkennung
leichter Demenzen nicht gut geeignet. 

Besonders gut zur Erkennung leichter kognitiver Einbu�en eignet sich
der Dem\-Tect-Test \autocite{Kalbe2004}. Gerade diese leichten
kognitiven Einbu�en, die im t�glichen Leben oft noch nicht auffallen,
sind bei der Durchf�hrung einer Chemotherapie durchaus
relevant. Ausgewertet wird f�r unter 60 j�hrige und �ber 60 j�hrige
getrennt. Maximal sind 18 Punkte erreichbar, bei weniger als 8 Punkten
besteht der Verdacht auf das Vorliegen einer Demenz.

Wenn klinisch der Verdacht auf eine Demenz bestand und der MMST
grenzwertige Ergebnisse lieferte, wurde durch die Konferenz 
zus�tzlich die Durchf�hrung des Dem\-Tect angeordnet. Dadurch
ergaben sich Verz�gerungen oder fehlende Ergebnisse, da 
der Patient gegebenenfalls bereits entlassen war.
Aus diesem Grund wird der Dem\-Tect-Test an den Kliniken
Essen-Mitte seit Februar 2015 zus�tzlich regelhaft durchgef�hrt.

\subsection{Komorbidit�t}

Komorbidit�ten, erfasst durch den Charlson-Score
\autocite{Charlson1987} korrellieren nur gering
mit dem funktionellen Status \autocite{Balducci2000}. Daher liefert der
Charlson-Score eine vom funktionellen Status unabh�nige, zus�tzliche,
haupts�chlich prognostische Information. Die Lebenserwartung von
�lteren onkologischen Patienten wird wesentlich von den Komorbidit�ten
mitbestimmt \autocite{Extermann2000}. Die Komorbidit�t hat auf die
Vertr�glichkeit einer Chemotherapie einen �hnlich starken Einfluss wie
der funktionelle Status \autocite{Frasci2000}. Der
Charlson-Komorbidit�tsindex ist f�r die Erfassung der Komorbidit�t bei
�lteren Tumorpatienten validiert. Aufgenommen
wurden neunzehn Punkte, die entsprechend ihres Einflusses auf die
Einjahresmortalit�t gewichtet wurden.

\subsection{Psychologischer Status}
\label{sec:depression}

Jeder Mensch geht anders mit einer belastenden Lebenssituation um. Die
Grenze zwischen normaler Traurigkeit aufgrund schlechter Nachrichten
und einer interventionsbed�rftigen Depression ist flie�end. Oft werden
jedoch auch vorbestehende Depressionen, eine h�ufige Diagnose bei
geriatrischen und onkologischen Patienten, vom Behandler nicht
erkannt. Ein Screening auf das Vorliegen einer Depression kann anhand
der Geriatric Depression Scale erfolgen \autocite{Yesavage1982}. Gestellt werden 15
Fragen, die mit Ja oder Nein beantwortet werden. Ein Punktwert von
mehr als 5 spricht f�r das Vorliegen einer Depression.

Die Erfahrung bei der Anwendung des GDS durch die Mitarbeiter der
Ergotherapie zeigte, das die in diesem Test gestellten Fragen, von
Patienten mit neu diagnostizierter Krebserkrankung zum gro�en Teil mit
gro�em Befremden aufgenommen werden. Problematisch waren zum Beispiel
Fragen wie: "`Finden Sie es sch�n, jetzt zu leben?"' oder "`Kommen Sie
sich in Ihrem jetzigen Zustand ziemlich wertlos vor?"' Nach Absprache
mit dem kooperierenden Geriater wurde ab September 2014 der GDS gegen
den DIA-S \autocite{Heidenblut2010} getauscht. In der Evaluation der
DIA-Skala zeigte sich gegen�ber der GD-Skala sogar eine �berlegenheit
bez�glich der Trennsch�rfe und der internen Konsistenz. Au�erdem
wurden die Aussagen dieses Tests deutlich positiver von den Patienten
aufgenommen.

\subsection{Soziale Unterst�tzung}

Der Mensch ist ein soziales Wesen. Das soziale Netzwerk eines Menschen
ist in jeder Lebenssituation von entscheidender Bedeutung. Das soziale
Netzwerk eines geratrischen Patienten spielt daher bei der Planung und
Durchf�hrung von onkologischen Therapien eine entscheidende
Rolle. Interessanterweise beeinflusst das soziale Netzwerk nicht nur
die Compliance des Patienten, sondern hat sogar einen signifikanten
Einfluss auf die Morbidit�t \autocite{Blazer1982}.
Die Komplexit�t des sozialen Umfeldes ist nicht ohne Weiteres
maschinell erfassbar. In der Diskussion in den Konferenzen stellte
sich jedoch heraus, dass das Merkmal "`alleinlebend"' Einfluss auf die
Entscheidung haben konnte. Diese Merkmal wurde daher als kategorielle
Variable mit aufgenommen.

\subsection{Ern�hrungsstatus}

Der Ern�hrungszustand eines Patienten bestimmt ma�geblich den
Allgemeinzustand mit. Mangelern�hrung vermindert die allgemeine
Leistungsf�higkeit, verschlechtert die Prognose einer Krebserkrankung
\autocite{Soubeyran2012} und auch die Vertr�glichkeit einer
Chemotherapie \autocite{Hurria2011}. Zus�tzlich k�nnen Probleme wie
�belkeit oder Schw�che, die durch die Tumorerkrankung oder die
Chemotherapie ausgel�st werden, den Ern�hrungszustand weiter
verschlechtern. Der h�ufigste Grund einer Mangelern�hrung bei
geriatrischen Patienten ist eine Depression. Aber auch Ursachen wie
ein schlechter Zahnstatus oder ungen�gende Kaufunktion sind h�ufig. In
43\% wird keine spezifische Ursache gefunden
\autocite{Wilson1998}. Der Body Mass Index (BMI) ist ein Ma�, der das
Gewicht im Verh�ltnis zur K�rpergr��e ber�cksichtigt.

\begin{equation}
\label{eq:bmi}
\text{BMI} = \frac{\text{Gewicht in kg}}{(\text{Gr��e in m})�}
\end{equation}

Ein BMI kleiner 18,5 gilt als untergewichtig. Durch
altersbedingte Umverteilungen von Muskelmasse und Fettanteil ist
jedoch bei geriatrischen Patienten bei einem BMI kleiner als 23
bereits die Wahrscheinlichkeit einer Mangelern�hrung hoch.
Das aktuelle Gewicht und damit auch der BMI ist eine Momentaufnahme
un spiegelt das Ern�hrungsverhalten der zur�ckliegenden Jahre
wieder. Um die Aussage des BMI durch eine Prognose der weiteren
Entwicklung des Ern�hrungszustands zu erg�nzen, hat sich das
Mini-Nutritional Assessment \autocite{Guigoz1997} f�r geriatrische
Patienten bew�hrt.

\section{Die geriatrisch-onkologische Konferenz}

In Abschnitt~\ref{sec:konferenz} wurde die Zusammensetzung und
Arbeitsweise der geriatrisch-onkologischen Konferenz erl�utert.

Die geriatrisch-onkologische Konferenz der Kliniken Essen-Mitte findet
zweimal w�chentlich, Dienstags und Donnerstags statt. Anwesend sind
die behandelnden �rzte, Physiotherapeuten, Ergotherapeuten,
Pflegemitarbeiter. Die Konferenz wird geleitet von einem Facharzt f�r
Geriatrie und einem Facharzt f�r Onkologie.
Die Ergebnisse des umfassenden geriatrischen Assessments und der
Konferenzbeschluss werden in eine Excel-Tabelle eingetragen und in
einer Ordnerstruktur im Intranet der Kliniken Essen-Mitte archiviert
(Abb.~\ref{fig:vorlage}).

\begin{figure}[htbp]
  \centering
  \includegraphics[width=\textwidth]{Grafiken/Vorlage.png}
  \caption{Beispiel einer Assessmentvorlage (Excel-Datei)}
  \label{fig:vorlage}
\end{figure}

\subsection{Pers�nliche Einsch�tzung}

Bei der Vorstellung des Patienten in der Konferenz wurde der
behandelnde Assistenzarzt befragt, wie das Behandlungsteam bestehend
aus einem Assistenzarzt in Weiterbildung und dem zust�ndigen
onkologischen Oberarzt den Patienten einsch�tzt. Diese
Einsch�tzung beruhte auf dem klinischen Eindruck, bestehend aus
Diagnose, Stadium, Anamnese und klinischer Untersuchung ohne
Einbeziehung der erhobenen Assessmentdaten.

\section{Hilfsmittel}

\subsection{\LaTeX{}}

\LaTeX{} ist eine Erweiterung des von Donald Knuth entwickelten
Satz-Systems \TeX{}. Mit Hilfe von \LaTeX{} ist es m�glich, alle nur
denkbaren Arten von Dokumenten zu erstellen. Von besonderem Vorteil
ist dabei, dass die Konzentration des Autors auf den Inhalt fokussiert
bleibt, und \LaTeX{} die professionelle Gestaltung des Layouts inklusive
Inhaltsverzeichnis und Literaturverzeichnis �bernimmt. Als
Ausgangspunkt wurde eine von der Arbeitsgruppe von Prof. T�rner
speziell auf die Bed�rfnisse von Dissertationen angepasste Vorlage
benutzt \autocite{Toerner2011}.

\subsection{R}

R ist eine Programmiersprache \autocite{R2015}, die besonders gut f�r
den Einsatz in der Statistik geeignet ist. Es ist eine open source Entwicklung, die
sich aus der kommerziellen Sprache S ableitet. R besitzt eine
aktive Entwicklergemeinde, die f�r zahlreiche Anwendungsgebiete
Erweiterungen in Form von Bibliotheken erstellt. In der
vorliegenden Arbeit wurden alle statistischen Berechnungen in R,
Version 3.2.2 durchgef�hrt.

Daten aus Excel-Dateien wurden mit dem Paket xlsx, Version 0.5.7
\autocite{Dragulescu2014} extrahiert. Das Geschlecht der Patienten
wurde anhand des Vornamens durch das Paket gender, Version 0.5.1
\autocite{Mullen2015} ermittelt und von Hand �berpr�ft und erg�nzt.
Die Grafiken wurden mit der Grafikbibliothek ggplot2, Version 1.0.1
\autocite{Wickham2009} erstellt. Die Berechnung der ROC-Kurven der
Modelle wurde mit der Bibliothek pROC, Version 1.8 \autocite{Robin2011}
vorgenommen. Um mehrere Grafiken in einer Seite zu setzen wurde ein
Codefragment aus dem R Graphics Cookbook \autocite{Chang2013}
verwendet.

\subsection{knitr}

knitr \autocite{Xie2014} ist ein Werkzeug, das R code sowie deren
erzeugten Output und die Dokumentation dazu verbindet. Es ist ein
wichtiges Hilfsmittel, um nachvollziehbar Datenmanipulationen zu
dokumentieren. Bei der vorliegenden Arbeit kamen mit der Arbeit an den
Daten jeden Monat zus�tzliche Daten aus den laufenden Assessments
hinzu. Um nicht immer wieder neu die beschriebenen Kennzahlen
berechnen zu m�ssen, wurden mit knitr, Version 1.11 jederzeit neu die
interessierenden Ergebnisse erzeugt. Es wird hierbei der
entpsrechenden Code in R geschrieben, der dann jedoch dynamisch aus
den vorliegenden und aktualisierten Daten die entsprechenden
Kennzahlen neu berechnet.

\section{Statistische Methoden}

\subsection{Beschreibende Statistik}

Gr��e und Gewicht folgen eine Normalverteilung. Alle anderen im
Assessment erhobenen Daten folgen keiner bekannten
Verteilungsfunktion. Da es sich um diskrete Verh�ltnisskalen handelt
wird zur Beschreibung der Verteilung minimaler und maximaler Wert, der
Median und der Interquartilenabstand benutzt. 

\subsection{Fehlende Variablen}

In praktisch allen statistischen Erhebungen kommen einzelne
Antwortausf�lle vor. Das hei�t, das einzelne Daten des Datensatzes
fehlen.

Alle statistischen Verfahren, und deren Ergebnisse und Interpretation
h�ngen direkt auch von fehlenden Werten ab. Zum Beispiel ist streng
genommen eine Regressionsanalyse oder aber auch nur die Berechnung
eines einfachen Mittelwertes formal nicht m�glich, wenn der Datensatz
fehlende Werte aufweist. Bei der Berechnung von Mittelwerten werden
bei fehlenden Werten diese einfach ignoriert. Nun ist jedoch
vorstellbar, je nach Mechanismus des Fehlens der Werte, dass dies das
Ergebnis erheblich verf�lscht. Wenn zum Beispiel aus
unbekannten Gr�nden nur hohe Werte fehlen, wird der Mittelwert zu
niedrig gesch�tzt. Auch die Fehlergrenzen k�nnen diese Fehlerursache
nicht ber�cksichtigen und werden daher zu schmal berechnet.

Bei einer Regressionsanalyse k�nnen Datens�tze mit fehlenden
Werten von der Analyse ausgeschlossen werden.
Dieser einfachste Fall der Behandlung fehlender
Werte, die so genannte complete case analysis, also die Analyse nur
vollst�ndiger F�lle, hat jedoch, eine Reihe schwerwiegender
Probleme. So f�llt die zu analysierende Fallzahl meist erheblich
geringer aus, und auch alle zu ermittelnden
Parameter ver�ndern ihre Eigenschaften zum Teil deutlich. Den
entstehenden Fehler nennt man Verzerrung oder Bias.
Daher ist es unumg�nglich, sich bei einer statistischen Analyse auch
mit fehlenden Werten und den m�glichen Ursachen zu besch�ftigen.

In der mathematischen Statistik gibt es ausgefeilte Mechanismen mit
diesen fehlenden Daten umzugehen. Unter dem Begriff der Imputation
werden dabei mathematische Verfahren zusammengefasst, fehlende Daten
nachtr�glich zu generieren. Zwei m�gliche einfache Verfahren sind zum
Beispiel das Ersetzen der fehlenden Variable durch den Mittelwert aus
den vorhandenen Variablen oder das Ersetzen durch einen aus einer
Regressionsanalyse ermittelten Wert.

Das hat nat�rlich Auswirkungen auf die nachfolgenden statistischen
Verfahren und der gefolgerten Aussagen. Es werden drei grunds�tzliche
Mechanismen unterschieden, warum einzelne Daten fehlen k�nnen.

\begin{enumerate}
\item Unabh�ngiges und zuf�lliges Fehlen, Missingness completely at random
\item Zuf�lliges Fehlen, Missingness at random
  \begin{itemize}
  \item Fehlen abh�ngig von beobachteten Pr�diktoren
  \item Fehlen abh�ngig von unbeoabachteten Pr�diktoren
  \end{itemize}
\item Fehlende Variablen, deren Fehlen von der fehlenden Variablen
  abh�ngen, Missingness that depends on the missing value itself
\end{enumerate}

Bezogen auf den hier analysierten Datensatz kommen verschiedene
grunds�tzliche M�glichkeiten fehlender Daten vor. Systematisch fehlen
Daten in Bereichen, die nicht erfasst wurden. So wurde bis August 2014 der GDS
zum Depressionsscreening benutzt, aufgrund einer besseren
Durchf�hrbarkeit jedoch ab September 2014 der DIA-S durchgef�hrt. Au�erdem
wurde anfangs zur Einsch�tzung der kognitiven F�higkeiten lediglich
der MMST durchgef�hrt, seit Februar 2015 zus�tzlich der Demtect um
Fr�hformen der Demenz besser zu erfassen. In den Zeitr�umen in denen
die entsprechenden Tests nicht durchgef�hrt wurden, gibt es dazu
nat�rlich auch keine Daten.

Diese systematisch fehlenden Daten machen den gr��ten Teil der
fehlenden Daten im vorliegenden Datensatz aus. Das beeinflusst auch die
Analyse; es hat sich zum Beispiel herausgestellt, dass in der
vorliegenden Anwendung der Demtect der bessere Test zur Beurteilung
des kognitiven Status ist. Die besonderen Gr�nde daf�r sind in
Abschnitt~\ref{sec:kognition} bescreiben. Daher w�re der Demtect auch
f�r die Erstellung der Modelle als Pr�diktor wertvoll gewesen, was
jedoch die Fallzahl durch fehlende Werte nur eines Pr�diktors auf �ber
die H�lfte reduziert h�tte. Aus diesem Grund kam in den
Modellen durchg�ngig der MMST zur Anwendung.

Als zweith�ufigster Mechanismus kommt im vorliegenden Datensatz
unabh�ngiges und zuf�lliges Fehlen vor. Das bedeutet, dass einzelne
Werte durch organisatorische Probleme nicht erhoben wurden. Diese
organisatorischen Probleme sind unabh�ngig von den Patienten, und
daher nicht von beobachteten oder unbeobachteten Pr�diktoren
abh�ngig. Bez�glich zus�tzlichen Bias sind diese fehlenden Werte
unproblematisch. Jedoch reduzieren sie die zu analysierende Fallzahl
und tragen damit zu h�heren Fehlerbereichen der Sch�tzwerte bei.

Zus�tzlich kommen in den Daten des TUG-Tests fehlende Werte vor, die
von beoabachteten oder unbeobachteten Pr�diktoren abh�ngen. Zum
Beispiel konnten einige TUG-Tests nicht durchgef�hrt werden, weil die
Patienten bettl�gerig waren. Eine wichtige Information, die teilweise
zwar auch im Tinetti-Test und im Barthel-Index erfasst wurde, jedoch
nicht zur G�nze in einem eigenen Pr�diktor. Andere TUG-Tests wurden
jedoch nicht durchgef�hrt, weil die Patienten das aus unbekannten
Ursachen abgelehnt haben. Und auch bei dem TUG-Test ist es m�glich,
dass organisatorische Probleme ein Fehlen verursachen und es sich
damit wieder um die im vorangegangenen Absatz beschriebene Ursache
handelt.

Bei allen Analysen wurde der Stichprobenumfang angegeben und damit der
Umfang der fehlenden Werte.

\subsection{Korrelation}

Zur Bestimmung der Korrelation zweier Merkmale wurde der
Rangkorrelationskoeffizient nach Spearman verwendet. Ber�cksichtigt
wurden Korrelationen gr��er 0,4. Der Standardfehler $\sigma$ wurde nach der von
Spearman ermittelten Gleichung~\ref{eq:spearman} berechnet. Dabei
bezeichnet n die Gr��e der Stichprobe.

\begin{equation}
\label{eq:spearman}
\sigma = \frac{0,6325}{\sqrt{n-1}}
\end{equation}

\subsection{Signifikanztests}
\label{subsec:Signifikanztests}

\subsubsection{Unterschiede zwischen Stichproben}

Der Kruskal-Wallis-Test ist ein statistischer Test, der
bei nicht normalverteilten Stichproben und mehr als zwei Gruppen
eingesetzt werden kann. Es wird anhand von Rangplatzsummen gepr�ft, ob
Unterschiede zwischen den Gruppen bestehen. Bei signifikantem
Testergebnis wurde post hoc mittels Wilcoxon-Rangsummentest die
einzelnen Stichproben paarweise verglichen. Zur Korrektur einer
Alphafehler-Kumulation kam die Bonferroni-Methode zum Einsatz.

\subsection{Logistische Regression}

In einer Regressionsanalyse wird versucht, eine abh�ngige Variable
durch eine oder mehrere unabh�ngige Variablen (Pr�diktoren)
darzustellen. Dadurch k�nnen Zusammenh�nge sowohl quantitativ
beschrieben werden, als auch Werte der abh�ngigen Variable durch die
unabh�ngigen Variablen prognostiziert werden.
Handelt es sich bei der abh�ngigen Variable um eine kategorielle
Variable wird die Wahrscheinlichkeit $p(X)$ in eine der zur Verf�gung
stehenden Kategorien zu fallen, durch eine oder mehrere quantitative
unabh�ngige Variablen, z.B. $X$
beschrieben. Bei der logistischen
Regression findet zur Beschreibung dieses Zusammenhangs die logistische
Funktion Anwendung.

\begin{equation}
\label{eq:logistic_function}
p(X) = \frac{e^{\beta_0 + \beta_1 X + \epsilon}}{1 + e^{\beta_0 + \beta_1 X + \epsilon}}
\end{equation}

F�r zwei zur Verf�gung stehende Klassen, z.B. tod/lebend,
krank/gesund, behandlungsf�hig/nicht behandlungsf�hig  muss die
Wahrscheinlichkeit lediglich bez�glich einer Klasse ermittelt werden, da sich
daraus dann direkt die zweite Wahrscheinlichkeit durch
$1 - p(X)$ ergibt. Durch Umformung von
Gleichung~\ref{eq:logistic_function} ergibt sich
Gleichung~\ref{eq:logit}.

\begin{equation}
\label{eq:logit}
\ln\bigg(\frac{p(X)}{1 - p(X)}\bigg) = \beta_0 + \beta_1 X
\end{equation}

Im Gl�cksspiel ist die Chance auf ein Ereigniss gel�ufiger als dessen
Wahrscheinlichkeit. Eine Wahrscheinlichkeit von $0,5$ entspricht einer
1 zu 1 ($1:1=1$) Chance. Allgemein gelten die
Gleichungen~\ref{eq:odds1} und \ref{eq:odds2}. Im weiteren Text wird f�r Chance das
in der Medizin gel�ufigere englische Wort Odds verwendet.

\begin{equation}
\label{eq:odds1}
o(A) = \frac{p(A)}{1 - p(A)}
\end{equation}

\begin{equation}
\label{eq:odds2}
p(A) = \frac{o(A)}{1 + o(A)}
\end{equation}

Dabei ist $p(A)$ die Wahrscheinlichkeit und $o(A)$ die Odds, dass
Ereignis $A$ eintritt. Ber�cksichtigt man die Definition der Odds
zeigt Gleichung~\ref{eq:logit}, dass die Beziehung zwischen den
logarithmierten Odds und der unabh�ngigen Variablen linear ist. Die
logarithmierten Odds werden auch als logit bezeichnet.

Im vorliegenden Fall sind jedoch nicht zwei Klassen vorhanden, sondern
drei. M�glich ist eine Erweiterung der logistischen Regression auf
drei Klassen. Dabei wird die 
Wahrscheinlichkeit modelliert in eine der drei Klassen nicht
therapief�hig/eingeschr�nkt therapief�hig/uneingeschr�nkt
therapief�hig eingeteilt zu werden. In der Vorhersagefunktion dieses
Modells findet die Ermittlung der Wahrscheinlichkeit statt, nach der
dann die Klassifikation erfolgt. Bei drei verschiedenen Klassen ergibt
sich ein Gleichungssystem mit zwei Gleichungen. Die
Wahrscheinlichkeit, in die dritte Klasse zu fallen ergibt sich aus der
Tatsache, dass sich alle drei Wahrscheinlichkeiten zu $1$ summieren. 

Die Erweiterung kann die Tatsache der Ordnung der Klassen
bei ordinalen Skalen, wie im vorliegenden Fall, ber�cksichtigen oder
nicht. Wenn die Ordnung ber�cksichtigt wird
(Proportional Odds Model), ist die sogenannte proportional odds
assumption Vorraussetzung daf�r. Diese besagt, dass die Beziehung
zwischen den Stufenpaaren gleich ist. Es l�sst sich statistisch
zeigen, dass dies bei den vorliegenden Daten nicht der Fall
ist. Aus der klinischen eigenen Erfahrung aber auch der Erfahrung in
der geriatrisch-onkologischen Konferenz erkl�rt sich die Verletzung
dieser Vorraussetzung recht einfach. Der Unterschied im
Allgemeinzustand eines Patienten, der nicht behandlungsf�hig
eingesch�tzt wird ist viel gravierender, als die manchmal nur
minimalen Unterschiede die bei Patienten bestehen, die eingeschr�nkt
oder uneingeschr�nkt therapief�hig sind.

Wenn die Ordnung nicht ber�cksichtigt wird, die ordinale Skala also
wie eine diskrete benutzt wird, geht zwar die Zusatzinformation der
Ordung verloren. Die oben beschriebene Voraussetzung ist dann jedoch
nicht mehr n�tig. Dieser Ansatz benutzt jedoch nur einen Satz von
Pr�diktoren. Dabei kommt also erstens nicht zur Darstellung, dass sich
die Pr�diktoren die zwischen therapief�hig und nicht therapief�hig
unterscheiden andere sein k�nnten, als die Pradiktoren, die das Ma�
der Therapief�higkeit bestimmen und umgekehrt.
Bei einem vollst�ndig pflegebed�rftigen und dementen Patienten k�nnte
die Komorbidit�t und der Ern�hrungszustand zum Beispiel keine Rolle
mehr spielen. F�r die Einteilung in die Kategorien k�nnten also
jeweils andere Pr�diktoren von Bedeutung sein.
Theoretisch macht dies noch keinen Unterschied, da die Pradiktoren
dann im vorliegenden Gleichungssystem niedrige Koeffizienten
h�tten. Praktisch ist das bei einem Datensatz mit m�glichen fehlenden
Werten jedoch von erheblicher Bedeutung, da jeder zus�tzlicher
Pr�diktor in einer niedrigeren Fallzahl resultiert, die damit die
Analyse erschwert.

Daher wurde bei der Erstellung des statistischen Modells ein
zweischrittiger Ansatz gew�hlt. In einer ersten binomialen
logistischen Regression wird ermittelt, ob Therapief�higkeit besteht
oder nicht. Sollte Therapief�higkeit bestehen, wird in einer zweiten
Regression die Unterteilung zwischen eingeschr�nkt therapief�hig und
uneingeschr�nkt therapief�hig vorgenommen.

\section{Code book}

Das Code book einer statistischen Analyse beschreibt den verwendeten
Datensatz inklusive der erhobenen Variablen und ihrer Werte und wie
der Datensatz erstellt wurde.

Die Protokolle der geriatrisch-onkologischen Konferenz befinden sich
in einer Verzeichnisstruktur im Intranet der Kliniken Essen-Mitte.
Jedes Jahr entspricht einem Ordner, jeder Monat einem Unterordner. Der
verwendete Datensatz wurde dynamisch aus dieser 
Ordnerstruktur heraus erstellt. Aus Sicherheitsgr�nden wurde die
einzulesenen Ordnerstruktur aus dem Intranet in ein lokales
Verzeichnis kopiert. Daraus las dann ein R-Script rekursiv
jede einzelne Excel-Datei aus und legte die Daten \emph{nach Entfernung des
Namens des Patienten} in jeweils einer Zeile eines Datensatzes ab. Ein
zweites R-Script bereitete diese Daten weiter auf. Es wurden
den Datenfeldern entsprechende Skalenniveaus zugewiesen, das
Alter wurde aus dem Geburtsdatum und dem Datum der Konferenz
berechnet. Nach der Berechnung des Alters in Jahren wurde zus�tzlich
noch \emph{das Geburtsdatum aus dem Datensatz entfernt.} Seit diesem
Schritt sind die Daten \emph{anonymisiert.}

Das Ergebnis dieser Manipulationen ist ein Datensatz, der lokal
gespeichert wird. Im folgenden wird die interne Struktur des
Datensatzes weiter beschrieben. Eine Zeile entspricht einem
Patienten. In den einzelnen Spalten befinden sich die erhobenen Daten,
sowie der Konferenzbeschluss. NA (not available) bedeutet, dass der
ensprechende Wert fehlt. Der Demtect-Test wurde wie in
\ref{sec:kognition} beschrieben, regelhaft erst seit
Februar 2015 im Assessment zus�tzlich zum MMST erhoben. Der GDS wurde
aus dem in \ref{sec:kognition} genannten Gr�nden seit
September 2014 durch den DIA-S ersetzt. Alle anderen punktuellen NA
dokumentieren im Einzelfall nicht erhobene Befunde oder nicht
eingetragene und damit auch in der Konferenz nicht zur Verf�gung
stehende Parameter.

Die einzelnen Datenfelder werden in Tabelle~\ref{tab:datensatz} beschrieben.

\begin{table}[htbp]
\caption{Beschreibung des Datensatzes}
\label{tab:datensatz}
\centering
\begin{tabular}{p{2.5cm}p{10cm}}
\toprule
Datenfeld & Beschreibung \\
\midrule
barthel & Barthel-Index, Ordinalskala von 0 bis 100\\
iadl & IADL-Skala, Ordinalskala von 0 bis 8\\
mmst & MMS-Test, Ordinalskala von 0 bis 30\\
Dem\-Tect & Dem\-Tect, Ordinalskala von 0 bis 18\\
gds & Geriatric Depression Score, Ordinalskala 0 bis 15\\
dia & Depression im Alter, Ordinalskala 0 bis 10\\
tug & Timed up and go, Zeitskala in Sekunden\\
tinetti & Tinetti-Test, Ordinalskala, hier Summe beider Anteile von 0 bis 28 \\
height & Gr��e in cm, kontinuierliche Skala\\
weight & Gewicht in kg, kontinuierliche Skala\\
mna & Mini Nutritional Assessment, Ordinalskala 0 bis 14\\
charlson & Charlson-Komorbidit�ts-Score, Ordinalskala, 0 bis 37\\
sozial & Sozialanamnese, freier Text \\
konferenz & Datum der Konferenz, Date format (R base package) \\
beschluss & freier Text des Konferenzbeschlusses \\
diagnose & freier Text der Diagnose \\
personal\_\newline assessment & freier Text der pers�nlichen Einsch�tzung der Behandlers \\
sex & kategorielle Variable, ungeordnet, \emph{male} und \emph{female} \\
beschluss\_f & kategorielle Variable, geordnet, \emph{no\_go}, \emph{slow\_go}, \emph{go\_go} \\
personal\_\newline assessment\_f & kategorielle Variable, geordnet, \emph{no\_go}, \emph{slow\_go}, \emph{go\_go} \\
alter & Verh�ltnisskala des Alters in Jahren von 60 bis 93 \\
tug\_possible & kategorielle Variable, \emph{TRUE}, \emph{FALSE}, beschreibt, ob TUG Test m�glich war oder nicht \\
tinetti1 & Tinetti-Test, Ordinalskala von 0 bis 16, erster Teil\\
tinetti2 & Tinetti-Test, Ordinalskala von 0 bis 12 zweiter Teil\\
bmi & Body Mass Index, kontinuierliche Skala aus height und weight berechnet\\
living\_ alone & kategorielle Variable, \emph{TRUE}, \emph{FALSE}, aus Variable sozial gebildet, beschreibt alleinlebend oder nicht \\
alter\_days & Verh�ltnisskala, Alter in Tagen berechnet \\
\bottomrule
\end{tabular}
\end{table}

\subsubsection{Zusammenfassung der Datenmanipulation}

\begin{enumerate}
\item Dateien aus Intranet lokal kopieren.
\item Script \textit{gather\_data\_V3.Rmd} ausf�hren, Geschlecht nach
  Vornamen �berpr�fen und erg�nzen, erzeugt \textit{xlsx\_data.rds}
  ohne Patientennamen aber noch mit Geburtsdatum.
\item Script \textit{tidying\_data.Rmd} ausf�hren, erzeugt
  \textit{tidy\_data.rds} ohne Geburtsdatum, also vollst�ndig
  anonymisiert.
\item Script \textit{balducci2000.Rmd} ausf�hren, erzeugt
  \textit{geon.rds} mit zus�tzlich erzeugter Beurteilung nach
  Balducci.
\item Script \textit{patientencharakteristika.Rmd} erzeugt
  beschreibende Statistik und Grafiken.
\item Script \textit{modell.Rmd} berechnet und evaluiert statistische
  Modelle und generiert beschreibende Grafiken.
\end{enumerate}


% FILE: chapter03.tex  Version 2.1
% AUTHOR:
% Universit�t Duisburg-Essen, Standort Duisburg
% AG Prof. Dr. G�nter T�rner
% Verena Gondek, Andy Braune, Henning Kerstan
% Fachbereich Mathematik
% Lotharstr. 65., 47057 Duisburg
% entstanden im Rahmen des DFG-Projektes DissOnlineTutor
% in Zusammenarbeit mit der
% Humboldt-Universitaet zu Berlin
% AG Elektronisches Publizieren
% Joanna Rycko
% und der
% DNB - Deutsche Nationalbibliothek

\chapter{Ergebnisse}

% Definition der Ergebnisse, nur hier berechnete Daten eingeben,
% sp�ter im Text dann nur per Variablennamen ansprechen.

% Beschreibung der analysierten Patienten
\newcommand{\ersterPatient}{Januar 2014}
\newcommand{\letzterPatient}{Juli 2015}
\newcommand{\nPatients}{337} % Gesamtzahl der Patienten
\newcommand{\nMale}{184}
\newcommand{\nFemale}{153}
\newcommand{\percentMale}{55}
\newcommand{\percentFemale}{45}
\newcommand{\alterMedian}{76}
\newcommand{\alterMin}{60}
\newcommand{\alterMax}{93}
\newcommand{\nUnterAltersDef}{8}
\newcommand{\percentGeriatrischOnkologie}{54}
% zwei Abbildungen: altersverteilung.pdf und altersverteilungAllgemein.pdf

% Assessmentkategorien

% Ende Ergebnisdefinition

\section{Analyse der Daten des geriatrischen Assessments}

\subsection{Beschreibung der analysierten Patienten}

Von \ersterPatient{} bis \letzterPatient{} wurden \nPatients{} Patienten in der
geriatrisch-onkologischen Konferenz besprochen, \nMale{} M�nner (\percentMale{}\%)
und \nFemale{} Frauen (\percentFemale{}\%).
Abbildung~\ref{fig:altersverteilung} zeigt ein Histogramm der 
Altersverteilung der Patienten.
Das mediane Alter betrug \alterMedian{} Jahre, der j�ngste Patient war
\alterMin{}, der �lteste \alterMax{} Jahre alt. Insgesamt wurde bei
\nUnterAltersDef{} Patienten, die j�nger waren als 65 Jahre, ein
Assessment durchgef�hrt.

\begin{figure}[htbp]
  \centering
  \includegraphics{Grafiken/altersverteilung.pdf}
  \caption{Alter der untersuchten Patienten}
  \label{fig:altersverteilung}
\end{figure}

Zur Einordnung zeigt Abbildung~\ref{fig:altersverteilungAllgemein} die
Altersverteilung aller in der onkologischen Klinik behandelten
Patienten aus dem gleichen Zeitraum. \percentGeriatrischOnkologie{}\%
der Patienten waren 65 Jahre oder �lter.

\begin{figure}[htbp]
  \centering
  \includegraphics{Grafiken/altersverteilungAllgemein.pdf}
  \caption{Alter aller onkologischer Patienten}
  \label{fig:altersverteilungAllgemein}
\end{figure}

Diagnosen der Patienten

\section{Umfassendes Geriatrisches Assessments}

\subsection{Assessmentkategorien}

Abbildung 4 zeigt als Boxplot die Verteilung des Barthel-Indexes in
den einzelnen Kategorien. Man sieht eine scharfe Auftrennung nach den
einzelnen Kategorien. Dabei ist - wie erwartet - der Durchschnitt von
nicht therapief�hig �ber eingeschr�nkt therapief�hig bis hin zu
uneingeschr�nkt therapief�hig zunehmend. Der Unterschied des
Durchschnitts des Barthel-Index in den einzelnen Kategorien ist dabei
signifikant. Der Barthel-Index trennt die einzelnen Kategorien also
gut auf.

Abbildung 4 Barthel-Index

Auch der IADL-Index zeigt eine deutliche und signifikante Trennung
zischen den einzelnen Beschlusskatekorien.

Abbildung 5 IADL-Index

Auch die Mobilit�tstests Tinetti und Timed up and go trennen
signifikant zwischen den Kategorien wie Abbildung 6 und 7 zeigen.

Bei den Kognitionstests MMST und Demtect sind ebenfalls Unterschiede
in den verschiedenen Kategorien erkennbar. Dabei f�llt jedoch auf,
dass der in der Erkennung einer beginnenden Demenz sensitivere DEMTECT
deutlich besser trennt, als der MMST.

Abbildung 7 MMST

Abbildung 8 Demtect

Bis August 2014 wurde innerhalb des Assessments als
Depressionsscreening der GDS durchgef�hrt. Seit September 2014 dann
der DIA-Score. Diese Entscheidung wurde nicht aufgrund von Daten
gef�llt, sondern wegen der Durchf�hrbarkeit. In der Anwendung hat sich
der f�r geriatrische Patienten ohne Tumorerkrankung entwickelte GDS aufgrund der
verwendeten Fragen bei onkologischen Patienten als problematisch
erwiesen. Manche Fragen sind bei einer Tumorerkrankung so
offensichtlich mit ja zu beantworten, dass erstens die Befragung
schwerf�llt und zweitens auch die Auswertung fraglich erscheint. Daher
wurde ab September 2014 der in der Anwendung und Befragung besser
angenommene DIA-Score verwendet. (Zahlen von GDS und DIA-S einf�gen)
In der Analyse der Ergebnisse f�llt eine bessere Trennung beim GDS
auf. Auch im DIA-S besteht ein messbarer Unterschied in den
verschiedenen Kategorien, jedoch ist die Variabilit�t innerhalb der
Kategorie gr��er.

Abbildung 9 zeigt die Erhebung der Komorbidit�t. Es besteht ein Trend
zu h�herer Komorbidit�t in den Kategorien. Die Variabilit�t ist jedoch
hoch und die Trennung nicht signifikant.

Abbildung 9 Charlson-Score

Auch die Ern�hrungsituation erfasst durch den MNA zeigt einen
Unterschied in den einzelnen Kategorien, jedoch ebenfalls mit einer
hohen Variabilit�t.

Das Sozialassessment wurde bei der Erhebung rein beschreibend erhoben
und in der Konferenz diskutiert, falls es die Behandlungf�higkeit
mitbetraf. Ein Parameter war in den Diskussionen in der Konferenz
dabei erheblich, n�mlich die Tatsache eines allein lebenden
Patienten. Bei der Durchf�hrung einer Chemotherapie ist die h�usliche
Unterst�tzung durch einen Lebenspartner oft erheblich. Aber auch die
�berwachung von Symptomen geht oft besser vom Partner als vom
Patienten aus. Aus diesem Grund wurde aus dem Sozialassessment dieser
Parameter generiert, und weiter analysiert. Abbildung 11 zeigt die
allgemeine H�ufigkeit und Abbildung 12 die H�ufigkeit in den einzelnen
Kategorien. Ausschlaggebend war dieser Parameter in der Konferenzdiskussion
oft in grenzwertigen Situationen, in denen zwischen
uneingeschr�nkter Therapief�higkeit und eingeschr�nkter
Therapief�higkeit entschieden werden musste. Entsprechend schl�gt sich
auch das in der H�ufigkeit wieder. Der Anteil der allein lebenden
Patienten ist in der eingeschr�nkt therapief�higen Gruppe
gr��er. Einige Patienten in dieser Gruppe w�ren wahrscheinlich mit entsprechender
Unterst�tzung und �berwachung durch den Partner als
uneingeschr�nkt therapief�hig eingesch�tzt worden.

\subsection{Konferenzbeschluss}

Abbildung 3 zeigt als Balkendiagramm die absolute Verteilung der
einzelnen Kategorien des Konferenzbeschlusses. 307 Patienten wurden
von Januar 2014 bis Juni 2015 in der geriatrisch-onkologischen
Konferenz besprochen. Bei 290 Patienten (94\%) wurde eine Empfehlung
ausgesprochen. 31 oder 11\% der beurteilten Patienten wurden als
nicht therapief�hig eingesch�tzt, 113
Patienten (39\%) als eingeschr�nkt therapief�hig und 146 Patienten
(50\%) als uneingeschr�nkt therapief�hig. Von den 17 Patienten (6\%),
bei denen kein Konferenzbeschluss zustande kam, waren zum
Zeitpunkt der Konferenz 2 Patienten verstorben. Bei 7 Patienten war
das Assessment so unvollst�ndig, dass eine Einsch�tzung nicht m�glich
war. Bei den restlichen 8 Patienten gab es verschiedene Gr�nde f�r
eine fehlende Konferenzentscheidung. Ein Patient war aufgrund des
Alters (62 Jahre) und fehlender spezifischer Probleme als nicht
geriatrisch klassifiziert worden; ein Patient hatte
keine Tumorerkrankung; bei einem Patienten gab es keine
Behandlungsindikation; bei drei Patienten aus den Anf�ngen der
Konferenz wurde als Beschluss
eine Therapie empfohlen, jedoch keine Angabe zur eingesch�tzten
Behandlungsf�higkeit gegeben;
bei einem Patienten wurde eine geriatrische Komplexbehandlung
empfohlen und bei einem Patienten ein Reassessment nach Abschluss
einer Hirnsch�delbestrahlung empfohlen.
Innerhalb der einzelnen Balken ist farbkodiert die pers�nliche
Einsch�tzung des Behandlers.

Abbildung 3
Balkendiagramm zur Verteilung der jeweiligen Konferenzbeschl�sse

Abbildung 3 zeigt als Balkendiagramm die absolute Verteilung der
einzelnen Kategorien des Konferenzbeschlusses. 307 Patienten wurden
von Januar 2014 bis Juni 2015 in der geriatrisch-onkologischen
Konferenz besprochen. Bei 290 Patienten (94\%) wurde eine Empfehlung
ausgesprochen. 31 oder 11\% der beurteilten Patienten wurden als
nicht therapief�hig eingesch�tzt, 113
Patienten (39\%) als eingeschr�nkt therapief�hig und 146 Patienten
(50\%) als uneingeschr�nkt therapief�hig. Von den 17 Patienten (6\%),
bei denen kein Konferenzbeschluss zustande kam, waren zum
Zeitpunkt der Konferenz 2 Patienten verstorben. Bei 7 Patienten war
das Assessment so unvollst�ndig, dass eine Einsch�tzung nicht m�glich
war. Bei den restlichen 8 Patienten gab es verschiedene Gr�nde f�r
eine fehlende Konferenzentscheidung. Ein Patient war aufgrund des
Alters (62 Jahre) und fehlender spezifischer Probleme als nicht
geriatrisch klassifiziert worden; ein Patient hatte
keine Tumorerkrankung; bei einem Patienten gab es keine
Behandlungsindikation; bei drei Patienten aus den Anf�ngen der
Konferenz wurde als Beschluss
eine Therapie empfohlen, jedoch keine Angabe zur eingesch�tzten
Behandlungsf�higkeit gegeben;
bei einem Patienten wurde eine geriatrische Komplexbehandlung
empfohlen und bei einem Patienten ein Reassessment nach Abschluss
einer Hirnsch�delbestrahlung empfohlen.
Innerhalb der einzelnen Balken ist farbkodiert die pers�nliche
Einsch�tzung des Behandlers.

Abbildung 3
Balkendiagramm zur Verteilung der jeweiligen Konferenzbeschl�sse

Todo: Tabelle erstellen mit allen Assessmentdaten, Mean, Range, p-Wert
bezogen auf Unterschiede in den Kategorien

\section{Statistisches Modell der Konferenzentscheidung}

\subsection{Pers�nliche Einsch�tzung des Behandlers und Konferenzentscheidung}

Bei der Vorstellung des Patienten in der Konferenz wurde der
behandelnde Assistenzarzt befragt, wie das Behandlungsteam bestehend
aus einem Assistenzarzt in Weiterbildung und dem zust�ndigen
onkologischen Oberarzt, wie der Patient eingesch�tzt wird. Diese
Einsch�tzung beruht auf dem klinischen Eindruck, bestehend aus
Diagnose, Stadium, Anamnese und klinischer Untersuchung ohne
Einbeziehung der erhobenen Assessmentdaten. Diese Einsch�tzung wurde
unter der Bezeichnung pers�nliche Einsch�tzung protokolliert.

Tabelle 1
Konferenzbeschluss versus Pers�nliche Einsch�tzung

Bei 229 Patienten gibt es sowohl eine pers�nliche Einsch�tzung der
behandelnden �rzte als auch einen Konferenzbeschluss. Insgesamt zeigt sich
eine bemerkenswerte �bereinstimmung und Pr�zision der pers�nlichen
Einsch�tzung. In 85\% der F�lle stimmen pers�nliche Einsch�tzung und
Konferenzbeschluss �berein. Bemerkenswert ist au�erdem, dass die
beobachteten Abweichungen lediglich benachbarte Einsch�tzungen
betreffen. So gibt es keinen Fall, der als nicht
therapief�hig eingesch�tzt wird, der nach Konferenzbeschluss
uneingeschr�nkt therapief�hig eingesch�tzt wird. Umgekehrt gibt es
ebenfalls keinen Fall, der von der Konferenz als nicht therapief�hig
eingesch�tzt wird, der in der pers�nlichen Einsch�tzung
uneingeschr�nkt therapief�hig bezeichnet wird. Alle 22 vom Behandler als nicht
therapief�hig eingesch�tzten Patienten sind laut Konferenzbeschluss
nicht therapief�hig. Und nur 4 Patienten, die in der pers�nlichen
Einsch�tzung eingeschr�nkt therapief�hig eingesch�tzt wurden, wurden
von der Konferenz letztendlich als nicht therapief�hig
klassifiziert. Abbildung 1 macht die einzelnen Abweichungen deutlich.

Abbildung 1
Grafische Darstellung des Konferenzbeschlusses versus Pers�nliche Einsch�tzung zur Fehlerklassifikation

In der Abstufung zwischen eingeschr�nkt und uneingeschr�nkt
therapief�hig kam es zu folgenden Abweichungen. 17 Patienten wurden im
Konferenzbeschluss hochgestuft, waren also im pers�nlichen Assessment
nur eingeschr�nkt therapief�hig eingesch�tzt worden, von der Konferenz
jedoch als uneingeschr�nkt therapief�hig. 14 Patienten wurden im
Konferenzbeschluss heruntergestuft, wurden also pers�nlich als
uneingeschr�nkt therapief�hig, von der Konferenz jedoch als
eingeschr�nkt therapief�hig klassifiziert.
Damit ist der positiv pr�diktive Wert des pers�nlichen Assessments wie
folgt f�r die einzelnen Kategorien: 88\% f�r uneingeschr�nkt
therapief�hig, 77\% f�r eingeschr�nkt therapief�hig und 100\% f�r
nicht therapief�hig.

Wie gerade ausgef�hrt betraf eine
Fehlklassifikation jedoch immer nur die benachbarte Kategorie.
Abbildung 1 illustriert in der Gegen�berstellung von
Konferenzentscheidung und per�nlicher Einsch�tzung die dargelegte
Situation. Dabei sind besonders gut die zwei m�glichen Arten von
Fehlern sichtbar.
Es sind prinzipiell zwei Arten von Fehler im pers�nlichen Assessment
m�glich. Entweder werden Patienten durch das pers�nliche Assessment
als zu gut oder zu schlecht eingestuft. Mit der jetzt beschriebenen
Klassifkation kommen beide Arten von Fehlern vor, 17 Patienten wurden
herab- (slow go, in der Konferenz go go) und 18
Patienten heraufgestuft (4 slow go, in der Konferenz no go und 14 Patienten
go go, in der Konferenz slow go). Wie bereits erw�hnt kommt es bei
einem Klassifikationsfehler also nur zu einer Fehlklassifikation in eine
benachbarte Kategorie.

An dieser bemerkenswerten Pr�zision der pers�nlichen Einsch�tzung ohne
Vorliegen eines Assessments muss sich ein statistisches
Vorhersagemodell messen lassen. 

Bei der Ansicht der Verteilung des Konferenzbeschlusses �ber die
jeweiligen Jahrg�nge ist die ungef�hre Gleichverteilung der laut
Beschluss nicht therapief�higen Patienten interessant. Bei der
Verteilung der als eingeschr�nkt therapief�hig beurteilten Patienten
ist wie erwartet jedoch ein Trend zum h�heren Lebensalter
erkennbar. Auch einzelne Patienten jenseits der 80
Lebensjahre sind als uneingeschr�nkt therapief�hig eingesch�tzt
worden. Das best�tigt den oft zitierten klinischen Eindruck der
h�heren Bedeutung des biologischen Alters gegen�ber dem
chronologischem Alter. Dies zeigt auch in einer etwas anderen Darstellung
ebenfalls die Abbildung 13.

Abbildung 13
Altersverteilung je nach Konferenzbeschluss

\subsection{Logistische Regression}

\subsection{KNN}

\subsection{Neuronale Netze}

% FILE: main.tex  Version 2.1
% AUTHOR:
% Universit�t Duisburg-Essen, Standort Duisburg
% AG Prof. Dr. G�nter T�rner
% Verena Gondek, Andy Braune, Henning Kerstan
% Fachbereich Mathematik
% Lotharstr. 65., 47057 Duisburg
% entstanden im Rahmen des DFG-Projektes DissOnlineTutor
% in Zusammenarbeit mit der
% Humboldt-Universitaet zu Berlin
% AG Elektronisches Publizieren
% Joanna Rycko
% und der
% DNB - Deutsche Nationalbibliothek

\chapter{Diskussion}

\epigraph{Get your facts first, and then you can distort them as much
  as you please.}{--- \textup{Mark Twain}}

\section{Vorbemerkungen}
\label{sec:vorbemerkungen}

Die vorliegende Arbeit beschreibt erstmals die Einbeziehung einer
geria\-trisch"=onkolo\-gischen Konferenz
(Abschnitt~\ref{subsec:konferenz}) in den Entscheidungsprozess der
Behandlung onkologischer Patienten. Bisher legt der behandelnde
Onkologe bestenfalls unter Einbeziehung einer interdisziplin�ren
Tumorkonferenz, die Behandlung geria\-trisch"=onkolo\-gischer
Patienten fest. Seit einigen Jahren w�chst im unter der Bezeichnung
Geriatrische Onkologie das Bewu�tsein, dass bei �lteren Patienten
zus�tzliche Probleme bestehen k�nnen, die entscheidend sowohl die
Toxizit�t der verabreichten Therapie als auch die Prognose
beeinflussen. Die seit vielen Jahren in der Geriatrie etablierte
Einbeziehung spezieller Assessmentwerkzeuge in die Behandlungsplanung
geriatrischer Patienten wird im Rahmen eines umfassenden geriatrischen
Assessments zunehmend auch in der Onkologie eingesetzt. Die
Bedeutung der erhobenen Daten ist derzeit jedoch weitgehend
ungekl�rt. Dabei steht im Zentrum der Forschung, welche Parameter die
Prognose des Patienten und das Ausma� der Toxizit�t einer
onkologischen Therapie am besten voraussagen k�nnen.

In einem ersten Vorschlag von Balducci und Kollegen
\autocite{Balducci2000} wurde dabei anhand der erhobenen Parameter
innerhalb eines Assessments eine Unterscheidung der Patienten in fit,
verletzlich und gebrechlich (fit, vulnerable, frail)
vorgenommen. Basierend auf dieser Klassifizierung kann in der Onkologie
nun eine Differenzierung der Patienten bez�glich der Therapief�higkeit
zwischen uneingeschr�nkt, eingeschr�nkt und nicht behandelbar
vorgenommen werden. Auch wenn Balducci und Kollegen in der erw�hnten
Arbeit ausdr�cklich darauf hinweisen, dass die vorgeschlagenen
Parameter und auch deren Grenzwerte lediglich ein erster Vorschlag
sind, die anhand neuer Erkenntnisse angepasst werden sollten, verfehlt
dieser Ansatz wichtige klinische Aspekte. Es ist w�nschenswert von der
subjektiven, Behandler-abh�ngigen Entscheidung des aktuell gelebten
klinischen Standards, hin zu einer objektivierbaren und
reproduzierbaren Behandlungsrealit�t innerhalb und au�erhalb
klinischer Studien zu gelangen. Jedoch geht mit der v�lligen
Ausklammerung der klinischen Erfahrung der behandelnden �rzte
sehr viel Information verloren. Auch durch ausgereifteste
Assessmentwerkzeuge kann diese Art der Information nicht erfasst
werden.

Daher ist meiner Ansicht nach bis zum prospektiven Nachweis eines
besseren Vorgehens, das optimale Verfahren, der Goldstandard, die
klinische Erfahrung sowohl der Behandler als auch eines mit den
Assessmentverfahren vertrauten Arztes, vorzugsweise einem Geriater, zu
kombinieren und auf Grundlage der in einem umfassenden geriatrischen
Assessment erhobenen Parameter eine Entscheidung vorzunehmen. Aufgrund
der besseren Evaluierbarkeit k�nnte die Dreiteilung vorerst �bernommen
werden, auch wenn durch immer differenziertere onkologische
Behandlungsverfahren auch eine noch feinere Differenzierung der
Beurteilung n�tig sein k�nnte.

\section{Patientenkollektiv}
\label{sec:patientenkollektiv}

Das hier analysierte Patientenkollektiv entpsricht dem einer breit
onkologisch t�tigen Klinik. Bemerkenswert ist dabei, dass mehr als die
H�lfte (56\%) der insgesamt an der Klinik behandelten Patienten die
hier verwendete Definition eines geria\-trisch"=onkolo\-gischen
Patienten, also 65 Jahre oder �lter, erf�llen. Dies unterstreicht die
klinische Relevanz des Themas.

Abbildung~\ref{fig:altersverteilung} zeigt die Altersverteilung der
dem Assessment zugef�hrten Patienten. Im Vergleich mit
der Altersverteilung aller behandelten Patienten
Abbildung~\ref{fig:altersverteilungAllgemein}, f�llt auf, dass 65 bis 70
j�hrige Patienten im Assessment unterrepr�sentiert sind, also in
dieser Altersgruppe relativ weniger Patienten dem Assessment zugef�hrt
wurden. Das k�nnte anzeigen, dass auch an einer Klinik
mit seit einigen Jahren etabliertem Bewu�tsein f�r die Belange der
geriatrischer Onkologie mit entsprechendem Assessment und Konferenz,
trotzdem st�ndige Werbung, Information und Verbesserung der Abl�ufe
n�tig sind, damit auch wirklich alle Patienten einem geriatrischen
Assessment und der anschlie�enden Beurteilung zugef�hrt werden k�nnen.
Ein weiterer m�glicher Grund w�re die ambulante Behandlung in der
onkologischen Tagesklinik, in der ein Assessment bisher aus
organisatorischen Gr�nden nicht durchf�hrbar ist. J�ngere Patienten
werden relativ h�ufiger rein ambulant behandelt als �ltere Patienten.
Um auch dort den Bedarf zu erfassen und beide aufgef�hrten m�glichen
Gr�nde unterscheiden zu k�nnen, ist bei ambulanten Patienten, die 65
Jahre oder �lter sind, ein Screening geplant. Damit soll unterschieden
werden, welche Patienten weiter rein ambulant, also ohne die
Durchf�hrung eines umfassenden geriatrischen Assessments,
behandelt werden k�nnen, und welche Patienten station�r einem
Assessment und damit der Beurteilung in der geriatrisch-onkologischen
Konferenz zugef�hrt werden sollten.

Bei den zw�lf Patienten die j�nger als 65 Jahre alt waren, ist die in
dieser Arbeit verwendete formale Definition eines
geria\-trisch"=onkolo\-gischen Patienten nicht erf�llt. Von diesen
zw�lf Patienten wurden jedoch lediglich vier Patienten als
uneingeschr�nkt therapief�hig durch die Konferenz eingesch�tzt. Bei
keinem dieser Patienten war das durchgef�hrte Assessment unauff�llig
(Tabelle~\ref{tab:defiziteAlter}). Es handelt sich also um Patienten,
die zwar formal der Altersdefinition nicht gen�gten, bei denen jedoch
trotz des Alters bereits relevante Probleme bestanden und damit ein
Assessment durchaus gerechtfertigt war. Auch in der Geriatrie wird aus
diesem Grund eine starre Altersdefinition f�r einen geriatrischen
Patienten vermieden (Abschnitt~\ref{sec:begriffe}).

Die Verteilung der Diagnosen bei den untersuchten Patienten entspricht
weitgehend den entsprechenden epidemiologischen H�ufigkeiten der
Tumore und damit dem klinischen Alltag einer onkologischen Klinik.

Gr��e und Gewicht und damit auch der daraus berechnete \ac{BMI} sind
normalverteilt. Durchschnitt und Standardabweichung sind in
Tabelle~\ref{tab:patienten} aufgef�hrt.

\section{Ergebnisse des Umfassenden Geriatrischen Assessments}

Die wichtigsten beschreibenden Merkmale der Verteilungsfunktionen der
Messwerte ist in Tabelle~\ref{tab:cga_summary} zusammengefasst. Bei
allen Variablen au�er dem Sozialassessment handelt es sich um
Ordinalskalen, bei diesen wurde zur Beschreibung der Median dem
Durchschnitt als Ma� des Mittelwerts vorgezogen.
Das Sozialassessment wird in einer Nominalskala ausgedr�ckt, die aus
dem beim Assessment gewonnenen beschreibenden Text erstellt gefolgert
wurde (Abschnitt~\ref{sec:sozial}). Die gemessenen Variablen folgen
keiner �blichen Verteilungsfunktion, daher kommen als
statistische Tests nicht-parametrische Verfahren zur Anwendung. Alle
erhobenen Parameter sind zu mindestens als 85\%
vollst�ndig. Lediglich der \ac{DemTect} wurde nur in 79\% der F�lle
durchgef�hrt; die h�ufigsten Gr�nde der Antwortausf�lle in diesem
Bereich waren Sprachproblemen oder eine bereits erkennbare
�berforderung der Patienten w�hren des \ac{MMST}.
Abbildung~\ref{fig:cga_summary_boxplot} beschreibt die Verteilung der
einzelnden Parameter grafisch.

Bei der Analyse der erhobenen Daten des \ac{MMST} und des \ac{DemTect}
ist erw�hnenswert, dass auch in der vorgestellten Population die h�here
Sensitivit�t zur Erkennung einer Demenz durch den \ac{DemTect} gegen�ber
dem \ac{MMST} erkennbar wird. Beim \ac{MMST} haben 78\% der untersuchten ein
normales Testergebnis, beim DemTect lediglich 62\%. In der Zeit, in
der zum Assessment nur der MMST zur Anwendung kam, gab es regelm��ig
F�lle, bei denen ein DemTect nachtr�glich angefordert werden musste,
um eine bessere Einsch�tzung der kognitiven Funktion zu gewinnen.
Um die entstehende Verz�gerung zu vermeiden wurde seit Februar 2015
der \ac{DemTect} zus�tzlich routinem��ig durchgef�hrt.
Ist aus Zeitgr�nden die Durchf�hrung nur eines Tests m�glich,
sollte bevorzugt der \ac{DemTect} zur Anwendung kommen, da bei
onkologischen Therapien auch milde kognitive Beeintr�chtigungen, die
beim \ac{MMST} m�glicherweise unter der Nachweisgrenze liegen,
relevant sein k�nnen.

Beim Depressionsassessment wurde durch die \ac{GDS} bei 23\% der
untersuchten Patienten, durch die \ac{DIA}-Skala bei 48\% der
Patienten ein auff�lliges Testergebnis gemessen. Da es unpraktikabel
ist, seriell zwei Depressionstests bei einem Patienten durchzuf�hren,
wurde nach der getroffenen Entscheidung den \ac{GDS} durch die
\ac{DIA}-Skala zu ersetzen, lediglich letztere weiterhin
angewendet. Geht man davon aus, dass sich die
Patientenpopulation ab diesem Zeitpunkt nicht relevant ge�ndert hat,
f�llt auf, dass die DIA-S zu einem wesentlich h�heren Prozentsatz
auff�llige Werte misst. Eine Depression bei onkologischen Patienten
ist h�ufig, dabei sind die ausl�senden Mechanismen
sicherlich andere, als bei psychiatrischen Patienten. Oft wird auch
von einer reaktiven Depression gesprochen, also einer schweren
Traurigkeit aufgrund eines ad�quaten ausl�senden Ereignisses. Die
Mitbetreuung dieser Patienten durch speziell geschulte Berufsgruppen
wie zum Beispiel Psychoonkologen ist notwendig und wird von vielen
Patienten und auch deren Angeh�rigen als hilfreich empfunden.

\subsection{Anzahl der Defizite}
\label{sec:defizite_discussion}

Angelehnt an die Publikation von Wedding \autocite{Wedding2007}
wurde die Anzahl der Defizite definiert (Abschnitt~\ref{sec:defizite})
und bei den hier untersuchten Patienten bestimmt (siehe
Tabelle~\ref{tab:defizite}). Lediglich 12\% der Patienten, bei denen
alle definierten Tests vorlagen hatten �berall normale
Testergebnisse. Dies zeigt, dass der Gro�teil der untersuchten 
Patienten Auff�lligkeiten aufweist, die von etablierten
Assessmentwerkzeugen gemessen werden k�nnen.
Das sind dann Ansatzpunkte, an denen eine
Therapie direkt nach dem Assessment ansetzen kann, um Defizite
positiv zu beeinflussen. Die Mobilit�t kann durch Physiotherapie und
die Bereitstellung von Hilfsmitteln optimiert werden, der
Ern�hrungsstatus kann durch entsprechende Beratung durch geschultes
Personal verbessert werden, die Behandlung der relevanten
Komorbidit�ten kann �berpr�ft und gegebenenfalls optimiert werden,
selbst die kognitive Situation kann durch Trainingsma�nahmen positiv
beeinflusst werden.

Es zeigt aber auch, dass die Altersgrenze bei 65 Jahren
nicht zu tief angesetzt ist. In der Gruppe der 65 bis 69 Jahre alten
Patienten weisen 31 Patienten (84\%) nach oben genannter Definition
mindestens ein Defizit auf und lediglich sechs Patienten keins
(Tabelle~\ref{tab:defiziteAlter}. Selbst bei den unter 65 j�hrigen
Patienten kann ein umfassendes geriatrisches Assessment gerechtfertigt
sein. Alle 12 Patienten, die formal f�lschlicherweise einem Assessment
zugef�hrt wurden wiesen Defizite auf.

Wenn bei der Definition der Defizite streng der Publikation von
Wedding \autocite{Wedding2007} gefolgt wird, sind beide untersuchten
Populationen diesbez�glich vergleichbar. In dem von Wedding
untersuchten Kollektiv finden sich bei 27\% der Patienten keine
Defizite und bei 4\% die maximale Anzahl von f�nf Defiziten. Auf die
hier untersuchten Patienten angewandt, ergeben sich bei 19\% keine
Defizite und bei 6\% f�nf Defizite. Bei Wedding wurden 200 Patienten
ab 70 Jahren untersucht, die in einer onkologischen Praxis, also
ambulant behandelt wurden. Hier wurden �ber 400 Patienten ab 65 Jahren
untersucht, die in einer onkologischen Klinik station�r behandelt
wurden. Trotz des j�ngeren Ausl�sealters eines Assessments im
station�ren Bereich ist die untersuchte Poplulation
bez�glich der Zahl der Defizite st�rker belastet als die ab 70
j�hrigen Patienten in der ambulanten Behandlung. Es l�sst sich also
schlussfolgern, dass in der station�ren Behandlung ein geringeres
Ausl�sealter f�r die Durchf�hrung eines geriatrischen Assessments von
generell 65 Jahren durchaus angemessen ist. Im Einzellfall sollten
weiterhin auch Patienten, die j�nger sind als 65 Jahre einem
Assessment zugef�hrt werden.

\subsection{Kolinearit�t der unterschiedlichen Tests}

Es besteht eine gro�e Kolinearit�t der verschiedenen
im Assessment ermittelten Testwerte, ausgedr�ckt im
Rangkorrelationskoeffizienten wie in Tabelle~\ref{tab:correlation}
angegeben. Dabei ist eine Kolinearit�t von Tests aus dem selben
Bereich, also Tinetti und TUG (-0,86) bez�glich der Mobilit�t, aber auch
DemTect und MMST (0,65) direkt nachvollziehbar. Nat�rlich ist aber
auch die F�higkeit zur Selbstversorgung (Barthel-Index) und der
Teilnahme am normalen Leben (IADL) stark von der Mobilit�t abh�ngig,
was die hohen Korrelationskoeffizienten auch wiederspiegeln
(Barthel-Tinetti 0,73; Barthel-\ac{TUG} -0,63; \ac{IADL}-Tinetti 0,52;
\ac{IADL}-\ac{TUG} -0,47).

Bemerkenswert ist aber auch die relativ geringe Kolinearit�t des
Ern�hrungszustandes gemessen durch den BMI und gemessen durch den
MNA. Hier ergibt sich nur eine Korrelation von 0,50. Dies
unterstreicht die Relevanz eines zus�tzlichen Ern�hrungsassessments
und der potentiell wichtigen zus�tzlichen Information neben der
alleinigen Gewichtsmessung.

\section{Verschiedene Einsch�tzungsm�glichkeiten}

Wie bereits zu Beginn des Abschnitts beschrieben
(Abschnitt~\ref{sec:vorbemerkungen}) wird die Therapie eines
geriatrisch-onkologischen Patienten vom Behandler ausgew�hlt. Dabei
spielt die pers�nliche Erfahrung und Einstellung des Behandlers eine
gro�e Rolle. Unter Ber�cksichtigung des pers�nlichen Eindrucks aus einem
Gespr�ch mit dem Patienten allein, bestenfalls auch mehreren
Gespr�chen mit dem Patienten unter Einbeziehung seiner Angeh�rigen und
einer klinische Untersuchung treffen H�matologen/Onkologen tagt�glich
Therapieentscheidungen. Auch das Alter der Patienten spielt dabei eine
Rolle. 

Schon immer in der Onkologie musste der behandelnde Arzt entscheiden,
welchen seiner Patienten er eine potentiell t�dliche Chemotherapie
zutraut und welchen nicht. Mit der Entwicklung immer zahlreicherer
Chemotherapiekombinationen mit jeweils eigenem Nebenwirkungs- und
Risikoprofil wurde eine weitere Unterteilung in verschiedene Grade der
Behandlungsf�higkeit n�tig. Zum Beispiel sind heute bei der palliativen
Prim�rtherapie des Pankreaskarzinoms vier Optionen denkbar. Zur
Auswahl stehen eine Behandlung im Rahmen eines best supportive care
Konzeptes, also der alleinigen Behandlung von Beschwerden und
Problemen. Weiterhin ist eine Monotherapie mit Gemcitabin, eine
Zweifachkombinationstherapie aus Gemcitabin und nab-Paclitaxel und
eine Dreifachkombinationstherapie aus Oxaliplatin, Irinotecan und
5-Fluorouracil m�glich. Diese drei Therapieoptionen haben eine
zunehmende Wirksamkeit, jedoch auch ein zunehmend problematischeres
Nebenwirkungsprofil. Es ist also auch eine feinere Unterteilung der
Behandlungsf�higkeit als in nur zwei Kategorien denkbar.

Tabelle~\ref{tab:classification} stellt die in der vorliegenden Arbeit
untersuchten, verschiedenen Einsch�tzungsm�glichkeiten tabellarisch dar. 
Dieser Vergleich gibt einen �berblick �ber die verschiedenen
Einsch�tzungsm�glichkeiten und die relativen H�ufigkeiten der
verschiedenen Unterteilungen.

\subsection{Einteilung nach Alter}
\label{subsec:alter}

Bis heute hat sich sich eine fast absolute Altersgrenze, nach der
bestimmte Therapien entweder durchgef�hrt werden oder nicht in der
Hochdosistherapie sowohl mit anschlie�ender autologer, als auch mit
allogener Knochenmarktransplantation. Auch wenn sich durch die moderne
Supportivtherapie die Altersgrenze f�r beide Therapieformen immer
weiter nach hinten verschoben hat, gilt bei sonst fehlender
Komorbidit�t und fitten Patienten f�r die allogene
Transplantation ein maximales Alter von 70 Jahren, f�r die autologe
Transplantation von 75 Jahren.

Au�erdem galt bis vor kurzem noch ein Alter von meist mehr als 65 oder
70 Jahren f�r den Einschluss von Patienten in klinische Studien als
Ausschlusskriterium.

Und auch bei der pers�nlichen Entscheidung des Behandlers spielt in
der Beurteilung der Behandlungsf�higkeit des Patienten das Alter eine
gro�e Rolle.

Tabelle~\ref{tab:cga_alter} zeigt die verschiedenen Mediane der
erhobenen Parameter nach Unterteilung der untersuchten Patienten in
drei verschiedene Alterskategorien.
Eine Dreiteilung der untersuchten Patientenpopulation wurde zur
besseren Vergleichbarkeit mit der Dreiteilung der Behandlungsf�higkeit
vorgenommen. Wie auch die Dreiteilung der Behandlungsf�higkeit sind
auch andere Unterteilungen denkbar. Eingeteilt wurde in 65 bis 74
Jahre alte Patiente, in 75 bis 84 Jahre und in �ber 85 Jahre alte
Patienten. In der letzten Gruppe war ebenfalls das Intervall des
Alters zehn Jahre, da die �ltesten untersuchten Patienten 94 Jahre alt
waren. Eine signifikante Trennung (Kruskal-Wallis-Test) zeigt
sich bei allen durchgef�hrten Tests, au�er den Depressionsassessments
(sowohl \ac{GDS} als auch \ac{DIA}), der Komorbidit�t und dem
Ern�hrungsstatus. Eine unterschiedliche H�ufigkeit alleinlebender
Patienten wurde nicht beobachtet.

Die mediane Anzahl der Defizite betrug in der Gruppe der 75 bis 84
j�hrigen und der 85 bis 94 j�hrigen 3, bei den unter 75 j�hrigen
2. Dieser Unterschied ist signifikant, jedoch gering und unterstreicht
die Tatsache, dass Therapieentscheidungen allein aufgrund des Alters
nicht sinnvoll sind.

\subsection{Einteilung nach Vorschlag von Balducci}

Nach dem Vorschlag von Balducci \autocite{Balducci2000} wird aufgrund
des Barthel-Index, des \ac{IADL} und des Charlson-Komorbidit�tsindex
eine Einteilung in die drei Kategorien fit (fit), gef�hrdet
(vulnerable) und gebrechlich (frail) vorgenommen. Es findet also eine
Einteilung anhand eines Teiles des umfassenden geriatrischen
Assessments statt. Da alle drei Parameter auch in dem hier
durchgef�hrten Assessment enthalten waren, k�nnen die untersuchten
Patienten auch nach dem Vorschlag von Balducci eingeteilt
werden. Tabelle~\ref{tab:cga_balducci} zeigt die Mediane der
zus�tzlich im Assessment gemessenen Werte.

Die Einteilung durch den Vorschlag von Balducci trennt die Patienten
in Gruppen mit signifikant unterschiedlicher Mobilit�t und kognitiver
Leistungsf�higkeit. Aber auch, anders als in der Unterteilung aufgrund
des Alters, das Depressionsassessment und das Ern�hrungsassessment
f�llt nun in den verschiedenen Gruppen signifikant unterschiedlich
aus. Lediglich die Anzahl der alleinlebenden Patienten unterscheidet
sich in den verschiedenen Gruppen nicht. 
Erkl�rt werden k�nnen die signifikanten Unterschiede allein mit dem
hohen Ma� an Kolinearit�t zwischen den im Assessment durchgef�hrten
Tests. Der Barthel-Index, der \ac{IADL} und der
Charlson-Komorbidit�tsindex f�hren zur Klassifikation nach dem
Vorschlag nach Balducci, also sind dann die dazu hochgradig kolinearen
Tests ebenfalls unterschiedlich.

Bemerkenswert sind jedoch die H�ufigkeiten der einzelnen Kategorien.
Auff�llig ist ein sehr hoher Anteil von Patienten, die von dieser
Klassifikation als gebrechlich eingesch�tzt werden. Mit 205 Patienten (55\%)
ist die Gruppe der gebrechlichen Patienten mit Abstand die Gr��te. Nur
55 Patienten (15\%) werden als fit klassifiziert, 111 Patienten (30\%)
als gef�hrdet. Durch Therapieentscheidungen, die aufgrund dieser
Klassifikation getroffen w�rden, w�re die Gefahr einer �bertherapie
der Patienten gering. Dem gegen�berstehend w�re allerdings die Gefahr
gro�, Patienten, die von einer onkologischen Therapie profitieren
w�rden, sowohl bez�glich Symptomkontrolle also Lebensqualit�t als auch
Gesamt�berleben, eine solche Therapie f�lschlicherweise vorzuenthalten.

\subsection{Einteilung nach Einsch�tzung des Behandlers}

Tabelle~\ref{tab:cga_personal} zeigt die verschiedenen Mediane der
erhobenen Parameter nach Unterteilung durch die Beurteilung des
Behandlers.

Die Einsch�tzung des Behandlers also der Quasi-Standard nach der
aktuell die meisten Therapieentscheidungen getroffen werden trennt die
verschiedenen innerhalb des Assessments durchgef�hrten Tests
bemerkenswert auf. Dabei sind die H�ufigkeiten der einzelnen
Kategorien deutlich abweichend von der Einteilung nach dem Vorschlag
nach Balducci. Mit 165 Patienten (46\%) ist die Gruppe der
uneingeschr�nkt behandlungsf�higen Patienten (fit) mit Abstand die
Gr��te. Nur 44 Patienten (12\%) werden als nicht behandlungsf�hig
(gebrechlich) klassifiziert, 147 Patienten (41\%) als eingeschr�nkt
behandlungsf�hig (gef�hrdet). Therapieentscheidungen, die aufgrund der
pers�nlichen Einsch�tzung des Behandlers getroffen werden, tragen also
eine gro�e Gefahr einer �bertherapie und nur eine geringe Gefahr,
Patienten f�lschlicherweise eine n�tzliche Therapie vorzuenthalten. Im
Vergleich mit der Einteilung nach dem Vorschlag von Balducci also eine
genau entgegengesetzte Situation.

Die sehr genaue Trennung der Gruppen bez�glich der durchgef�hrten
Tests zeigt an, dass das Gespr�cht mit dem Patienten mit der
anschlie�enden klinischen Untersuchung kombiniert mit der klinische
Erfahrung des Behandlers ein wertvolles Instrument ist, Patienten
einzusch�tzen. Ohne das Vorliegen der Assessmentdaten gelingt den
Behandlern, wozu sonst ein zus�tzliches multiprofessionelles
Assessment n�tig ist. Im Vergleich mit der Einteilung nach dem
Vorschlag nach Balducci ist die Trennung innerhalb der Gruppen sogar
besser. Das zeigt an, das trotz des hohen Anteils an uneingeschr�nkt
therapief�hig eingesch�tzten Patienten diese nicht wahllos zu gut
eingesch�tzt werden, sondern dass dies mit einem hohen Ma� an Gesp�r
f�r die Leistungsf�higkeit der Patienten, gemessen durch die Tests im
Assessment, geschieht.

Der einzige bei dieser Art der Einteilung problematische Punkt f�llt
den Behandlern w�hrend der entweder intuitiven stattfinden, oder wie
der hier durchgef�hrten expliziten, in drei verschieden Gruppen
benannten Einteilung t�glich auf: Das m�gliche Entgehen wichtiger,
Therapieentscheidender Details wie zum Beispiel eine mild ausgepr�gt
Demenz, oder eine gro�e Abh�ngigkeit bei Verrichtungen des t�glichen
Lebens von Angeh�rigen. Gewisse kleine aber wichtige Details k�nnen
dem Behandler im klinischen Kontakt entgehen, k�nnten jedoch bei
Vorhandensein die pers�nliche Einsch�tzung entscheidend
ver�ndern. 

\subsection{Einteilung nach Konferenzbeschluss}

Die soeben angesprochenese Problematik der pers�nlichen Einsch�tzung
versucht die geriatrisch-onkologische Konferenz zu beheben. Bei dieser
Art der Entscheidung der Behandlungsf�higkeit soll die klinische
Erfahrung der Behandler mit dem Vorliegen objektivierbarer Daten aus
einem umfassenden geriatrischen Assessment kombiniert werden und unter
Ber�cksichtigung der Art, des Stadiums und der speziellen
individuellen Problematik der Erkrankung eine Therapieentscheidung
getroffen werden.

Auch der Konferenzbeschluss trennt in Gruppen mit signifikant
unterschiedlichen Medianen in allen erhobenen Kategorien au�er der
Zahl der Alleinlebenden auf, wie Tabelle~\ref{tab:cga_konferenz} zeigt.
Die H�ufigkeiten der einzelnen Kategorien �hneln stark der Einteilung
durch den Behandler, weichen also ebenfalls deutlich von der
Einteilung nach dem Vorschlag nach Balducci ab. Mit 183 Patienten
(47\%) ist die Gruppe der uneingeschr�nkt behandlungsf�higen Patienten
(fit) mit Abstand die Gr��te. Nur 59 Patienten (15\%) werden als nicht
behandlungsf�hig (gebrechlich) klassifiziert, 150 Patienten (38\%) als
eingeschr�nkt behandlungsf�hig (gef�hrdet). Therapieentscheidungen,
die aufgrund der Einteilung durch die Konferenz getroffen werden,
tragen also wie bei der Einsch�tzung durch den Behandler eine gro�e
Gefahr einer �bertherapie und nur eine geringe Gefahr, Patienten
f�lschlicherweise eine n�tzliche Therapie vorzuenthalten. Im 
Vergleich mit der Einteilung nach dem Vorschlag von Balducci also
ebenfalls eine genau entgegengesetzte Situation.

Bemerkenswert ist ein erstmals signifikanter Unterschied im Anteil der
alleinlebenden Patienten zwischen den laut Konferenz eingeschr�nkt und
uneingeschr�nkt behandelbaren Patienten. Der Anteil der alleinlebenden
Patienten ist in der Gruppe der eingeschr�nkt behandelbaren Patienten
signifikant h�her. Wie in Abschnitt~\ref{sec:sozial} beschrieben war
ein wichtiger und wiederkehrender Diskussionspunkt in den Konferenzen
die soziale Situation, speziell die Einbindung in ein funktionierendes
soziales Netz. Man denke zum Beispiel an ein neutropenisches Fieber
bei einem Patienten, der zus�tzlich eine milde kognitive Beentr�chtigung hat
und daher der vorab gemachten Handlungsanweisungen nicht Folge leisten
kann. Wenn in dieser Situation der Lebenspartner oder andere
Bezugspersonen integriert wird, kann das f�r die Durchf�hrung
einer Therapie entscheidend sein. Damit erkl�rt sich der beobachete
Unterschied. Alleinlebende Patienten schnitten in den Tests also nicht
schlechter ab, sondern die geriatrisch-onkologische Konferenz sch�tzte bei
manchen Patienten das n�chste soziale Umfeld f�r so entscheidend ein,
dass zusammen mit den anderen erhobenen Befunden dadurch entweder die
Empfehlung eingeschr�nkt oder uneingeschr�nkt therapief�hig
gerechtfertigt war.

Im Vergleich mit der Trennung durch die pers�nliche Einsch�tzung des
Behandlers f�llt auf, dass die Werte des Depressionsassessment und der
Komorbidit�t nicht so deutlich unterschiedlich sind, wie bei Trennung
der Gruppen durch den Behandler. Dies k�nnte ein Hinweis darauf sein,
dass bei der klinischen Einsch�tzung die Komorbidit�t und eine
Depression den klinischen Eindruck st�rker beeinflussen, als dies in
der Abw�gung der einzelnen Punkte, wie sie in der Konferenz
stattfindet, der Fall ist.

Zur zus�tzlichen Analyse des Konferenzbeschlusses zeigt
Abbildung~\ref{fig:beschlussverteilung} die Altersverteilung der
Patienten jeder einzelnen Kategorie. Die Antwortausf�lle sind �ber die
gesamte Altersspanne verteilt. Das best�tigt die in
Abschnitt~\ref{subsec:na} aufestellte Behauptung, dass es sich bei den
fehlenden Daten um unabh�ngiges und zuf�lliges Fehlen
handelt. Interessant ist die Altersverteilung der als nicht
therapief�hig eingesch�tzten Patienten. Diese sind �ber die gesamte
Altersspanne verteilt und ohne erkennbare H�ufung. Die
hier erfolgte geriatrisch-onkologische Einsch�tzung der
Nicht-Therapief�higkeit von Patienten ist also generell
altersunabh�ngig und stattdessen von anderen im Assessment
und der klinischen Einsch�tzung erhobenen Befunden abh�ngig. F�r die
an der Einsch�tzung beteiligten �rzte ist also das Alter allein kein
Ausschlusskriterium f�r eine onkologische Therapie. Die
Altersverteilung der als uneingeschr�nkt therapeif�hig eingesch�tzten
Patienten gleicht der Altersverteilung aller eingeschlossener
Patienten (Abbildung~\ref{fig:altersverteilung}). Bemerkenswert ist
die Einsch�tzung auch �ber 80 j�hriger Patienten als uneingeschr�nkt
therapief�hig. Die als eingeschr�nkt therapief�hig eingesch�tzten
Patienten sind ebenfalls fast gleichm��ig �ber die gesamte
Altersspanne verteilt mit einer leichten H�ufung bei �lteren
Patienten.

\subsection{Vergleich verschiedener Einsch�tzungsm�glichkeiten
  untereinander}

Im vorangegangenen Abschnitt wurden die Testergebnisse in Abh�ngigkeit
von den einzelnen Einteilungsm�glichkeiten diskutiert. Hier sollen die
verschiedenen Einsch�tzungsm�glichkeiten miteinander verglichen
werden. Es geht also um Unterschiede und Gemeinsamkeiten in der
Beurteilung einzelner Patienten durch die verschiedenen
Einsch�tzungsm�glichkeiten. Da vier verschiedene
Einsch�tzungsm�glichkeiten (Alter, Behandler, Balducci, Konferenz)
vorgestellt wurden sind prinzipiell sechs Vergleiche m�glich. Jede der
verschiedenen Einsch�tzungsm�glichkeiten mit den jeweils
anderen. Dabei sind nicht alle Vergleichsm�glichkeiten gleich
sinnvoll. Herausgegriffen und n�her diskutiert werden folgende Vergleiche:

\begin{enumerate}
\item Einteilung durch das Alter und Einteilung durch den Behandler,
  da in den Anf�ngen der Onkologie haupts�chlich die Altersgruppe die
  Therapie vorgab und heute haupts�chlich der Behandler diese
  Entscheidung trifft.
\item Einteilung durch den Behandler und Einteilung nach Vorschlag
  von Balducci, da die Einteilung durch den Behandler den aktuellen
  Standard darstellt und ein abgespecktes Assessment und die
  anschlie�ende Einteilung nach dem Vorschlag von Balducci eine
  M�glichkeit darstellt den aktuellen Standard abzul�sen.
\item Einteilung durch den Behandler und Einteilung durch die
  geriatrisch-onkologische Konferenz nach Durchf�hrung eines
  umfassenden geriatrischen Assessments, da die Einteilung durch den
  Behandler den aktuellen Standard darstellt und die Einteilung durch
  die geriatrisch-onkologische Konferenz eine weitere M�glichkeit
  darstellt, den aktuellen Standard abzul�sen.
\item Einteilung nach Vorschlag von Balducci und Einteilung durch die
  geriatrisch-onkologische Konferenz, da beides neue M�glichkeiten
  darstellen, den aktuellen Standard abzul�sen und ich daher die Frage
  stellt, ob die wirklich eine aufw�ndigere Konferenz n�tig ist und
  nicht die einfachere Einteilung durch den Vorschlag von Balducci
  ausreicht.
\end{enumerate}

Die restlichen denkbaren Vergleiche, Einteilung nach Alter und
Einteilung nach Vorschlag von Balducci und Einteilung nach Alter und
Einteilung nach Konferenzbeschluss ergeben keine zus�tzlichen
Informationen.

\subsubsection{Einteilung nach Alter und nach Einsch�tzung des Behandlers}

Tabelle~\ref{tab:classification} zeigt die H�ufigkeiten der
verschiedenen Kategorien der verschiedenen
Einteilungsm�glichkeiten. Die H�ufigkeiten der Kategorien bei
Einteilung nach dem Alter und der pers�nlichen Einsch�tzung sind
�hnlich. Tabelle~\ref{tab:alter_personal} im Ergebnisteil
zeigt die Ergebnisse f�r individuellen Patienten. Eine Unabh�ngigkeit
beider Einsch�tzungm�glichkeiten l�sst sich statistisch ausschlie�en.

Bei 346 Patienten ist sowohl das Alter als auch die pers�nliche
Einsch�tzung der behandelnden �rzte bekannt. Folgt man der in
Abschnitt~\ref{subsec:alter} erl�uterten Alterseinteilung beobachtet
man immerhin in 54\% der Patienten eine �bereinstimmung. Interessant
ist ein Blick auf die von der Diagonale abweichenden F�lle. Ein
bestimmtes Muster zeigt sich hier nicht. Vom Behandler werden
Patienten in der �ltesten Patientengruppe (85 bis 94 Jahre) noch als
eingeschr�nkt und in einem Fall sogar als uneingeschr�nkt
therapief�hig eingesch�tzt. Auf der anderen Seite werden vom Behandler
auch in der j�ngsten Patientengruppe (65 bis 74 Jahre) auch zahlreiche
Patienten als eingeschr�nkt und auch als nicht therapief�hig
eingesch�tzt. Insgesamt ist die Korrelation von Alter und Einsch�tzung
des Behandlers zwar vorhanden, jedoch nicht sehr gut. Die zeigt, dass
die Behandler bei den hier untersuchten Patienten ihre
Therapieentscheidungen generell nicht allein auf das Alter st�tzen.

\subsubsection{Einsch�tzung des Behandlers und Einteilung nach
  Vorschlag von Balducci}

Die H�ufigkeiten der verschiedenen Kategorien der Einteilung nach dem
Alter und der pers�nlichen Einsch�tzung sind entgegengesetzt.
Tabelle~\ref{tab:personal_balducci} im Ergebnisteil
zeigt die Ergebnisse f�r individuellen Patienten. Eine Unabh�ngigkeit
beider Einsch�tzungm�glichkeiten l�sst sich auch hier statistisch
ausschlie�en.

Bei 322 Patienten ist sowohl die die pers�nliche Einsch�tzung der
behandelnden �rzte als auch die Einteilung nach dem Vorschlag von
Balducci bekannt. In nur 34\% der Patienten beobachtet man eine
�bereinstimmung. Interessant ist hier das Muster der
Abweichungen. Lediglich bei 6 Patienten ist die Einteilung durch den
Vorschlag von Balducci optimistischer. In allen anderen 207
abweichenden F�llen ist die pers�nliche Einsch�tzung des Behandlers
optimistischer. Bei der Einteilung durch den Vorschlag von Balducci
handelt es sich also verglichen mit dem aktuellen Standard der
Einsch�tzung durch den Behandler um eine extrem konservative
Beurteilung. W�rde man in der t�glichen Praxis dieser Einteilung
folgen, w�rde man vielen Patienten eine onkologische Therapie
vorenthalten. M�glicherweise ist bei diesen Patienten die Gefahr nicht
zu tolerierender Toxizit�t wirklich erh�ht. Vor einem so radikalen
Wechsel der t�glichen Praxis w�ren hierzu jedoch prospektive Daten
zwingend erforderlich. Gegebenenfalls w�re auch eine Modifikation des
auch von Balducci und Kollegen \autocite{Balducci2000} so genannten
Vorschlags der Einteilung ausreichend, m�glicherweise unter
Einbeziehung weitere in einem umfassenden geriatrischen Assessment
gewonnener Daten. Die prinzipielle Frage lautet dabei: Ist ein
umfassendes geriatrisches Assessment allein ausreichend und in der
Lage die klinische Einsch�tzung des Behandlers zu ersetzen? Die
spezielle Definition der Kategorien anhand der Daten des Assessment
w�re dann erst der zweite Schritt.

Durch eine alleinige Ber�cksichtigung der im Assessment erhobenen
Daten gehen jedoch zahlreiche wichtige und nur schwer genau erfassbare
Informationen des klinischen Eindrucks, wie ihn der Behandler vom
Patienten gewinnt, verloren. Um auf diese Informationen bei der
Entscheidung nicht zu verzichten, ist meiner Ansicht nach eine
Kombination aus klinischem Eindruck des Behandlers mit den
objektivierbaren und zum Teil auch zus�tzlichen Daten eines
umfassenden geriatrischen Assessments n�tig. Dies geschieht in der
geriatrisch-onkologischen Konferenz.

\subsubsection{Einsch�tzung des Behandlers und Einteilung nach
  Konferenzbeschluss}

Die H�ufigkeiten der verschiedenen Kategorien der pers�nlichen
Einsch�tzung und der Konferenzentscheidung sind auch hier
�hnlich. Tabelle~\ref{tab:konferenz_personal} im Ergebnisteil 
zeigt die Einsch�tzung der individuellen Patienten. Diese ist
zwischen der pers�nlichen Einsch�tzung und der Konferenzentscheidung
verbl�ffend �hnlich. Abbildung~\ref{fig:konferenz_personal} stellt dies
und die individuellen Abweichungen zur besseren Anschaulichkeit und
Analyse grafisch dar. Dabei entspricht jeder Punkt einem Patienten.

Bei 337 Patienten gibt es sowohl eine pers�nliche Einsch�tzung der
behandelnden �rzte als auch einen Konferenzbeschluss. Insgesamt zeigt sich
eine bemerkenswerte �bereinstimmung. In \percentAgreementKP{} der
F�lle stimmen pers�nliche Einsch�tzung und
Konferenzbeschluss �berein. Bemerkenswert ist au�erdem, dass die
beobachteten Abweichungen lediglich benachbarte Einsch�tzungen
betreffen. So gibt es keinen Fall, der als nicht
therapief�hig eingesch�tzt wird, der nach Konferenzbeschluss
uneingeschr�nkt therapief�hig eingesch�tzt wird. Umgekehrt gibt es
lediglich einen Fall, der von der Konferenz als nicht therapief�hig
eingesch�tzt wurde, der in der pers�nlichen Einsch�tzung als
uneingeschr�nkt therapief�hig bezeichnet wurde. Fast alle 43 vom
Behandler als nicht therapief�hig eingesch�tzten Patienten sind auch laut
Konferenzbeschluss nicht therapief�hig. Lediglich bei einem Patienten
stellte die Konferenz noch eine eingeschr�nkte Behandlungsf�higkeit
fest.

In der Abstufung zwischen eingeschr�nkt und uneingeschr�nkt
therapief�hig kam es zu folgenden Abweichungen. 17 Patienten wurden im
Konferenzbeschluss hochgestuft, waren also im pers�nlichen Assessment
nur eingeschr�nkt therapief�hig eingesch�tzt worden, von der Konferenz
jedoch als uneingeschr�nkt therapief�hig. 14 Patienten wurden im
Konferenzbeschluss heruntergestuft, wurden also pers�nlich als
uneingeschr�nkt therapief�hig, von der Konferenz jedoch als
eingeschr�nkt therapief�hig klassifiziert.
Damit ist der positiv pr�diktive Wert des pers�nlichen Assessments wie
folgt f�r die einzelnen Kategorien: 88\% f�r uneingeschr�nkt
therapief�hig, 77\% f�r eingeschr�nkt therapief�hig und 100\% f�r
nicht therapief�hig.

Wie gerade ausgef�hrt betraf eine
Fehlklassifikation jedoch immer nur die benachbarte Kategorie.
Abbildung~\ref{fig:konferenz_personal} illustriert in der Gegen�berstellung von
Konferenzentscheidung und per�nlicher Einsch�tzung die dargelegte
Situation. Dabei sind besonders gut die zwei m�glichen Arten von
Fehlern sichtbar.
Entweder werden Patienten durch das pers�nliche Assessment
als zu gut oder zu schlecht eingestuft. Beide Arten von Fehlern kommen
vor, 32 Patienten wurden von der Konferenz schlechter eingesch�tzt,
also und 24 Patienten besser. Wie bereits erw�hnt kommt es bei
einem Klassifikationsfehler mit nur einer Au�nahme zu einer
Fehlklassifikation in eine benachbarte Kategorie. Bei nur einem
Patienten kam es in der Konferenz zu einer nicht
behandlungsf�hig-Einsch�tzung, die zuvor der Behandler als
uneingeschr�nkt behandlungsf�hig eingesch�tzt hat.

An dieser bemerkenswerten Pr�zision der pers�nlichen Einsch�tzung ohne
Vorliegen eines Assessments muss sich ein statistisches
Vorhersagemodell messen lassen. 

Abbildung~\ref{fig:beschlussverteilung} zeigt die Altersverteilung der
Patienten und den Konferenzbeschluss. Dabei f�llt die ungef�hre
Gleichverteilung der laut Beschluss nicht therapief�higen Patienten
auf. Bei der Verteilung der als eingeschr�nkt therapief�hig
beurteilten Patienten ist ein Trend zum h�heren Lebensalter
erkennbar. Aber auch jenseits der 80 Lebensjahre kommen noch
uneingeschr�nkt therapief�hig eingesch�tzte Patienten vor. Das
best�tigt die insgesamt h�here Bedeutung des biologischen Alters
gegen�ber dem chronologischen Alter.

Der Vorteil einer geriatrisch-onkologischen Konferenz ist die
Kombination der objektivierbaren Daten des umfassenden 
geriatrischen Assessments mit der klinischen Erfahrung sowohl der
behandelnden �rzte als auch eines erfahrenen Geriaters. Nur in dieser
Zusammenarbeit ist es m�glich, nicht in der routinem��igen klinischen
Untersuchung erkennbare Probleme aufzudecken und somit zus�tzlich
ber�cksichtigen zu k�nnen. Nur in dieser Zusammenarbeit ist es m�glich
erhobene Befunde aus dem klinischen Kontakt und aus dem Assessment
zu deuten, gem�� ihrer Bedeutung einzuordnen und zus�tzlich
unter Einbeziehung der zugrundeliegenden Erkrankung zu bewerten.

So ist zum Beispiel ein Patient, der wegen eines rasch progredienten
aggressiven Lymphoms bettl�gerig ist und somit h�ufig als nicht
therapief�hig eingesch�tzt w�rde, aufgrund der m�glichen raschen
Remission trotzdem in einer Konferenzentscheidung unter
Ber�cksichtigung der Biologie der Erkrankung als uneingeschr�nkt
therapief�hig beurteilt worden. Die Ber�cksichtigung dieser
zahlreichen zus�tzlichen Informationen, macht eine
geriatrisch-onkologische Konferenz meiner Meinung nach zum
Goldstandard der Beurteilung der Therapief�higkeit
geriatrisch-onkologischer Patienten, an der sich jedes andere
Verfahren messen lassen muss.

\subsubsection{Einteilung nach Vorschlag von Balducci und nach
  Konferenzbeschluss}



\subsection{Zusammenfassung}

Meiner Meinung nach stellt zum heutigen Zeitpunkt die
Therapieentscheidung aufgrund eines geriatrisch-onkologischen
Konferenzbeschlussess zu treffen die beste Behandlungm�glichkeit
geriatrisch-onkologischer Patienten dar. 
Diese Hypothese m�sste anhand prospektiver Studien gepr�ft werden.

\section{Modell der Konferenz als logistische Regression}

\subsection{Beschreibendes Modell}
\label{subsec:model_expl}

\subsubsection{Therapief�higkeit}

Gleichung~\ref{eq:logistic1} und Tabelle~\ref{tab:logistic1}
beschreiben eine logistische Regression zur Beurteilung der
Therapief�higkeit. Die abgesch�tzte Wahrscheinlichkeit in die Kategorie nicht
therapief�hig klassifiziert zu werden ist dabei gegeben durch
Gleichung~\ref{eq:logistic1-prediction}.

\begin{equation}
\label{eq:logistic1-prediction}
\hat{P}(no\_go) = \frac{e^{\beta_0 + \beta_1 X_1 + \beta_2 X_2 + \epsilon}}{1 + e^{\beta_0 + \beta_1 X_1 + \beta_2 X_2 + \epsilon}}
\end{equation}

Daraus ergibt sich durch Einsetzen der Koeffizienten und der nach
Gleichung~\ref{eq:scale} umgerechneten Pr�diktoren zum Beispiel f�r
einen Barthel-Index von 100 einen MMST-Wert von 30 eine
Wahrscheinlichkeit von 1\%, also eine fast sichere (99\%)
Wahrscheinlichkeit einer Therapief�higkeit. Die negativen
Koeffizienten bedeuten eine Erh�hung der Wahrscheinlichkeit, als nicht
therapief�hig klassifiziert zu werden bei fallendem Barthel-Index und
MMST-Wert. Der h�here Wert des Koeffizienten des Barthel-Index
beschreibt einen gr��eren Einfluss, also eine gr��ere �nderung der
Wahrscheinlichkeit bei ge�ndertem Barthel-Index. F�llt zum Beispiel
der Barthel-Index eine Standardabweichung (22,6) auf 75, so steigt bei
gleich bleibendem MMST von 30 die Wahrscheinlichkeit, als nicht
therapief�hig klassifiziert zu werden auf 5\%. F�llt der MMST um eine
Standardabweichung (3,0) auf 27, so steigt bei gleich bleibendem
Barthel-Index von 100 die Wahrscheinlichkeit lediglich auf 2\%. Fallen
beide Pr�diktoren gem�� ihrer Standardabweichung gleichm��ig, besteht
bei einer Standardabweichung niedrigerem Barthel-Index (75) und MMST
(27) nur eine Wahrscheinlichkeit von 12\% als nicht therapief�hig
klassifiziert zu werden. Erst bei fast zwei Standardabweichungen
niedrigerem Barthel-Index von 60
und MMST von 25 steigt die Wahrscheinlichkeit auf 45\% als nicht
therapief�hig klassifiziert zu werden. Ab diesem Punkt l��t jede
weitere Absenkung sowohl
isoliert des Barthel-Index auf 55, oder auch des MMST auf 24 die
Wahrscheinlichkeit auf jeweils 54\% steigen und damit dann in die Kategorie
nicht behandlungsf�hig klassifiziert zu werden.

Die Auswahl der Pr�diktoren im Modell deckt sich mit der Erfahrung
sowohl aus der klinischen Praxis als auch der
geriatrisch-onkologischen Konferenz, dass das Ausma� an
Pflegebed�rftigkeit und der kognitive Status die Kriterien waren,
einzelne Patienten nicht therapief�hig einzusch�tzen. Ausnahmen davon
w�ren eine Demenz wegen behandelbarer Hirnmetastasen oder
Bettl�gerigkeit aufgrund starker Tumorschmerzen.

\subsubsection{Ausma� der Therapief�higkeit}

Analog Gleichung~\ref{eq:logistic1-prediction} werden die abgesch�tzen
Wahrscheinlichkeiten gem�� Gleichung~\ref{eq:logistic2} und
Tabelle~\ref{tab:logistic2} mit den nun insgesamt 6 Pr�diktoren nach
Gleichung~\ref{eq:logistic2-prediction} berechnet.

\begin{equation}
\label{eq:logistic2-prediction}
\hat{P}(slow\_go) = \frac{e^{\beta_0 + \beta_1 X_1 + \beta_2 X_2 \dots \beta_6 X_6 + \epsilon}}{1 + e^{\beta_0 +
    \beta_1 X_1 + \beta_2 X_2 \dots \beta_6 X_6 + \epsilon}}
\end{equation}

Daraus ergibt sich durch Einsetzen der Koeffizienten und der nach
Gleichung~\ref{eq:scale} umgerechneten Pr�diktoren zum Beispiel f�r
die optimalen Testergebnisse und ein Alter von 65 Jahren eine
Wahrscheinlichkeit von 0,3\%, also eine fast sichere (99,7\%)
Wahrscheinlichkeit einer uneingeschr�nkten Therapief�higkeit.
Den betragsm��ig gr��ten Einfluss hat f�r das Ausma� der
Therapief�higkeit mit Abstand die Mobilit�t, ausgedr�ck durch den
Tinetti-Index. Als n�chstes folgt schon das Alter und dann die
Komorbidit�t. Das Merkmal Alleinlebend kommt dann vor dem kognitiven
Status und dem Ern�hrungszustand.

Die Koeffizienten des Tinetti-Tests, des MMST, und des MNA sind
jeweils negativ, schlechtere (niedrigere) Testergebnisse erh�hen also
die Wahrscheinlichkeit nur eingeschr�nkt behandlungsf�hig zu
sein. Umgekehrt sind die Koeffizienten des Alters, des Charlson-Index
und des Merkmals Alleinlebend positiv. Dies bedeutet das zunehmendes
Alter, zunehmende Komorbidit�t und das vorhandene Merkmal alleinlebend,
alle die Wahrscheinlichkeit eingeschr�nkt behandlungsf�hig zu sein,
erh�hen. 

Wird lediglich der Wert des Tinetti-Tests um eine Standardabweichung
(7,9) erniedrigt, steigt die Wahrscheinlichkeit lediglich auf 2\% an.
Verschlechtert man jedoch auch jeden anderen Pr�diktor um jeweils eine
Standardabweichung steigt die Wahrscheinlichkeit eingeschr�nkt
behandelbar klassifiziert zu werden auf 71\% und damit �ber die
Schwelle von 50\%. Dieses Modell klassifiziert also im Vergleich zum
vorangehenden deutlich fr�her in die andere Kategorie.
Interessant ist in diesem Zusammenhand die Beobachtung, dass in der
geschilderten Situation, wenn alle Pr�diktoren um eine
Standardabweichung verschlechtert sind, das Merkmal Alleinlebend
bestimmend f�r die Klassifikation. Wird in dieser Situation dieses
Merkmal auf nicht allein lebend ge�ndert, sinkt die Wahrscheinlichkeit
auf 48\%, das hei�t es w�rde die Kategorie uneingeschr�nkt
therapief�hig gew�hlt werden. Dies deckt sich mit den in der Konferenz
gemachten Beobachtungen. Der soziale Status ist bei einem Patienten
der in allen Tests optimal abschneidet von keinerlei Bedeutung. In
Grenzsituationen jedoch, kann die Tatsache, dass ein Patient allein
lebt, oder zusammen mit seinem Partner jedoch entscheidend sein f�r
die Klassifikation in die jeweilige Kategorie.

Proportional Odds Asumption, Widerlegung der multinomialen Regression???

IADL, TUG und DIA-S sind in keinem der erkl�renden Modelle zu
finden. IADL hat eine hohe Kolinearit�t mit dem Barthel-Index, wurde
daher nur aus Gr�nden der besseren Erkl�rbarkeit weggelassen. Im
vorhersagenden Modell verbessert der Einschluss des IADL-Wertes das
Modell soweit, dass er mit eingeschlossen wurde. Der TUG-Wert weist
sogar die h�chste beobachtete Kolinearit�t mit einem
Rangkorrelationskoeffizienten von -0,86 zum Tinetti-Test auf. Dies ist
nicht verwunderlich, das beide Tests die Mobilit�t beurteilen. Die
erkl�rt, warum der Wert des TUG in den Modellen keine Ber�cksichtigung
findet.

Das Depressionsassessment, gemessen �ber die DIA-S ist ebenfalls in
keinem Modell enthalten. Ein sehr hoher Wert in der DIA-S l�st eine
sofortige intensive Mitbetreuung durch die Kollegen der
Psychoonkologie aus, war jedoch auch in den Konferenzen kein Grund,
eine Therapieempfehlung zu �ndern.

\subsubsection{Vergleich beider Modelle}

Wenn man beide Modelle vergleicht f�llt auf, dass in der
Klassifikation der Therapief�higkeit nur zwei Pr�diktoren, n�mlich der
Barthel-Index und der MMST enthalten sind, also nur die Tatsache und
das Ausma� der Pflegebed�rftigkeit erfasst durch den Barthel-Index und
das Vorliegen eine Demenz, angezeigt durch den MMST, relevant
sind. Die Klassifikation von behandlungsf�hig und nicht
behandlungsf�hig also zum Beispiel auch unabh�ngig vom Alter ist.

Das Ausma� der Therapief�higkeit, also eingeschr�nkt therapief�hig
oder uneingeschr�nkt therapief�hig ist hingegen von sechs Pr�diktoren
abh�ngig. Alle Bereiche, die durch das umfassende geriatrische
Assessment abgedeckt werden, finden sich auch in dem Modell
wieder. Der Barthel-Index ist in diesem Modell zwar nicht enthalten,
spielt jedoch trotzdem eine wichtige Rolle, da es ja erst zum Einsatz
kommt, wenn allgemeine Therapief�higkeit unter Benutzung des
Barthel-Index festgestellt wurde. Trotz mehr Pr�diktoren und damit
mehr Flexibilit�t und weniger Datens�tze (fehlende nicht
behandlungsf�hige Patienten) ist die G�te der Regression des Ausma�es
der Therapief�higkeit, gemessen �ber Akaikes Informationskriterium mit
198 deutlich schlechter als die des Modells der Therapief�higkeit (AIC
134). Das bedeutet, dass bei der Beurteilung des Ausma�es der
Therapief�higkeit durch die Konferenz mehr nicht erfasste Parameter
Eingang in die Entscheidung finden, als allein bei der Feststellung
der Therapief�higkeit. Beispiele aus den Konferenzen sind Immobilit�t 
aufgrund der Tumorerkrankung, die anders gewertet werden, als andere,
konkurrierende Ursachen. Oder auch ein schlechter Ern�hrungsstatus
wegen eines �sophagus- oder Magenkarzinoms. Oder alleinlebenden
Patienten, die jedoch bestm�glich sozial eingebunden sind, da zum
Beispiel die Kinder im gleichen Haus leben. Oder eine differenziertere
Bewertung der Komorbidit�t, als es allein durch den Zahlenwert des
Charlson-Index m�glich ist. All dies sind Beispiele, dass die
Entscheidung der Konferenz in diesen speziellen F�llen anders lautet,
als es die Assessmentwerte eigentlich anzeigen w�rden.

\subsection{Vorhersagendes Modell}

\subsubsection{Therapief�higkeit}

Tabelle~\ref{tab:prediction_no} beschreibt die Vorhersagequalit�t des Modells
zwischen nicht behandlungsf�hig und behandlungsf�hig zu unterscheiden. Wie
beschrieben ergibt sich eine �bereinstimmung von 94\%. Bei einem Cut-off
der Wahrscheinlichkeit bei 0,5 ergibt sich eine Sensitivit�t von 88\%
und eine Spezifit�t von 95\%. Der Cut-off ist willk�hrlich gew�hlt, von ihm
h�ngen jedoch Sensitivit�t und Spezifit�t zusammen, variiert man diesen
ver�ndern sich auch Sensitivit�t und Spezifit�t. Um diesen Zusammenhang
zu veranschaulichen ist die ROC-Grafik sinnvoll, da sie den Zusammenhang
zwischen Sensitivit�t und Spezifit�t abh�ngig vom Cut-off-Wert grafisch
darstellt. Die Fl�che unter der Kurve (AUC, area under the curve) kann
als G�te eines diagnostischen Test herangezogen werden. Ein optimaler
Test mit perfekter Vorhersagequalit�t besitzt eine maximale AUC von 1.

\subsubsection{Ausma� der Therapief�higkeit}

Tabelle~\ref{tab:prediction_slow} beschreibt die Vorhersagequalit�t des Modells
zwischen nicht behandlungsf�hig und behandlungsf�hig zu unterscheiden. Wie
beschrieben ergibt sich eine �bereinstimmung von 83\%. Bei einem Cut-off
der Wahrscheinlichkeit bei 0,5 ergibt sich eine Sensitivit�t von 0,84
und eine Spezifit�t von 0,82.

\section{Simulation der Konferenz als Web-basierte Abfrage}

Wie gezeigt wurde beinhaltet die Einsch�tzung eines geriatrisch-onkologischen
Patienten durch den Behandler eine gute Einsch�tzung einer
onkologischen Behandlungsf�higkeit, wie sie oftmals vergleichbar auch
eine geriatrisch-onkologische Konferenz trifft. Trotzdem gibt es
subklinische oder auch pers�nlich anders gewertete Einschr�nkungen von
Patienten, die dadurch allein durch die pers�nliche Einsch�tzung
anders eingesch�tzt werden. Dies folgt in m�glicherweis erh�hter
Toxizit�t und damit Gefahr f�r die Patienten bei besserer Einsch�tzung
durch den Behandler (9\% der Patienten). Bei
schlechterer Einsch�tzung durch den Behandler (7\% der Patienten) wird
zwar Toxizit�t vermieden, jedoch den Patienten auch wirksame Therapien
vorenthalten. Bei insgesamt 17\% der Patienten kommt die hier
durchgef�hrte geriatrisch-onkologische Konferenz also zu
Therapie-ver�ndernden Anderseinsch�tzungen. Es bleibt in prospektiven
Studien zu pr�fen, ob sich dieser Unterschied in nachweisbaren
Parametern, wie zum Beispiel messbarer Toxizit�t oder messbarer
Wirksamkeit der Therapie �bersetzt.

Vielerorts ist ein erster Schritt die Einf�hrung eines umfassenden
geriatrisch-onkologischen Assessments, ohne jedoch die M�glichkeit
einer geriatrisch-onkologischen Konferenz. Wie in
Abschnitt~\ref{subsec:vorhersagendes_modell} gezeigt wurde ist die
Vorhersagequalit�t des statistischen Modells der
geriatrisch-onkologischen Konferenz vergleichbar mit der
Vorhersagequalit�t der pers�nlichen Einsch�tzung. Trotzdem kann das
statistische Modell Zusaztinformationen liefern, da es die Ergebnisse
des umfassenden geriatrischen Assessments miteinbezieht, und sich
daher auf andere Daten bezieht, als der Behandler. Im Diskrepanzfall
kann also der Behandler sich die Testergebnisse genauer anschauen und
im Bedarfsfall sogar mit einem Geriater konferieren um diesen
speziellen Fall zu besprechen. Aus diesem Grund werden die hier
vorgestellten vorhersagenden Modelle auf der Internetseite der
Geriatrischen Onkologie der Kliniken
Essen-Mitte ver�ffentlicht. Dort k�nnen die Parameter aus dem
Assessment eingetragen werden, woraufhin eine Klassifikation durch die
Modelle in nicht behandlungsf�hig, eingeschr�nkt behandlungsf�hig und
uneingeschr�nkt behandlungsf�hig vorgenommen wird. Als Vergleich wird
ebenfalls eine Klassifikation aufgrund des im Jahr 2000 von Balducci
und Kollegen gemachten Vorschlags vorgenommen.

\section{Ausblick}

Zahlreiche Studien belegen somit die Potenz eines geriatrischen
Assessments sowohl die Prognose von Patienten als auch die Toxizit�t
von Behandlungen vorherzusagen. Inwiefern es m�glich ist, mit den
zus�tzlichen Informationen des Assessments durch Anpassung der
Behandlung diese wichtigen Parameter zu beeinflussen, ist derzeit
unklar und sollte in prospektiven Studien untersucht werden.

Genau wie der Vorschlag der Einteilung, den Balducci vor jetzt schon
16 Jahren gemacht hat, sind die hier vorgestellten Ergebnisse
ebenfalls als Vorschlag zu verstehen. Es bleibt anhand von
prospektiven Studien zu pr�fen, welcher Ansatz die beste Einsch�tzung
bietet. Reicht die pers�nliche Einsch�tzung durch den behandelnden
Arzt aus? Ist eine pers�nliche Einsch�tzung durch einen
geriatrisch-onkologisch geschulten Experten n�tig? Ist eine
Klassifikation allein aufgrund von Parametern des umfassenden
geriatrischen Assessments m�glich. Welche Parameter sollten dies sein?
Welche Grenzwerte sind f�r die jeweiligen Tests anzusetzen? Oder ist
doch eine Kombination aus beiden Ans�tzen notwendig, bei dem ein
umfassendes geriatrisches Assessment durchgef�hrt wird, auf dessen
Grundlage dann in der Diskussion zwischen Behandler, Geriater,
Physiotherapeuten, Erotherapeuten, Pflegekr�ften unter Einbeziehung
einer Vielzahl zus�tzlicher, nicht so einfach zu erhebender
Informationen, eine Klassifikation vorgenommen wird?

Sich daran direkt anschlie�ende Fragen beinhalten, zum Beispiel
retrospektiv, oder aber auch prospektiv im Sinne einer Registerstudie,
welche Patienten nach der Konferenzentscheidung welche Therapie mit
welcher Toxizit�t erhalten haben. Wurden die Empfehlungen der
Konferenz also im klinischen Alltag umgesetzt? Welche Therapien wurden
bei nicht therapief�hig eingestuften Patienten trotzdem gestartet?
Werden diese Therapien schnell wegen �berm��iger Toxizit�t
abgebrochen? Oder wom�glich gut vertragen?

\section{Empfehlungen f�r die t�gliche Praxis}

\begin{itemize}

\item Alle Patienten, 65 Jahre oder �lter, f�r die eine potentiell
  nebenswirkungsreiche Therapie geplant ist (Chirurgie,
  Strahlentherapie, medikament�se Therapie), sollen auf das Risiko von
  funktionellen Einschr�nkungen und geriatrischen Syndromen gescreent
  werden (G8 oder VES-13 oder in Kombination).

\item Bei station�ren Patienten oder auff�lligem Screening soll ein
  umfassendes geriatrisches Assessment (CGA) durch ein erfahrenes
  multiprofessionelles Team stattfinden.

\item Die Ergebnisse dieses Assessments sollen in einer Konferenz von
  einem multiprofessionellen Team bestehend aus Ergotherapie, Pflege,
  Geriatrie und Onkologie besprochen werden.

\item Basierend auf der Konferenz sollen Ma�nahmen eingeleitet werden,
  um funktionelle, psychische, soziale oder medizinische Probleme zu bessern.

\item Basierend auf der Konferenz soll die weiterf�hrende Therapie
  angepasst an die individuelle Leistungsf�higkeit und Lebenssituation
  geplant werden.

\item Die Ergebnisse der Assessments und der Konferenz sollen dem
  Patienten, den Angeh�rigen und dem Hausarzt mitgeteilt werden.

\end{itemize}

% FILE: main.tex  Version 2.1
% AUTHOR:
% Universit�t Duisburg-Essen, Standort Duisburg
% AG Prof. Dr. G�nter T�rner
% Verena Gondek, Andy Braune, Henning Kerstan
% Fachbereich Mathematik
% Lotharstr. 65., 47057 Duisburg
% entstanden im Rahmen des DFG-Projektes DissOnlineTutor
% in Zusammenarbeit mit der
% Humboldt-Universitaet zu Berlin
% AG Elektronisches Publizieren
% Joanna Rycko
% und der
% DNB - Deutsche Nationalbibliothek

\chapter{Diskussion}

\section{Pers�nliche Einsch�tzung des Behandlers}

\section{Modell der Konferenz als logistische Regression}

\section{Simulation der Konferenz als Web-basierte Abfrage}

\section{Wie geht es weiter, weitere Analysen, Studie planen}

\section{Empfehlungen f�r die t�gliche Praxis}

* Alle Patienten, 65 Jahre oder �lter, f�r die eine potentiell nebenswirkungsreiche Therapie geplant ist (Chirurgie, Strahlentherapie, medikament�se Therapie), sollen auf das Risiko von funktionellen Einschr�nkungen und geriatrischen Syndromen gescreent werden (G8 oder VES-13 oder in Kombination).
* Bei station�ren Patienten oder auff�lligem Screening soll ein umfassendes geriatrisches Assessment (CGA) durch ein erfahrenes multiprofessionelles Team stattfinden.
* Die Ergebnisse dieses Assessments sollen in einer Konferenz von einem multiprofessionellen Team bestehend aus Ergotherapie, Pflege, Geratrie, Onkologie, Palliativmedizin, Sozialdienst besprochen werden.
* Basierend auf der Konferenz sollen Ma�nahmen eingeleitet werden, um funktionelle, psychische, soziale oder medizinische Probleme zu bessern.
* Basierend auf der Konferenz soll die weiterf�hrende Therapie angepasst an die individuelle Leistungsf�higkeit und Lebenssituation geplant werden.
* Das geriatrische Assessment soll zur Verlaufsbeoabachtung w�hrend und nach der Therapie regelm��ig wiederholt werden, um neue Probleme zu entdecken und auch ggf. eine Besserung bestehender Probleme zu dokumentieren
* Die Ergebnisse der Assessments und der Konferenzen sollen dem Patienten, den Angeh�rigem und dem Hausarzt mitgeteilt werden.

% usw.
 
% Anhang*
\appendix
% FILE: appendixA.tex  Version 2.1
% AUTHOR:
% Universit�t Duisburg-Essen, Standort Duisburg
% AG Prof. Dr. G�nter T�rner
% Verena Gondek, Andy Braune, Henning Kerstan
% Fachbereich Mathematik
% Lotharstr. 65., 47057 Duisburg
% entstanden im Rahmen des DFG-Projektes DissOnlineTutor
% in Zusammenarbeit mit der
% Humboldt-Universitaet zu Berlin
% AG Elektronisches Publizieren
% Joanna Rycko
% und der
% DNB - Deutsche Nationalbibliothek

\chapter{Anhang}

\section{Abk�rzungen}

\begin{longtable}[l]{p{2.5cm}p{10cm}}
CGA & Comprehensive geriatric assessment, umfassendes geriatrisches Assessment\\
CLL & Chronisch lymphatische Leuk�mie\\
ECOG & Eastern Cooperative Oncology Group\\
MMST & Mini Mental Status Test\\
WHO & World Healt Organisation\\
PS & Performance Status\\
\end{longtable}

\section{Quelldateien}

\url{https://github.com/ertls/geriatric_oncology}

\section{Quelldokumente}

\begin{figure}[htbp]
  \centering
  \includegraphics[width=\textwidth]{Anhang/Information.pdf}
  \caption{Einverst�ndniserkl�rung}
  \label{fig:einverstaendnis}
\end{figure}

\begin{figure}[htbp]
  \centering
  \includegraphics[width=\textwidth]{Anhang/Ethik.pdf}
  \caption{Votum der Ethikkommission}
  \label{fig:ethik}
\end{figure}

\begin{figure}[htbp]
  \centering
  \includegraphics[width=\textwidth]{Anhang/Barthel.pdf}
  \caption{Barthel-Index}
  \label{fig:barthel}
\end{figure}

\begin{figure}[htbp]
  \centering
  \includegraphics[width=\textwidth]{Anhang/IADL.pdf}
  \caption{IADL-Skala}}
  \label{fig:iadl}
\end{figure}

\begin{figure}[htbp]
  \centering
  \includegraphics[width=\textwidth]{Anhang/TUG.pdf}
  \caption{TUG-Test}
  \label{fig:tug}
\end{figure}

\begin{figure}[htbp]
  \centering
  \includegraphics[width=\textwidth]{Anhang/Tinetti.pdf}
  \caption{Tinetti-Test}
  \label{fig:tinetti}
\end{figure}

\begin{figure}[htbp]
  \centering
  \includegraphics[width=\textwidth]{Anhang/MMST.pdf}
  \caption{MMS-Test}
  \label{fig:mmst}
\end{figure}

\begin{figure}[htbp]
  \centering
  \includegraphics[width=\textwidth]{Anhang/DemTect.pdf}
  \caption{DemTect}
  \label{fig:demtect}
\end{figure}

\begin{figure}[htbp]
  \centering
  \includegraphics[width=\textwidth]{Anhang/Charlson.pdf}
  \caption{Charlson-Score}
  \label{fig:charlson}
\end{figure}

\begin{figure}[htbp]
  \centering
  \includegraphics[width=\textwidth]{Anhang/GDS.pdf}
  \caption{GDS}
  \label{fig:gds}
\end{figure}

\begin{figure}[htbp]
  \centering
  \includegraphics[width=\textwidth]{Anhang/DIA.pdf}
  \caption{DIA-Skala}
  \label{fig:dia}
\end{figure}

\begin{figure}[htbp]
  \centering
  \includegraphics[width=\textwidth]{Anhang/MNA.pdf}
  \caption{MNA}
  \label{fig:mna}
\end{figure}

% usw

% Verzeichnisse in appendixA

% Lebenslauf*
%\include{cv}

% Selbstständigkeiterklärung
%\include{declaration}

% Dokumentenende
\end{document}
