% FILE: main.tex  Version 2.1
% AUTHOR:
% Universität Duisburg-Essen, Standort Duisburg
% AG Prof. Dr. Günter Törner
% Verena Gondek, Andy Braune, Henning Kerstan
% Fachbereich Mathematik
% Lotharstr. 65., 47057 Duisburg
% entstanden im Rahmen des DFG-Projektes DissOnlineTutor
% in Zusammenarbeit mit der
% Humboldt-Universitaet zu Berlin
% AG Elektronisches Publizieren
% Joanna Rycko
% und der
% DNB - Deutsche Nationalbibliothek

\chapter{Geriatrische Onkologie}

\section{Einführung}

Kein Onkologe würde einen einjährigen Säugling chemotherapeutisch
wegen einer malignen Erkrankung behandeln. In dieser speziellen
Lebensphase des Menschen, charakterisiert durch eine besonderen
Verletzlichkeit der sozialen, psychischen und physischen Situation der
kleinen Patienten, bedarf es neben onkologischer auch spezieller
pädiatrischer Expertise. In Deutschland ist dafür eine eigener
Facharzt zuständig.
Der Großteil der Patienten fast jedes tätigen Onkologen und
Hämatologen besteht aus Patienten, die sich ebenfalls in einer
speziellen Lebensphase befinden. Auch die späte Lebensphase ist
gekennzeichnet durch eine besondere Verletzlichkeit bezogen auf die
soziale, psychische und physische Situation der älteren Patienten. Um
dieser speziellen Situation bedarf es ebenfalls neben
hämatologisch/onkologischer Expertise auch spezieller geriatrischer
Expertise. Versucht wird dies in Deutschland aktuell durch die von
allen beteiligten Fachgesellschaften empfohlenen, jedoch nur sehr
spärlich durchgesetzte Zusammenarbeit von Hämatologen/ Onkologen mit
Geriatern.

Die Geriatrische Onkologie ist ein Teilgebiet der
Hämatologie/Onkologie und beschäftigt sich mit den Besonderheiten des
älteren Menschen mit einer Krebserkrankung.

Klinisches Ziel der Geriatrischen Onkologie ist eine bestmögliche,
individuell auf die allgemeine (physische und psychische)
Belastungsfähigkeit und die Lebenssituation angepasste Behandlung der
Patienten. Dies beinhaltet die interdisziplinäre Betreuung der
Patienten unter Einbeziehung  verschiedener Berufsgruppen, der
Angehörigen sowie natürlich der Patienten selbst bis hin zur Steuerung
der spezifischen antitumorösen Therapie.

Wissenschaftliches Ziel der Geriatrische Onkologie ist speziell die
Erforschung der Besonderheiten von Krebserkrankungen älterer Menschen
und der physiologischen Besonderheiten des älteren Organismus in Bezug
auf Krebsentstehung, Krebsentwicklung und Reaktion auf antitumoröse
Therapien. In den meisten Zulassungsstudien und auch
Therapieoptimierungsstudien der Onkologie sind im Vergleich mit dem
klinischen Alltag ältere Patienten dramatisch
unterrepräsentiert. Daher ist ein allgemeines Ziel der Geriatrischen
Onkologie, den Anteil älterer Patienten in onkologischen Studien zu
erhöhen und auch zusätzliche Studien durchzuführen, die sich mit den
Besonderheiten der Behandlung der älteren Patienten beschäftigt. Dabei
ist entscheidend zwischen dem chronologischen Alter, dem
physiologischen Alter und begleitenden geriatrischen Syndromen und
anderen Komorbiditäten zu unterscheiden.

\section{Begriffserklärungen}

Der geriatrische Patient ist durch die Deutsche Gesellschaft für
Geriatrie definiert als Patient mit Geriatrie-typische Multimorbidität
und höherem Lebensalter, wobei explizit eine kalendarische
Altersangabe vermieden wird. Dies lässt bewusst viel Spielraum für
Interpretationen.

Der ältere Mensch wird in vielen geriatrisch-onkologischen Arbeiten ab
65 Jahren definiert. Diese kalendarische Definition habe ich hier
aufgrund der für ein Screening besser geeigneten festen Definition
übernommen.  In dieser Arbeit werden die beiden Begriffe älterer
Patient und geriatrischer Patient synonym verwenden und meinen
Patienten die 65 Jahre oder älter sind.

Umfassendes geriatrisches Assessment, geriatrisches Assessment und
Assessment werden synonym gebraucht und meinen ein in der Geriatrie
gebräuchliches Erhebungsverfahren mit meist in Kombination von
verschiedenen Tests. Abgekürzt wird dies mit CGA, comprehensive
geriatric assessment.

Im folgenden Text wird das Wort signifikant auschließlich im
statistischen Sinn gebraucht, mit einem wie in der Medizin üblichen
Signifikanzniveau von 0,05.

\section{Geschichte}

In der Geriatrie werden seit langer Zeit Assessments genutzt [17932635].
Bereits 1946, interessanterweise in der gleichen Zeit, in der erstmals
systematisch toxische und oftmals tödliche Chemotherapien in der
Behandlung von Krebserkrankungen eingesetzt wurden, wurde die moderne
Form eines Assessments in der Medizin durch Marjory Warren in England
entwickelt [6368651]. Ziel von Warren war es, mit Hilfe eines
Assessments Patienten aus einem heterogenen Patientengut zu
selektionieren, die von einer medizinischen Intervention profitieren
würden. Bis heute ist dies der prinzipielle Nutzen von verschiedenen
Assessments, aus einer heterogenen Gruppe von Patienten bezüglich
bestimmter Merkmale homogenere Untergruppen zu selektionieren, die
dann Ausgangspunkt bestimmter Interventionen oder weiterer
Untersuchungen sind. Ganz gemäß dem Aphorismus von Osler von 1892: "If
it were not for the great variability among individuals medicine might
as well be a science and not an art."

Die erste mir bekannte Veröffentlichung erschien 1973, in der von
Williams und Kollegen der Nutzen eines ambulanten geriatrischen
Assessments in Bezug auf die Vermeidung einer Pflegeheimeinweisung
gezeigt wurde [4202389]. Der Nutzen eines stationären geriatrischen
Assessments wurde erstmals 1984 nachgewiesen [6390207]. Dort wurden
Patienten die systematisch mit Hilfe eines geriatrischen Assessments
in einer spezialisierten geriatrischen Abteilung behandelt wurden mit
Patienten aus der Normalversorgung verglichen. Zum
Entlassungszeitpunkt waren die Patienten die im Rahmen der
Normalversorgung behandelt wurden mehr auf fremde Hilfe angewiesen und
wurden häufiger in ein Pflegeheim eingewiesen. Die Behandlung unter
Einbeziehung eines geriatrischen Assessments verbessert also die
Selbständigkeit und kann Pflegeheimeinweisung vermeiden.

\section{Epidemiologie}

Die Lebenserwartung der Menschen hat im vergangenen Jahrhundert stetig
zugenommen. Da die meisten bösartigen Erkrankungen mit steigendem
Alter zunehmen, stieg damit auch die Inzidenz von
Krebserkrankungen. Hinzu kommen bei älteren Patienten außerdem eine
erhöhte Inzidienz an relevanten Komorbiditäten und funtionellen
Defiziten. Anders als bei jüngeren Tumorpatienten sind ältere
Tumorpatienten sowohl tumorbiologisch als auch bezüglich relevanter
Komorbiditäten und funktioneller Probleme eine erheblich heterogenere
Gruppe.

Bezogen auf den einzelnen Menschen steigt die Krebsinzidenz in den
letzten Lebensdekaden exponentiell. Mehr als 60 \% aller neu
diagnostizierter Krebserkrankungen und sogar mehr als 70 \% der
Krebstodesfälle betreffen Patienten, die 65 Jahre oder älter sind. Die
altersangeglichene Krebsinzidenz ist bei den größer gleich 65 jährigen
10 fach höher, die altersangeglichene Krebsterblichkeit sogar 16 fach
höher verglichen mit den unter 65 jährigen [18528470, 25642321, 4].

Die moderne evidenzbasierte Medizin basiert auf der Durchführung von
randomisierten, kontrollierten Studien. In diesen Studien waren bis
vor kurzem ältere Patienten von vornherein ausgeschlossen. Dies ändert
sich zwar in letzter Zeit, so ist allein das Alter in aktuellen
Studien kein Ausschlusskriterium mehr, jedoch sind auch heute noch die
älteren Patienten in den großen klinischen Studien, inklusive der
Zulassungsstudien der meisten etablierten Chemotherapien,
unterrepräsentiert. Daher sind die Ergebnisse dieser Studien streng
genommen nicht auf diese ältere Patientengruppe übertragbar, was
jedoch mangels Alternativen im klinischen Alltag durchgängig getan
werden muss. Zusätzlich ist das Verhältnis der verschiedenen
Altersgruppen im klinischen Alltag genau umgekehrt wie in den
klinischen Studien. Der Großteil unserer Patienten sind älter als 64
Jahre. (eigene Daten 2014: 62 \% 64 Jahre, mediane Alter 67 Jahre)
Daher muss jeder klinisch tätige internistische Hämatologe und
Onkologe sich diesem Problem bewusst sein und somit auch ein
geriatrischer Onkologe sein.

Ein Ziel der Geriatrischen Onkologie ist es, diese Situation zu
bessern. So müssen in den Studien mehr ältere Patienten eingeschlossen
werden, um die Anwendbarkeit der Daten im klinischen Alltag zu
ermöglichen. Zusätzliche Studien sollten außerdem die Besonderheiten
der Krebserkrankungen älterer Patienten und die Besonderheiten der
älteren Patienten in Reaktion auf die spezifische Therapie
untersuchen. Gerade das verspricht nicht nur Erkenntnisse für die
tägliche Arbeit, sondern auch zusätzliches Verständnis des Phänomens
einer Tumorerkrankung.

Altern ist ein hoch komplexer biologischer Vorgang. Sowohl angeborene
und erworbene genetische Veränderungen als auch vergangene und
aktuelle äußere Einflüsse spielen eine bedeutende Rolle. Es resultiert
eine enorme interindividuelle Verschiedenartigkeit der Krebserkrankung
und auch des erkrankten Patienten. Damit ist die ohnehin schon enorme
Komplexität einer Krebserkrankung eines Menschen noch erheblich
gesteigert. Daher weisen ältere Patienten eine größere
interindividuelle Schwankungsbreite bezüglich Organfunktion und
Wiederstandsfähigkeit bezogen auf äußere Einflüsse auf, als jüngere
Patienten.

Ein geriatrisches Assessment ist der pragmatische Versuch durch einige
wenige, einfach durchzuführende Tests eine Dreiteilung der Population
in voll behandlungsfähig, eingeschränkt behandlungsfähig und nicht
behandlungsfähig eingeschätzte Patienten vorzunehmen.

(Noch weitere Ausführung der Bedeutung dieser Kategorien, z.B. nicht
behandlungsfähig bedeutet nicht, einem Patienten eine Chemotherapie
nicht zu verabreichen, wenn dies für die Symptomkontrolle wichtig ist)

Sowohl die Dreiteilung als auch die Bezeichnung der Kategorien ist
willkürlich, jedoch weit verbreitet. Ausgehend von dieser
Kategorisierung ist zu bestimmen, inwiefern dies relevant ist, und
vorhersagenden Charakter auf den Verlauf der Erkrankung oder aber auch
auf die Wiederstandsfähigkeit gegenüber toxischer Therapien ist. Dies
ist jedoch nicht Gegenstand der vorliegenden Arbeit.

Warum sind geriatrische Assessments nötig? Jede medizinische Therapie,
oder weiter gefasst, jede medizinische Intervention soll dem Patienten
helfen. Symptome lindern, Funktionalität wiederherstellen, kausal
heilen, Krebswachstum eindämmen. Dies ist allerdings immer auch
zwingend mit Wirkungen auf den Organismus verbunden, die diesem Ziel
nicht entsprechen, unerwünsche Wirkungen, oder allgemein auch
Nebenwirkungen genannt. Manchmal sind diese harmlos, sie können jedoch
auch lebensbedrohlich sein. Die moderne Medizin beschäftigt sich
zwingend auch mit der Einschätzung und bestenfalls Vermeidung und
Behandlung dieser Nebenwirkungen von medizinischen Therapien. Es gibt
sogar Erhebungen, die feststellen, dass ein Großteil aller
Krankenhausaufenthalte zur Behandlung von unerwünschten
Arzneimittelwirkungen nötig sind.

In der Onkologie, in der es oft um Leben und Tod geht, wo Krankheiten
oft schrecklicher sind, sind auch deren Therapien, sei es chirurgisch
(weiter ausführen: entstellende Chirurgie in den Anfängen der
Krebsbehandlung beim Mammakarzinom), strahlentherapeutisch, oder
klassich internistisch-onkologisch, also medikamentös durch
Chemotherapien immer potenter und dadurch auch immer reicher an
Nebenwirkungen. Dies geht sogar soweit, dass manche Chemotherapien,
aber auch andere, gerade erwähnte Therapieen zum Tod von Patienten
führen. Die große Frage ist jetzt: Wie können wir die Risiken unserer
potentiell tödlichen Therapien einschätzen? In experimentellen,
kontrollierten Studien sehen wir die Wirkung, aber auch die
prozentuale Rate und die Schwere der Nebenwirkungen der getesteten
Therapien auf die Studienpopulation. Der Patient, aber auch ich als
Behandler, möchte nicht wissen, wie das Medikament in der
Studienpopulation gewirkt hat, und in welchem Prozentsatz es welche
Nebenwirkungen verursacht hat, im Ernstfall, wie viele Patienten an
dieser Behandlung gestorben sind. Der Patient, die Angehörigen und die
Behandler wollen wissen, wie der potentielle Nutzen und das
potentielle Risiko dieser Therapie, bezogen auf das vor uns sitzende
Individuum aussieht. Dies ist kein spezielles Anliegen der
Geriatrischen Onkologie, sondern der gesamten Medizin.

Seit über 50 Jahren wird in der Onkologie der Karnofsky-Index (KI)
eingesetzt [Karnofsky,1948], einer 10 Punkte-Skala um den
Allgemeinzustand eines Patienten einzuschätzen. Jüngere gängige Scores
sind der ECOG und der WHO-PS, einer 5 Punkte-Skala entsprechen, und
annähernd deckungsgleich. Validiert wurden alle Skalen an jüngeren
Tumorpatienten. Es ist ein von Onkologen häufig in der täglichen
Praxis verwendetes Instrument, um die Therapiefähigkeit von Patienten
einzuschätzen. Darüber hinaus ist ein ausreichend hoher Score die
Voraussetzung zur Aufnahme in eine klinische Studie. Es besteht nur
eine sehr geringe Korrelation mit einem geriatrischen Assessment
[9552069]. ECOG und KI erfassen die akuten Einschränkungen des
funktionellen Zustands durch die Tumorerkrankung. Alltagsrelevante
Einschränkungen, die schon vor der Tumorerkrankung vorhanden waren und
bei älteren Tumorpatienten häufig bestehen werden in diesen Skalen
jedoch nur unzureichend dargestellt. Um diese auch für die
tumorspezifische Behandlung relevanten Einschränkungen zu erkennen,
sind die geriatrischen Funktionscores entwickelt worden. Daher sollte
meiner Meinung nach zukünftig das sehr viel präzisere geriatrische
Assessment den Performance Score vor Aufnahme in eine klinische Studie
ablösen.

Anhand des über 3000 geriatrische Patienten umfassenden
IN-GHO-Registers (Initiative Geriatrische Häamtologie und Onkologie)
ist belegt, dass Mortalität und Therapieerfolg bei älteren
Tumorpatienten nicht aufgrund des Lebensalters, sondern auf Basis des
geriatrischen Assessments vorausgesagt werden können [noch nicht voll
publiziert].

Die Prognose eines Patienten mit einer Krebserkrankung wird bestimmt
von der Art der Krebserkrankung und den Organfunktionen des
Patienten. Dabei spielt weniger das chronologische Alter als das
biologische Alter eine Rolle, also die körperliche und psychische
Leistungsfähigkeit sowie Komorbiditäten. Routinemäßig werden Patienten
in jedem Bereich der Medizin von Ärzten in “biologisch älter” und
“biologische jünger” eingeteilt. Diese persönliche und subjektive
Einteilung hat jedoch keine prognostische Relevanz
[11527298]. Erstaunlicherweise können therapierelevante Defizite im
funktionellen Leistungsvermögen allein durch die Anamnese und
körperliche Untersuchung eines einzelnen Arztes nicht entdeckt
werden. Ein interdisziplinär durchgeführtes geriatrisches Assessment
kann unter zu Hilfe nahme von standardisierten Testverfahren häufige
im Alter auftretende Funktionsstörungen erkennen.

Das drängende Problem der Geriatrischen Onkologie ist nun, dass
geriatrische Patienten, oder besser allgemein ältere Patienten,
bezüglich der Organfunktion, als auch anderer funktioneller Parameter
eine heterogenere Gruppe darstellen, als jüngere Patienten. Dies macht
diese Population für Studien so ungeeignet und ist die Ursache, dass
für diese klinisch viel wichtigere Gruppe so viel weniger
wissenschaftlich belastbare Daten vorliegen.

Ein geriatrisches Assessment ist in diesem Dilemma eine diagnostische
Maßnahme, ein Versuch, das vor uns sitzende Individuum besser
einzuschätzen. Anhand einfacher und schnell zu erhebender Parameter
einen Eindruck zu gewinnen, in welchem Bereich des möglichen Spektrums
dieser Patient bezüglich Organfunktion und Funktionalität steht. Ein
systematisches Assessment kann dabei auch therapeutisch beeinflussbare
Funktionseinschränkungen erkennen, die in der normalen ärztlichen
Anamnese und Untersuchung unauffällig bleiben.

Ziele des geriatrischen Assessments:
- Prognose (Abschätzung der Lebenserwartung)
- Abschätzung der Wahrscheinlichkeit des Auftretens von tumorbedingten
Symptomen
- Abschätzung der Einschränkung der Lebensqualität durch die Krebserkrankung
- Abschätzung der Einschränkung der Lebensqualität durch die Therapie
- voraussichtliche Verträglichkeit der Therapie
- Möglichkeit die Therapie zeitgerecht und vollständig durchzuführen

Die Annahme, und in bisher keinster Weise zufriedenstellend
beantwortete Hypothese ist nun, dass ausgehend einer Einschätzung, die
Anhand eines geriatrischen Assessments getroffen wird, Nutzen und
Risiko der beabsichtigten Therapie gefolgert werden kann. Dies ist die
zentrale Frage: Kann ein geriatrisches Assessment, und die folgende
Konferenzeinschätzung eines individuellen Patienten die Toxizität
einer beabsichtigten Therapie und damit auch indirekt ihren
Gesamtnutzen voraussagen? (Weiter ausführen: optimal wäre eine Studie,
die anhand der drei Klassifikationsgruppen getrennt Überleben- und
Toxizitätsdaten nennt.)

Eine prospektive, randomisierte Studie müsste sechs armig sein. Nach
Assessment, müsste in jeder Gruppe randomisiert werden. Bei
uneingeschränkt therapiefähigen Patienten zwischen voller und
reduzierter Therapie. Bei eingeschränkt therapiefähigen Patienten
ebenfalls. Bei nicht therapiefähigen Patienten zwischen keiner
Therapie und einer reduzierten Therapie. Dann könnte man die
verschiedenen Armen bezüglich Überleben, Ansprechen und Toxizität
vergleichen. Eine Realisierung solch einer Studie ist sehr schwierig.
Alternativ könnte man schauen, retrospektiv, oder aber auch prospektiv
als Beobachtungsstudie, welche Patienten nach der
Konferenzentscheidung welche Therapie erhalten haben und dies
beschreiben und versuchen Trends zu entdecken. Welche Therapien wurden
bei nicht therapiefähig eingestuften Patienten gestartet? Werden diese
Therapien schnell wegen übermäßiger Toxizität abgebrochen? Oder
womöglich gut vertragen? Alles wichtige Fragen, denen man weiter
nachgehen sollte.

In meiner Arbeit ist es jedoch isoliert der Entscheidungsprozess der
geriatrisch-onkologsichen Konferenz und die daraus resultierende
Beurteilung. Therapiefähigkeit wird also nicht absolut bestimmt oder
beobachtet. Sondern der Beschluss spiegelt die Meinung der Anwesenden
in der Konferenz wieder. Durch diese Einschränkung ist jedoch die
Klassifizierung der Konferenzentscheidung analog eines diagnostischen
Tests möglich.

Inspirierend für diese Arbeit war für mich die in meiner Zeit in der
Hämatologie häufig benutzte Risiko-Einschätzung von neu
diagnostizierten AML-Patienten. Aufgrund standarmäßig erhobener
Parameter kann dort anhand statistischer Daten einer großen Datenbank
von AML-Patienten die Wahrscheinlichkeit einer Remission auf der einen
Seite und das Risiko von Frühsterblichkeit auf der anderen Seite,
ermittelt werden. Trotz fehlender prospektiver Validierung ist dies
meiner Meinung nach ein sinnvolles Hilfsmittel in der in
Grenzsituationen oft schwierigen Entscheidung zwischen einer
potentiell kurativen aber auch potentiell tödlichen
Induktionschemotherapie oder doch hin zu einer besser verträglichen
palliativen Therapie. Genau diese Hilfe kann auch das ebenfalls noch
nicht prospektiv validierte Hilfsmittel eines geriatrischen
Assessments mit anschließender geriatrisch-onkologischer Konferenz
bieten. Praktisches Ziel dieser Arbeit ist eine automatische
Verknüpfung der Daten des Assessments mit einer Konferenzentscheidung
als Hilfmittel im Internet.

\section{Sinnhaftigkeit eines Assessments in der Geriatrie}

Das umfassende geriatrische Assessment (CGA) erfasst multidiziplinär
verschiedene  altersassoziierte physiologsiche und psychologische
Faktoren, die Gesundheit und Krankheit älterer Patienten mehr
beinflussen könnten, als allein das chronologische Alter.

Die ersten Anwendungsgebiete von Assessments in der Geriatrie wurden
bereits im geschichtlichen Überblick gestreift. Bereits 1973 wurde
nachgewiesen, dass bei ambulanten Patienten durch ein geriatrisches
Assessment vorausgesagt werden kann, für welche Patienten eine
Pflegeheimeinweisung vermeidbar ist [4202389]. Der Nutzen des
stationären geriatrischen Assessments wurde 11 Jahre später
nachgewiesen [6390207]. Patienten, die systematisch mit Hilfe eines
geriatrischen Assessments in einer spezialisierten geriatrischen
Abteilung behandelt wurden, waren zum Entlassungszeitpunkt weniger auf
fremde Hilfe angewiesen und wurden seltener in ein Pflegeheim
eingewiesen. Dabei konnte also gezeigt werden, dass die Behandlung
unter Einbeziehung eines geriatrischen Assessments und eines
geriatrischen Behandlungsteams die Selbständigkeit verbessert und die
Wahrscheinlichkeit einer Pflegeheimeinweisung senkt. Zusätzlich ist
ein systematisches geriatrisches Assessment in der Lage
Funktionseinschränkungen zu erkenn, die der routinemäßigen ärztlichen
Anamnese und körperlichen Untersuchung entgehen [9288009]

Daher sind in der Geriatrie verschiedene Assessmentsysteme Teil der
Regelversorgung. Nicht nur zu initialen Erfassung von Problemen,
sondern auch in der Verlaufsbeurteilung der Behandlung, ist diese Art
der multiprofessionellen Erfassung der verschiedenen Parameter neben
klassischer ärztlicher Anamnese und körperlicher Untersuchung im
Routinebetrieb geriatrischer Abteilungen etabliert.

\section{Sinnhaftigkeit eines Assessments in der Onkologie}

Der Beginn der modernen Onkologie liegt in der ersten Hälfte des zwanzigsten Jahrhunderts mit dem Beginn der systematischen Verabreichung von Chemotherapien.  Durch die Verabreichung von Zellgiften konnten erstmals in der Behandlung von akuten Leukämien und aggressiven Lymphomen, später dann auch einzelner solider Tumore wie das Hodenkarzinomen Patienten mit im Körper verteilter Erkrankung geheilt werden. Wegen der zum Teil dramatischen Nebenwirkungen dieser Therapien die oftmals tödlich endeten, beschäftigten sich die in diesem Bereich tätigen Ärzte praktisch vom Beginn an auch mit der Frage, welchen Patienten eine solche Therapie zugemutet werden kann. Für welche Patienten also die Möglichkeit besteht eine Verbesserung des Erkrankung zu erreichen, ohne ihn durch die Therapie umzubringen. Die Entscheidung, welche Patienten für eine zytostatische Therapie in Frage kamen, traf der Behandler, im besten Fall das Behandlungsteam.
Diese Situation hat sich bis heute in der Mehrzahl der modernen Onkologien weltweit nicht geändert. Durch moderne supportive Therapien und Erfahrung aus kontrollierten Studien in der Verabreichung der Zytostatika sind die Therapien zwar sehr viel besser verträglich und auch sicherer als in den Anfängen der Onkologie. Trotzdem handelt es sich weiterhin um die kontrollierte Verabreichung von Zellgiften, die potentiell tödlich verlaufen kann. Jeder Onkologe macht im Verlauf seines Berufslebens die Erfahrung, Patienten nicht an der Krebserkrankung sterben zu sehen, sondern auch an den Folgen seiner eingeleiteten Therapie.

Zusätzlich hat sich die Altersgrenze durch die modernen supportiven Möglichkeiten sowohl zur Verbesserung der Akutverträglichkeit, als auch der Folgeprobleme, durch Wachstumsfaktoren und moderne Antibiotika immer weiter nach hinten verschoben. Hochdosischemotherapien mit autologen und allogenen Stammzelltransplantationskonzepten wurden zu Beginn dieser Therapien nur Patienten verabreicht, die jünger als 50 Jahre waren. Mit zunehmender Erfahrung mit diesen Behandlungsformen ist heute das Alter zum Beispiel bei autologen Stammzelltransplantation bis auf 75 Jahre in der Routineanwendung verschoben.
Durch die Weiterentwicklung in der Onkologie hin zu einer immer zielgerichteteren Therapie mit neuen Substanzgruppen der Antikörper, die gegen Merkmale des Tumors oder auch des Tumorumgebenen Gewebes gerichtet sind, sind andere Nebenwirkungsprofile als die der klassischen Zellgifte zu beobachten. Neue Medikamente, die direkt zum Teil hoch spezifisch in den Tumorzellstoffwechsel eingreifen haben ebenfalls ein völlig neues Nebenwirkungsprofil. Auch die aktuell erstmals breit angewendete erfolgreiche Beeinflussung des Immunsystems durch PD1- und PDL1-Antikörper bringt neue Herausforderungen mit sich.
Mit der besseren Verträglichkeit einiger Substanzklassen ist außerdem, wie bei Imatinib oder Ibrutinib eine lebenslange kontinuierliche Verabreichung möglich. Völlig ungeklärt ist in diesem Zusammenhang meist die Interaktion der Pharmakologie und Pharmakokinetik dieser Substanzen mit den bei älteren Patienten in zunehmender Zahl eingesetzten Substanzen, um die geriatrisch auftretenden Syndrome und Probleme zu behandeln.

Der erste mir bekannte Versuch einer systematischen Erfassung des
Allgemeinzustandes und der Leistungsfähigkeit, erfolgte durch
Karnofsky während der Behandlung von Lungenkarzinompatienten mit
Senfgas, einem chemischem Kampfstoff aus dem ersten Weltkrieg. Initial
nutzte Karnofsky den schon damals so genannten Performance Status
nicht zur Auswahl der Patienten, die eine Therapie erhalten konnten,
sondern zur Messung einer Besserung des Allgemeinzustandes mit dem
Ziel der Einschätzung der Wirksamkeit der Therapie. Bei dem von ihm
beschriebenen Performance Status der bis heute als Karnofsky-Index in
der Onkologie benutzt wird, handelt es sich um eine 10 Punkte Skala,
die von 0 \%, tod, bis 100 \% keine Krankheitszeichen reicht.
Benutzt wird Karnofsky-Index weiterhin, um eine Änderung des Allgemeinzustands des Patienten während einer Chemotherapie zu dokumentieren, aber auch um Patienten auszusortieren, die sich nicht für eine spezifische Therapie eignen.
Zwei weitere, heutzutage deckungsgleiche Skalen sind der ECOG oder WHO Performance Status, kurz ECOG oder PS. Dabei handelt es sich um eine 5 Punkte Skala, die von 0, tot, bis 5 keine Krankheitseinschränkungen reicht. Der Vorteil dieser Skala liegt in der Einfachheit und damit direkt möglichen Einschätzung des Behandlers nach dem ersten Patientenkontakt. Auch in Therapiestudien hat die Einschätzung des Allgemeinzustandes eines Patienten über den Performance Status große Bedeutung. So ist praktisch in allen onkologischen Studien ein PS von größer 2, bei vielen Studien auch größer 1, Ausschlusskriterium. Auch in der täglichen Arbeit eines Onkologen ist diese Einschätzung allgegenwärtig. Patienten mit einem PS von 4 werden sicherlich nur in sehr außergewöhnlichen Fällen chemotherapeutisch behandelt. Vorstellbar wäre zum Beispiel ein junger Patient, der durch eine akute Leukämie in diesem schlechten Allgemeinzustand ist, und bei dem man damit rechnet, dass eine rasche Reduktion der Tumorlast dies rasch bessert.
Aber auch bei Patienten mit einem PS von drei müssen gute Gründe für eine spezifische Behandlung vorliegen. Eine intensive Kombinationstherapie würde diesen Patienten wohl auch kein Onkologe verabreichen.
Dass diese alleinige Einschätzung aufgrund des Performance Scores, sei es durch Karnofsky oder ECOG/WHO nicht optimal ist, weiß jeder klinisch tätige Arzt. Aufgrund fehlender Alternativen ist es weiterhin ein breit eingesetzter Maßstab, entweder explizit bei der Auswahl von Studienpatienten oder auch implizit in der Einschätzung des Behandlers. 
Außerdem sind zahlreiche zusätzliche Umstände denkbar, die nicht unmittelbar den Performance Score beeinflussen, jedoch trotzdem Einfluss auf die Verträglichkeit und Tolerierbarkeit einer spezifischen Therapie haben. So ist zum Beispiel belegt, dass bei älteren Krebspatienten der Performance Score unabhängig von den Komorbiditäten ist [9552069].
Zusätzlich würde sich ein frühe Demenz ebenfalls nicht im Performance Score wiederfinden, hätte jedoch gerade bei allein lebenden Patienten eine erhebliche Bedeutung bei der Planung und Durchführung der Chemotherapie.

Ein umfassenden geriatrisches Assesssment bei Krebspatienten nutzt verschiedene in der Geriatrie etablierte und validierte Tests um eine Aussage über die folgenden Bereiche machen zu können:
* Die Fähigkeit des individuellen Patienten eine Chemotherapie zu tolerieren
* Das Risiko eines frühen Todes einzuschätzen
* Die Wahrscheinlichkeit von chemotherapieassoziierter Toxizität und sich davon wieder zu erholen
* Identifikation von beeinflussbaren Risikofaktoren, um das Behandlungsergebnis zu verbessern
Ohne ein strukturiertes Assessment laufen die genannten Abschätzungen intuitiv im Behandler ab, lediglich auf der Grundlage der beiden Parameter biologisches Alter und klinische Einschätzung.
Es bleiben prospektive Daten abzuwarten, ob ein geriatrisches Assessment allein, oder mit anschließender interdisziplinärer geriatrisch-onkologischer Konferenz, diese Abschätzung besser vornehmen kann, als der Behandler. Sicher ist jedoch, dass für eine systematische Untersuchung in Studien das Assessment besser geeignet ist.

\subsection{Therapiesicherheit}

Für die Planung einer onkologischen oft multimodalen Therapie ist die Einschätzung der Situation, kurativ oder palliativ, sowie die Abschätzung der Prognose besonders wichtig aber oft auch sehr schwierig. Bekannt ist, dass das chronologische Alter viel weniger Einfluss auf die Prognose eines Patienten hat, als allgemeinhin angenommen. Von größerer Bedeutung als das Alter sind eher körperliche und psychische Leistungsfähigkeit (Zitat, in IM FOCUS ONKOLOGIE 2 nicht voll publizierte Honecker Abstracts aus den IN-GHO-Registry).

In einer Studie an 348 Patienten, 70 Jahre oder älter, die eine aufgrund verschiedener Tumore eine Chemotherapie erhielten, war das Risiko eines frühen Todes innerhalb der ersten sechs Monate der Behandlung korreliert mit einem schlechten Ernährungsstatus, gemessen durch das MNA (siehe 3.1.7), männlichem Geschlecht und eine eingeschränkte Mobilität, angezeigt durch einen langsamen TUG-Test (siehe 3.1.6). Das Risiko eines frühen Todes innerhalb der ersten sechs Monate war nicht korreliert mit dem Performance Status (ECOG), ADL, IADL, MMST oder GDS (siehe 3.1). Interessant ist außerdem, dass eine durch das Assessment ausgelöste primäre Dosisreduktion der Chemotherapie nicht mit dem frühen Tod korrelierte [22508806].

Das mittlere Überleben von 566 Patienten, 75 Jahre oder älter, mit einem nicht-kleinzelligen Lungenkarzinom, die innerhalb einer Studie behandelt wurden war 30 Wochen. Eine multivariate Analyse für prognostische Faktoren ergab einen Einfluss vom IADL-Score (siehe 3.1.3) und der Lebenqualität, gemessen mit dem EORTC QLQ-C30, sowie der Anzahl der Orte der Metastasierung. Ein allgemeinerer Score als der IADL, der ADL-Score und auch die Komorbidität gemessen durch den Charlson-Score ergab keine zusätzlichen Informationen bezüglich der Prognose der Patienten [16192578].

Eine andere Studie ebenfalls mit Patienten mit einem nicht-kleinzelligen Lungenkarzinom, 70 Jahr oder älter, zeigte eine Korrelation von Ergebnissen eines geriatrischen Assessments mit neuropsychiatrischer Toxizität und der Fähigkeit eine voll dosierte Chemotherapie zu tolerieren, nicht jedoch mit der Lebensqualität [21252061].

In einer Studie an Patienten, 70 Jahre oder älter, mit aggressiven Lymphomen die eine dosismodifizierte CHOP-Therapie erhielten, konnte das Geriatrische Assessment, speziell ADL und IADL einen frühen Tod vorhersagen [25503576]. Speziell in der Situation der aggressiven Lymphome ist das führende Dilemma der geriatrischen Onkologie besonders deutlich sichtbar. Durch aggressivere Therapien auch der älteren Patienten konnte die Prognose dieser Patienten in den letzten Jahren deutlich gesteigert werden [15867204, 11790067]. Mit der Aggressivität der Therapie steigt jedoch auch die Toxizität vor allem auch bei älteren Patienten bis hin zur in Kaufnahme von therapieassoziierten Todesfällen. In einer Serie von Patienten mit einem aggressivem Lymphom, 65 Jahre oder älter, konnte gezeigt werden, dass ein geriatrischen Assessment effektiver in der Vorhersage von Ansprechen, prgressionsfreies Überleben und Gesamtüberleben als die klinische Einschätzung der Behandler war.

Bei Brustkrebspatientinnen, 65 Jahre oder älter, die eine palliative Erstlinienchemotherapie erhielten konnte ein prätherapeutisch durchgeführtes geriatrisches Assessment Grad 3/4 Toxizität der Chemotherapie voraussagen. Polypharmazie war ebenfalls ein unabhängiger Prädiktor von Grad 3/4 Toxizität durch die Chemotherapie. Dabei ist unklar ob dies auch unabhängig von der im Assessment erfassten Komorbidität durch zum Beispiel Arzneimittelinteration ein Faktor ist, oder lediglich, weil Patietinnen mit höherer Komorbidität auch mehr Medikamente einnahmen [24314824].

\subsection{Zusätzliche therapeutische Möglichkeiten}

Bereits im geschichtlichen Überblick wurde beschrieben, wie ein geriatrisches Assessment zusätzlich zur ärztlichen Anamnese und körperlichen Untersuchung in der Lage ist, Funktionseinschränkungen bei älteren Patienten zu erkennen [9288009]. Das Erkennen von Problemen ist der erste Schritt, diese Funktionseinschränkungen sind zusätzlich in den meisten Fällen durch spezifische Therapien, medizinisch, physiotherapeutisch ider ergotherapeutisch behandelbar und damit verbesserungsfähig. Aber auch Probleme die nicht therapeutisch angehbar sind, wie zum Beispiel die soziale Situation des Patienten, sind oftmals im multidisziplinären Team zumindest verbesserungsfähig.

\subsection{Vorteile für den Behandler}

Jede Sekunde sterben Menschen wegen verschiedenster beeinflussbarer und auch unbeeinflussbarer Gründe. In der Onkologie ist das Sterben aufgrund der palliativen Situation der meisten unserer stationären Patienten ein großer Teil der täglichen Arbeit. Selbst in dieser schwierigen Situation kann die moderne Onkologie zusammen mit der Palliativmedizin Großartiges für die Patienten leisten. Für mich persönlich ist das Sterben meiner Patienten aufgrund ihrer Erkrankung ein normaler Teil meiner Arbeit, bei dem ich sowohl den Patienten als auch ihren Angehörigen zum Teil erheblich helfen kann.
Stirbt ein Patient jedoch nicht an seiner Erkrankung sondern an der Therapie ist die Belastung des gesamten Behandlungsteams deutlich erkennbar. Auch ist das Leid der Patienten in dieser Situation meist höher, da palliative, linderne Therapien wie eine Sedierung seltener und Patient und Angehörige belastende intensivmedizinische Therapien häufiger zum Einsatz kommen.

In der folgenden Arbeit werden erstens die Daten eines umfassenden
geriatrischen Assessments von über 300 Patienten vorgestellt, die an
dieser Klinik 2014 und 2015 routinemäßig bei allen stationären
Patienten, die 65 Jahre oder älter waren, vor Einleitung einer
spezifischen Therapie erhoben wurden. Zweitens wird analysiert,
inwiefern die Entscheidung, die anhand der erhobenen Daten in einer
geriatrisch-onkologischen Konferenz getroffen wird, von der
persönlichen Einschätzung des behandelnden Arztes und von der
Klassifizierung durch das Assessment übereinstimmt oder
abweicht. Drittens wird geprüft inwiefern ein statistisches Modell in
der Lage ist, anhand der Assessmentdaten die Klassifizierung, die
innerhalb einer geriatrisch-onkologischen Konferenz getroffen wird, zu
modellieren.
